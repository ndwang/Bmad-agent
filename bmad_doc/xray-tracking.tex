\chapter{Tracking of X-Rays}
\label{c:xray.track}

\bmad can track both charged particles and X-rays. This chapter deals
with X-rays. Charged particles are handled in chapter~\sref{c:charged.track}.

%-------------------------------------------------------------------------
%-------------------------------------------------------------------------
\section{Coherent and Incoherent Photon Simulations}
\label{s:coher.incoher}

\index{coherent tracking}\index{incoherent tracking}
\bmad can track photons either \vn{coherently} or \vn{incoherently}.
In both cases, the photon has a transverse electric field 
\begin{equation}
  (E_x, E_y)
\end{equation}
$E_x$ and $E_y$ are complex and therefore have both amplitude and phase information. When photons
are tracked incoherently, the phase information is not used for calculating X-ray intensities.

In addition to coherent and incoherent tracking, partially coherent simulations can be done by using
sets of photons with the photons in any one set treated as coherent and the photons between sets
being treated as incoherent.

%-------------------------------------------------------------------------
\subsection{Incoherent Photon Tracking}
\label{s:incoher}

In a simulation with incoherent photons, some number of photons,
$N_0$, will be generated and the i\Th photon ($i = 1, \ldots, N_0)$
will have a initial ``electric field'' components $E_{x0}(i),
E_{y0}(i)$ assigned to it. The field amplitude $E_0$ will be
$\sqrt{E_{x0}^2 + E_{y0}^2}$.

At some an observation point, the power $S$ per unit area falling on
some small area $dA$ due to either $x$ or $y$ component of the
electric field is
\begin{equation}
  S_{x,y} = \frac{\alpha_p}{N_0 \, dA} \, \sum_{j \in \text{hits}} E_{x,y}^2(j)
  \label{panda1}
\end{equation}
where $\alpha_p$ is a constant that can be chosen to fit the
simulation against experimental results, and the sum is over photons
who intersect the area. The factors of $N_0$ and $dA$ in the above
equation make, within statistical fluctuations, $S$ independent of
$N_0$ and, for $dA$ small enough, $S$ will be independent of $dA$ as
it should be. The total power is just $S_x + S_y$.

When traveling through vacuum, the electric field of a photon is a
constant.  As an example, consider a point source radiating uniformly
in $4\pi$ solid angle with each photon having the same initial field
$E_0$. An observation area $dA$ situated a distance $R$ from the
source will intercept $N_0 \, dA / 4 \, \pi \, R^2$ photons which
gives a power of
\begin{equation}
  S_w = \frac{\alpha_p \, E_0^2}{4 \, \pi \, R^2}
\end{equation}
which falls off as $1/R^2$ as expected.

At some places the light may be split into various ``channels''. An
example is Laue diffraction where X-rays can excite the $\alpha$ and
$\beta$ branches of the dispersion surface. Or a partially silvered
mirror where some of the light is reflected and some is transmitted.
In such a case, the probability $P_i$ of a photon traveling down the
$i$\Th channel is
\begin{equation}
  P_i \, \what E_i^2 = \frac{S_i}{S_0}
  \label{rpss1}
\end{equation}
where $S_i$ is the power flowing into channel $i$, $S_0$ is the power
flowing into the junction, and $\what E_i = E_i / E_0$ is the ratio of
the electric field amplitudes of any photon just before and just after being
shunted into the $i$\Th channel. The probabilities must be properly
normalized
\begin{equation}
  \sum P_i = 1
  \label{p1}
\end{equation}

If the ratio of the electric field of any photon just before and just
after being shunted into the $i$\Th channel is not a constant, than
$\what E_i$ must be adjusted so that $\what E_i^2$ is equal to the average of
$\what E_i^2(j)$ for all photons $j$ channeled into channel $i$.

As long as \Eqs{rpss1} and \eq{p1} are satisfied, the choice of the
$P_i$, and $\what E_i$ are arbitrary. This freedom allows simulation to be
optimized for efficiency. For example, In an actual experiment much of
the light can be lost never to reach a detector and be counted. To
decrease the simulation time, simulated photons may be limited to be
generated with a direction to be within some solid angle $\Omega_1$ if
photons with a direction outside this solid angle will not contribute
to the simulation results. In this case, there are two channels.
Channel 1 consists of all photons whose direction is within
$\Omega_1$ and channel 2 is all the other photons. To limit the
photons to channel 1, $P_1$ is taken to be 1 and $P_2$ is taken to be
0. Additionally, if the light, say, is being generated isotropically
from a surface into a $\Omega_0 = 2 \, \pi$ solid angle then
\begin{equation}
  \what E_1 = \sqrt{\frac{\Omega_1}{\Omega_0}}
  \label{roo}
\end{equation}
$\what E_2$ is infinite here but since no photons are generated in channel 2
this is not a problem.

%-------------------------------------------------------------------------
\subsection{Coherent Photon Tracking}
\index{ss:coher}

In a simulation with coherent photons, some number of photons,
$N_0$, will be generated and the i\Th photon ($i = 1, \ldots, N_0)$
will have an initial electric field $E_{x0}(i), E_{y0}(i)$ assigned to
it. These quantities will be complex.

At some an observation point, the field $E$ at
some small area $dA$ due to either $x$ or $y$ component of the
electric field is
\begin{equation}
  E = \frac{\alpha_p}{N_0 \, dA} \, \sum_{j \in \text{hits}} E(j)
  \label{panda2}
\end{equation}
where $\alpha_p$ is a constant that can be chosen to fit the
simulation against experimental results, and the sum is over photons
who intersect the area. In the above equation $E(j)$ is either the $x$
or $y$ component of the electric field as is appropriate. The factors
of $N_0$ and $dA$ in the above equation make, within statistical
fluctuations, $E$ independent of $N_0$ and, for $dA$ small enough, $E$
will be independent of $dA$ as it should be.

When traveling through a a vacuum, the photons travel ballistically in
straight lines. This is justified by using the stationary phase
approximation with Kirchhoff's integral. the electric field of a
photon varies with the propagation length. There is nothing physical
in this and is just a way to make the bookkeeping come out
correctly. As an example, consider a point source radiating uniformly
in $4\pi$ solid angle with each photon having the same initial field
component (either $x$ or $y$) $E_1$.  An observation area $dA$
situated a distance $R$ from the source will intercept $N_0 \, dA / 4
\, \pi \, R^2$ photons and each photon will have a field of $E_1 \, R
\, \exp(i \, k \, R)$ where $k$ is the photon wave number (all photons
must have the same $k$ to be coherent). This gives an electric field
at the observation point of
\begin{equation}
  E = \frac{\alpha_p \, E_1 \, \exp(i \, k \, R)}{4 \, \pi \, R}
\end{equation}
which falls off as $1/R$ as expected.

At a \vn{diffraction_plate} element where diffraction effects are to
be simulated, the following procedure is used:
  \begin{enumerate}
  \item
The electric field components are multiplied by the propagation length $L$:
\begin{equation}
  E \rightarrow E \, L
\end{equation}
The propagation length is reset to zero so that the at the next point
where the propagation length is factored into the electric field the
propagation length will be the length starting at the aperture.
  \item
Depending upon the program, the photon is is either given a random
direction over $2 \, \pi$ solid angle or the photon's direction
is restricted to be within some solid angle chosen to increase
the probability that the photon will make it through some downstream aperture.

If the photon is restricted to some aperture dependent solid angle of area $\Omega$,
the photon's electric field is scaled by
\begin{equation}
  E \rightarrow E \, \frac{\Omega}{4 \, \pi}
  \label{eeo4p}
\end{equation}

  \item
The electric field components are scaled by
\begin{equation}
  E \rightarrow E \, \frac{k}{4 \, \pi \, i} \, (\cos\theta_1 + \cos\theta_2)
  \label{eek4p}
\end{equation}
where $\theta_1$ and $\theta_2$ are the direction cosines of the
incoming and outgoing directions of the photon with respect to the
longitudinal reference axis.
  \end{enumerate}
This algorithm is designed so that the resulting fields at points
downstream from the aperture as computed from a simulation will, to
within statistical errors, be the same as one would get using
Kirchoff's integral. That is, the simulation is constructed to be a
Monte Carlo integration of Kirchhoff's integral.

What is, and what is not considered a place where there are
diffraction effects is dependent upon the problem. For example, there
are diffraction effects associated with light reflecting from a mirror
(or any other object) of finite size. If these effects are important
to the experiment, then a procedure similar to the one above must be
followed. 

At places where there are no diffraction effects a simulation can
treat the photons ballistically or can use the aperture procedure
outlined above. While in theory it is possible to choose what to do, in
practice the aperture procedure increases the number of photons that
must be tracked for a given resolution. Thus, from a practical standpoint
the ballistic alternative should always be used.

As explained in \sref{s:incoher}, at some places the light may be split
into various ``channels''. With coherent photons, the analog to \Eq{rpss1} is
\begin{equation}
  P_i \, \what E_i = \frac{E_i}{E_0}
  \label{rpss2}
\end{equation}
where here $\what E_i$ can be complex to take into account phase shifts.
The same considerations about choosing the $P_i$ and $\what E_i$ apply to
coherent photons as incoherent photons. In particular, $\what E_1$ for the
case of isotropic emission from a surface as in the example in
\sref{s:incoher} (cf. \Eq{roo}) is
\begin{equation}
  \what E_1 = \frac{\Omega_1}{\Omega_0}
\end{equation}

%-------------------------------------------------------------------------
\subsection{Partially Coherent Photon Simulations}

When there is partial coherence the photons must be divided into
sets. All of the photons of a given set are considered coherent while
the photons of different sets are treated incoherently.

The procedure is to track all the photons of one set coherently and
calculate the field using equation \Eq{panda2}. The fields of
different sets are then combined to calculate a power using
\Eq{panda1}.

%-------------------------------------------------------------------------
%-------------------------------------------------------------------------
\section{Element Coordinate System}
\label{s:photon.ele.coords}

\index{element coordinates}
The general procedure for tracking through an element makes use of
\vn{element reference} coordinates (also called just \vn{element}
coordinates). Without any offsets, pitches or tilt (\sref{s:offset}), henceforth
called ``misalignments'', the \vn{element} coordinates are the same
as the \vn{laboratory reference} coordinates (or simply \vn{laboratory}
coordinates) (\sref{s:ref}). The \vn{element} coordinates stay fixed
relative to the element. Therefore, if the element is misaligned, the
\vn{element coordinates} will follow as the element shifts in the
laboratory frame as shown in \fig{f:ele.coord}.

\index{crystal}\index{mirror}\index{multilayer_mirror}
For \vn{crystal} (\sref{s:crystal}), \vn{mirror} (\sref{s:mirror}), and \vn{multilayer_mirror}
(\sref{s:multilayer}) elements, the ``kinked'' reference trajectory through the element complicates
the calculation. For these elements, there are three coordinate systems attached to the element as
shown in \fig{f:photon.ele.coords}. Besides the \vn{element entrance} and \vn{element exit}
coordinates, there are \vn{element surface} coordinates with $z$ perpendicular to the surface
pointing inward.

Tracking a particle through an element is therefore a three
step transformation:
\begin{enumerate}[itemsep=-0.1ex, topsep=-0.4ex]
\item
At the entrance end of the element, transform from the laboratory
reference coordinates to the element's \vn{entrance} or \vn{surface}
coordinates.
\item
Track through the element ignoring any misalignments.
\item
At the exit end of the element, transform from the element coordinates
to the \vn{laboratory} \vn{exit} coordinates.
\end{enumerate}


%-------------------------------------------------------------------------

\begin{figure}[tb]
  \centering
  \includegraphics[width=5in]{photon-ele-coords.pdf}
  \caption[Crystal, Mirror, and Multilayer_Mirror Element Coordinates.]
{The three element coordinate systems for \vn{crystal} (Bragg
configuration), \vn{mirror}, and \vn{multilayer_mirror} elements.  The
origin $\Bf O$ of all three are the same but are shown spread out for
clarity.  $\bfhat n$ is the normal to the element surface.}
  \label{f:photon.ele.coords}.
\end{figure}

%-------------------------------------------------------------------------
\subsection{Transform from Laboratory Entrance to Element Coordinates}

For elements that have a reference orbit kink
(\sref{s:photon.ele.coords}), the element coordinates here are the
\vn{surface} coordinates. Otherwise the element coordinates are
the entrance coordinates.

  \begin{enumerate}
  \item
Apply offsets, pitches and tilt using the formulas in
\sref{s:pos.trans} along with \Eqs{wws}, and \eq{swww}.
  \item
Apply the \vn{tilt} to the electric field (\Eq{ertee}).
  \item
For \vn{crystal}, \vn{mirror}, and \vn{multilayer_mirror} elements
rotate to element surface coordinates.
 \item
Transform the photon's position as if in a drift by a distance $-z$
where $z$ is the photon's longitudinal coordinate. That is, $z$ will
be zero at the end of the transform to element coordinates (remember
that $z$ is the distance from the start of the element
(\sref{s:photon.phase.space})).

\end{enumerate}

%-------------------------------------------------------------------------
\subsection{Transform from Element Exit to Laboratory Coordinate}

The back transformation from element to laboratory coordinates is
accomplished by the transformation
  \begin{enumerate}
  \item
For \vn{crystal}, \vn{mirror}, and \vn{multilayer_mirror} elements
rotate to element from element surface coordinates to element exit coordinates
  \item
Apply the reverse \vn{tilt} to the electric field (\Eq{ertee}).
  \item
Apply reverse offsets, pitches and tilt using the formulas in
\sref{s:pos.trans} along with \Eqs{wws}, and \eq{swww}.
  \end{enumerate}

%-------------------------------------------------------------------------
%-------------------------------------------------------------------------
\section[Mirror and Crystal Element Transformation]
{Transformation for Mirror and Crystal Elements Between 
Laboratory and Element Coordinates}
\label{s:photon.lab.ele}

\index{mirror}\index{crystal}

%-------------------------------------------------------------------------
\subsection{Transformation from Laboratory to Element Coordinates}
\label{s:crystal.trans.le}

\index{z_offset_tot}
With photons, the intensities must also be transformed.  The transformation from the entrance
laboratory coordinates to the entrance element coordinates is:
\begin{enumerate}
\item
Track as in a drift a distance \vn{z_offset_tot}.
\item
\index{x_offset}\index{x_pitch}\index{y_offset}\index{y_pitch}
Apply offsets and pitches: The effective ``length'' of the element is zero (\sref{s:mirror.coords})
so the origin of the element coordinates is the same point around which the element is pitched so
\begin{align}
  x_1    &= x_0 - x_{\text{off}} \CRNO
  p_{x1} &= p_{x0} - (1 + p_{z0}) \, x'_{pitch} \CRNO
  y_1    &= y_0 - y_{\text{off}} \\
  p_{y1} &= p_{x0} - (1 + p_{z0}) \, y'_{pitch} \CRNO
  z_1    &= z_0 + x'_{pitch} \, x_1 + y'_{pitch} \, y_1 \nonumber
\end{align}
where $x_{\text{off}} \equiv \vn{x_offset}$, $x'_{pitch} \equiv \vn{x_pitch}$, etc.
\item
Apply \vn{ref_tilt} and \vn{tilt}:
\begin{align}
  \begin{pmatrix} x_2 \\ y_2 \end{pmatrix} &=
    \bfR (\theta_{tot}) \,   
  \begin{pmatrix} x_1 \\ y_1 \end{pmatrix} \CRNO
  \begin{pmatrix} p_{x2} \\ p_{y2} \end{pmatrix} &=
    \bfR (\theta_{tot}) \, 
  \begin{pmatrix} p_{x1} \\ p_{y1} \end{pmatrix} \label{xyrtxy} \\ 
  \begin{pmatrix} \bfE_{x2} \\ \bfE_{y2} \end{pmatrix} &=
    \bfR (\theta_{tot}) \,   \begin{pmatrix} \bfE_{x1} \\ \bfE_{y1} \end{pmatrix} \nonumber
\end{align}
where $\bfE$ is shorthand notation for
\begin{equation}
  \bfE \equiv E \, e^{i \, \phi}
\end{equation}
with $E$ being the field intensity and $\phi$ being the field phase angle.
In the above equations $\bfR$ is the rotation matrix
\begin{equation}
  \bfR(\theta) = \begin{pmatrix} \cos\theta & \sin\theta \\ -\sin\theta & \cos\theta \end{pmatrix}
\end{equation}
\index{tilt}\index{ref_tilt}\index{tilt_corr}
with $\theta_{tot}$ being 
\begin{equation}
  \theta_{tot}  = 
  \begin{cases}
    \vn{ref_tilt} + \vn{tilt} + \vn{tilt_corr} & \vn{for crystal elements} \\
    \vn{ref_tilt} + \vn{tilt} & \vn{for mirror elements}
  \end{cases}
  \label{tttt}
\end{equation}
The \vn{tilt_corr} correction is explained in \sref{s:crystal.trans}.
\end{enumerate}

%-------------------------------------------------------------------------
\subsection{Transformation from Element to Laboratory Coordinates}
\label{s:crystal.trans.el}

The back transformation from exit element coordinates to exit laboratory coordinates is accomplished
by the transformation
  \begin{enumerate}
  \item
Apply \vn{ref_tilt} and \vn{tilt}: \vn{ref_tilt} rotates the exit laboratory coordinates with
respect to the exit element coordinates in the same way \vn{ref_tilt} rotates the entrance
laboratory coordinates with respect to the entrance element coordinates. The forward and back
transformations are thus just inverses of each other.  With \vn{tilt}, this is not true. \vn{tilt},
unlike \vn{ref_tilt}, does not rotate the output laboratory coordinates.  There is the further
complication in that \vn{tilt} is a rotation about the {\em entrance} laboratory coordinates. The
first step is to express \vn{tilt} with respect to the exit coordinates. This is done with the help
of the $\bfS$ matrix of \Eq{ustt} with $\alpha_t$ given by \Eq{agg}. The effect of the \vn{tilt} can
be modeled as a rotation vector $\Bf e_{in}$ in the entrance laboratory coordinates pointing along
the $z$-axis
\begin{equation}
 \Bf e_{in} = (0, 0, \text{tilt})
\end{equation}
In the exit laboratory coordinates, the vector $\Bf e_{out}$ is
\begin{equation}
  \Bf e_{out} = \bfS \, \Bf e_{in}
\end{equation}
The $z$ component of $\Bf e_{out}$ combines with \vn{ref_tilt} to give
the transformation
\begin{align}
  \begin{pmatrix} x_2 \\ y_2 \end{pmatrix} &=
    \bfR (-\theta_{t}) \,   \begin{pmatrix} x_1 \\ y_1 \end{pmatrix} \CRNO
  \begin{pmatrix} p_{x2} \\ p_{y2} \end{pmatrix} &=
    \bfR (-\theta_{t}) \,   \begin{pmatrix} p_{x1} \\ p_{y1} \end{pmatrix} \\
  \begin{pmatrix} \bfE_{x2} \\ \bfE_{y2} \end{pmatrix} &=
    \bfR (-\theta_{t}) \,   \begin{pmatrix} \bfE_{x1} \\ \bfE_{y1} \end{pmatrix} \nonumber
\end{align}
where $\theta_t$ is $\text{ref_tilt} + \Bf e_{out,z}$. The $x$ and $y$ components
of $\Bf e_{out}$ give rotations around the $x$ and $y$ axes
\begin{align}
  p_{x3} &= p_{x2} - \Bf e_{out,y} \CRNO
  p_{y3} &= p_{y2} + \Bf e_{out,x} \\
  z_3    &= z_2 + x_2 \, \Bf e_{out,y} - y_2 \, \Bf e_{out,x}
\end{align}
  \item
Apply pitches: Since pitches are defined with respect to the entrance laboratory coordinates, they
have to be translated to the exit laboratory coordinates
\begin{equation}
  \bfP_{out} = \bfS \, \bfP_{in}
\end{equation}
where $\bfP_{in} = (x'_{pitch}, y'_{pitch}, 0)$ is the pitch vector in the entrance laboratory frame
and $\bfP_{out}$ is the vector in the exit laboratory frame. The transformation is then
\begin{align}
  p_{x4} &= p_{x3} - \bfP_{out,y} \CRNO
  p_{y4} &= p_{y3} + \bfP_{out,x} \\
  z_4    &= z_3 + x_3 \, \bfP_{out,y} - y_3 \, \bfP_{out,x}
\end{align}
  \item
Apply offsets: Again, offsets are defined with respect to the entrance laboratory coordinates. Like
pitches, the translation is
\begin{equation}
  \bfO_{out} = \bfS \, \bfO_{in}
\end{equation}
where $\bfO_{in} = (x_{\text{off}}, y_{\text{off}}, s_{\text{off}})$ is the offset in the
entrance laboratory frame. The transformation is
\begin{align}
  x_5 &= x_4 + \bfO_{out,x} - p_{x4} \, \bfO_{out,z} \CRNO
  y_5 &= y_4 + \bfO_{out,y} - p_{y4} \, \bfO_{out,z} \\
  z_5 &= z_4 + \bfO_{out,z} 
\end{align}
  \end{enumerate}

%-------------------------------------------------------------------------

\begin{figure}[tb]
  \centering
  \includegraphics[width=5in]{crystal-diffraction.pdf}
  \caption[Reference trajectory reciprocal space diagram for crystal diffraction.]
{Reference trajectory reciprocal space diagram for for A) Bragg diffraction and B) Laue
diffraction. The bar over the vectors indicates that they refer to the reference trajectory. The
$x$-$z$ coordinates shown are the element surface coordinates. All points in the diagram are in the
plane of the paper except for the tip of $\bfH$.  $\bfKbar_{0}$, and $\bfKbar_{H}$ are the wave
vectors inside the crystal and $\bfkbar_{0}$ and $\bfkbar_{H}$ are the wave vectors outside the
crystal. The reference photon traveling along the reference trajectory has $\bfKbar_0$ and
$\bfKbar_H$ originating at the $Q$ point. For Laue diffraction, the crystal faces are assumed
parallel.  For Bragg diffraction the crystal normal is in the $-\bfhat x$ direction while for Laue
diffraction the crystal normal is in the $-\bfhat z$ direction
  }
  \label{f:crystal.diffraction}
\end{figure}

%-------------------------------------------------------------------------
%-------------------------------------------------------------------------
\section{Crystal Element Tracking}
\label{s:crystal.tracking}

\textit{\large [Crystal tracking developed by Jing Yee Chee, Ken Finkelstein, and David Sagan]}

Crystal diffraction is modeled using dynamical diffraction theory. The notation here follows
Batterman and Cole\cite{b:batterman}.  The problem can be divided up into two parts. First the
reference trajectory must be calculated. This means calculating the incoming grazing angle
$\theta_{B,in}$ and outgoing grazing angle $\theta_{B,out}$ as well as calculating the
transformations between the various coordinate systems. This is done in \sref{s:crystal.ref},
\sref{s:crystal.trans}, and \sref{s:laue.ref}.  The second part is the actual tracking of the photon
and this is covered in \sref{s:bragg.track} and \sref{s:coherent.laue}
%% and \sref{s:incoherent.laue}.

%-------------------------------------------------------------------------
\subsection{Calculation of Entrance and Exit Bragg Angles}
\label{s:crystal.ref}

\fig{f:crystal.diffraction} shows the geometry of the problem. The bar over the vectors indicates
that they refer to the reference trajectory. The reference trajectory is calculated such that the
reference photon will be in the center of the Darwin curve. That is, the internal wave vectors
$\bfKbar_0$ and $\bfKbar_H$ originate from the $Q$ point (See \cite{b:batterman} Figs.~8 and 29).

The external wave vectors $\bf k_0$, and $\bf k_H$ and the internal wave vectors
have magnitude
\begin{align}
  |\bf k_0| &= |\bf k_H| = \frac{1}{\lambda} 
  \label{kk1l1} \\
  |\bfKbar_0| &= |\bfKbar_H| = \frac{1 - \delta}{\lambda}
  \label{kk1l2}
\end{align}
where $\lambda$ is the wavelength, and $\delta$ is
\begin{equation}
  \delta = \frac{\lambda^2 r_e}{2 \, \pi \, V} \, F_0' = \frac{\Gamma}{2} \, F_0'
  = \frac{1}{2} \, \Gamma \, F_0'
\end{equation}
with $r_e$ being the classical electron radius, $V$ the unit cell volume, and $F_0'$ is the real
part of the $F_0$ structure factor.

In element surface coordinates (which will be the coordinate system used henceforth), $\bfkbar_0$
lies in the $x$-$z$ plane. $\bfKbar_0$ is related to $\bfkbar_0$ via Batterman Eq.~(25)
\begin{equation}
  \bfK_0 = \bfk_0 + q_0 \, \bfhat n
  \label{kkqn1}
\end{equation}
where the value of $q_0$ is to be determined. Here, and in equations below, if the equation is true
in general, and not just for the reference trajectory, the bar superscript is dropped.

Since $\bfhat n$ is in the $-\bfhat x$ direction, $\bfKbar_0$ is also in the $x$-$z$ plane. Thus
$\bfkbar_0$ and $\bfKbar_0$ can be written in the form
\begin{alignat}{3}
  \bfkbar_0 &= \frac{1}{\lambda} \, 
    \begin{pmatrix}
    -\cos\theta_{B,in} \\
    0 \\
    \sin\theta_{B,in}
    \end{pmatrix}
  \; , & \qquad
  \bfKbar_0 &= \frac{1 - \delta}{\lambda} \, 
    \begin{pmatrix}
    -\cos\theta_0 \\
    0 \\
    \sin\theta_0
    \end{pmatrix}
  & \qquad &\text{[Bragg]} \CRNO
  \bfkbar_0 &= \frac{1}{\lambda} \, 
    \begin{pmatrix}
    \sin\theta_{B,in} \\
    0 \\
    \cos\theta_{B,in}
    \end{pmatrix}
  \; , & \qquad
  \bfKbar_0 &= \frac{1 - \delta}{\lambda} \, 
    \begin{pmatrix}
    \sin\theta_0 \\
    0 \\
    \cos\theta_0
    \end{pmatrix}
  &\qquad & \text{[Laue]} 
  \label{k1lst}
\end{alignat}
Where, as shown in \fig{f:crystal.diffraction}, $\theta_{B,in}$, and
$\theta_0$ are the angles of $\bfkbar_0$ and $\bfKbar_0$ with respect
to the $x$-axis for Bragg reflections and with respect to the $z$-axis
for Laue reflection. 

$\alpha_H$ (\vn{alpha_angle}) is the angle that $\bfH$ makes with
respect to the $-\bfhat z$ axis and $\psi_H$ (\vn{psi_angle}) is the
rotation of $\bfH$ around the $-\bfhat z$ axis such that for $\psi_H =
0$, $\bfH$ is in the $x$-$z$ plane and oriented as shown in
\fig{f:crystal.diffraction}. Thus
\begin{equation}
  \bfH 
  \equiv \frac{1}{d} \, \bfhat{H} 
  = \frac{1}{d}
    \begin{pmatrix} 
       -\sin \alpha_H \, \cos \psi_H \\ \sin \alpha_H \, \sin \psi_H \\ -\cos \alpha_H
    \end{pmatrix}
  \label{h1daa}
\end{equation}
where $\bfhat{H}$ is $\bfH$ normalized to 1. $\alpha_H$ is determined
via the setting of \vn{b_param} and via \Eq{batat}.

The vectors $\bfK_0$ and $\bfH$ must add up to the reciprocal lattice vector $\bfK_H$
\begin{equation}
  \bfK_H = \bfK_0 + \bfH
  \label{kkh}
\end{equation}
Taking the length of both sides of this equation and using
\Eqs{kk1l2}, \eq{k1lst}, and \eq{h1daa} gives for
$\theta_0$
\begin{equation}
  \sin \theta_0 = 
  \begin{dcases}
    \dsfrac{-\beta \, \what{H}_z - \what{H}_x \, \sqrt{\what{H}_x^2 + \what{H}_z^2 - \beta^2}}
    {\what{H}_x^2 + \what{H}_z^2} & \vn{Bragg} \\
    \dsfrac{-\beta \, \what{H}_x + \what{H}_z \, \sqrt{\what{H}_x^2 + \what{H}_z^2 - \beta^2}}
    {\what{H}_x^2 + \what{H}_z^2} & \vn{Laue}
  \end{dcases}
\end{equation}
where
\begin{equation}
  \beta \equiv \frac{\lambda}{2 \, d \, (1 - \delta)}
\end{equation}
Once $\theta_0$ has been calculated, $\theta_{B,in}$ can be calculated from \Eq{kkqn1}
\begin{align}
  \cos\theta_{B,in} &= (1 - \delta) \, \cos\theta_0 \quad [\text{Bragg}] \\
  \sin\theta_{B,in} &= (1 - \delta) \, \sin\theta_0 \quad [\text{Laue}] 
\end{align}

The outgoing reference wave vector $k_H$ is computed using the equation
\begin{equation}
  \bfK_H = \bfk_H + q_H \, \bfhat n
  \label{kkqn2}
\end{equation}
Using this with \Eqs{h1daa} and \eq{kkh} gives
\begin{align}
  \kbar_{H,x} &= \Kbar_{H,z} = \frac{1}{d} \, \what{H}_x + \kbar_{0,x} \CRNO
  \kbar_{H,y} &= \Kbar_{H,y} = \frac{1}{d} \, \what{H}_y 
  \label{k1dapl} \\
  \kbar_{H,z} &= \sqrt{\frac{1}{\lambda^2} - \kbar_{H,x}^2 - \kbar_{H,y}^2} \nonumber
\end{align}

The total bending angle of the reference trajectory is then
\begin{equation}
  \theta_{bend} = \tan^{-1} 
  \left( \frac{ | \bfkbar_0 \cross \bfkbar_H | }{\bfkbar_0 \dotproduct \bfkbar_H} \right) 
\end{equation}
The outgoing Bragg angle $\theta_{B,out}$ is then {\em defined} to be
the difference between the total bend angle and the entrance Bragg angle.
\begin{equation}
  \theta_{B,out} \equiv \theta_{bend} - \theta_{B,in}
\end{equation}

%-------------------------------------------------------------------------
\subsection{Crystal Coordinate Transformations}
\label{s:crystal.trans}

There are four transformations needed between coordinates
denoted by $\Bf\Sigma_1$, $\Bf\Sigma_2$, $\Bf\Sigma_3$, and $\Bf\Sigma_4$
\begin{example}
  \(\Bf\Sigma_1\)  Transform from laboratory entrance to element entrance coordinates.  
  \(\Bf\Sigma_2\)  Transform from element entrance to surface coordinates.  
  \(\Bf\Sigma_3\)  Transform from surface to element exit coordinates.  
  \(\Bf\Sigma_4\)  Transform from element exit to laboratory exit coordinates.  
\end{example}
The total transformation is just the map represented by $\bfS$ and
$\bfV$ of \Eqs{vwlv} and \eq{wws}
\begin{equation}
  [\bfS, \bfV] = \Bf\Sigma_4 \, \Bf\Sigma_3 \, \Bf\Sigma_2 \, \Bf\Sigma_1
\end{equation}

\index{tilt_corr}\index{crystal!tilt correction}
The transformation $\Bf\Sigma_1$ is given in
\sref{s:crystal.trans.le} and the transformation $\Bf\Sigma_4$ is
given in \sref{s:crystal.trans.el}. In general, the transformation
$\Bf\Sigma_1$ needs a ``tilt correction'' (\Eq{tttt}), as explained
below, when $\psi_H$ is nonzero.  [The exception is when the
\vn{undiffracted} or \vn{forward_diffracted} beam is tracked with Laue
geometry. In these cases, no tilt correction is needed.] Since this
tilt correction is independent of any misalignments, the tilt
correction calculation proceeds assuming here that there are no
misalignments. The finite $\bfV$ due to the finite crystal thickness
in Laue diffraction will also be ignored for the moment.

Without misalignments, and with $\psi_H$ zero, the transformation
$\Bf\Sigma_1$ is, as it is for every other type of element,
just the unit matrix. 
\begin{equation}
  \Bf\Sigma_1 = \bfI
\end{equation}
That is, the two coordinate systems are
identical. Furthermore, the transformation $\Bf\Sigma_2$ from element
entrance coordinates to surface coordinates is a rotation around the $y$
axis
\begin{align}
  \Bf\Sigma_2 &= \bfR_y(\theta_{B,in}) \equiv \begin{pmatrix}
     \cos\theta_{B,in} & 0 & \sin\theta_{B,in} \\
     0                 & 1 & 0                 \\
    -\sin\theta_{B,in} & 0 & \cos\theta_{B,in} \\
  \end{pmatrix}
  \qquad &\text{[Laue]}
  \label{mt0t010} \\
  &= \bfR_y(\theta_{B,in} - \frac{\pi}{2})
  \qquad &\text{[Bragg]} \nonumber
\end{align}
The transformation from element surface coordinates to element exit
coordinates, $\Bf\Sigma_3$, is another rotation around the $y$ axis 
\begin{align}
  \Bf\Sigma_3 &= \bfR_y(\theta_{B,out})
  \qquad &\text{[Laue]} \\
  &= \bfR_y(\theta_{B,out} + \frac{\pi}{2})
  \qquad &\text{[Bragg]} \nonumber
\end{align}
and the transformation from element exit coordinates
to laboratory exit coordinates, $\Bf\Sigma_{out}$ is the unity matrix
\begin{equation}
  \Bf\Sigma_4 = \bfI
\end{equation}
Thus, the combined transformation $\bfS$ from laboratory entrance to
laboratory exit coordinates is a rotation around the $y$ axis of
$\theta_{B,in}+\theta_{B,out}$ as explained in section
\sref{s:global}
\begin{equation}
  \bfS = \Bf\Sigma_4 \, \Bf\Sigma_3 \, \Bf\Sigma_2 \, \Bf\Sigma_1 
  = \bfR_y(\theta_{B,in}+\theta_{B,out})
\end{equation}

\index{ref_tilt}\index{psi_angle}
When $\psi_H$ is non-zero, the situation is complicated since, if
$\bfS$ as calculated above is used, the vector $\bfkbar_H$ would be
bent out of the $x$-$z$ plane even though it has been assumed that the
\vn{ref_tilt} $\theta_t$ is zero. But $\bfkbar_H$ points in the same
direction as the $z$ axis of the outgoing reference
trajectory. Furthermore, by {\em definition}, the reference trajectory
has the form given by \Eq{ustt} with the $\bfR_{z}(\theta_t)$ matrix
depending only upon the \vn{ref_tilt} parameter (which is here taken
to be zero). To satisfy \Eq{ustt}, the crystal must be reoriented to
keep the $\bfk_H$ vector in the $x$-$z$ plane of the laboratory
entrance coordinates.  The reorientation is done by rotating the
crystal about the laboratory entrance $\Bf z$ axis by an amount
$\theta_{corr}$ (\vn{tilt_corr}).

With this tilt correction the transformation $\Bf\Sigma_1$ is a
rotation about the $z$ axis
\begin{equation}
  \Bf\Sigma_1 = 
  \begin{pmatrix}
    \cos\theta_{corr} & -\sin\theta_{corr} & 0 \\
    \sin\theta_{corr} &  \cos\theta_{corr} & 0 \\
    0                 &  0                 & 1                
  \end{pmatrix}
\end{equation}
To calculate a value for $\theta_{corr}$, note that
the transformation $\Bf\Sigma_2$ from element entrance coordinates to element surface
coordinates is not affected by a finite $\psi_H$ and so \Eq{mt0t010}
is unmodified. The $\bfk_H$ vector, expressed in laboratory entrance
coordinates, is $\Bf\Sigma_1^{-1} \, \Bf\Sigma_2^{-1} \, \bfk_H$ where the
components of $\bfk_H$ are given by \Eq{k1dapl}. To
satisfy \Eq{ustt}, this vector must have zero $y$ component
\begin{equation}
  \left( \Bf\Sigma_1^{-1} \, \Bf\Sigma_2^{-1} \, \bfk_H \right) \dotproduct
  \begin{pmatrix} 0 \\ 1 \\ 0 \end{pmatrix}
  = 0
\end{equation}
Solving gives
\begin{equation}
  \theta_{corr} = \tan^{-1} 
  \frac{k_{H,y}}{k_{H,z} \, \sin\theta_{B,in} - k_{H,x} \, \cos\theta_{B,in}}
\end{equation}
The transformation $\Bf\Sigma_3$ from element surface coordinates to
element exit coordinates is now obtained by requiring that the total
transformation from laboratory entrance to laboratory exit coordinates
be the $\bfR_{y}(-\alpha_b)$ matrix given in \Eq{ustt}
\begin{equation}
  \Bf\Sigma_3 \, \Bf\Sigma_2 \, \Bf\Sigma_1 = 
  \begin{pmatrix}
    \cos\theta_{bend} & 0 & -\sin\theta_{bend} \\
    0          & 1 & 0           \\
    \sin\theta_{bend} & 0 & \cos\theta_{bend}
  \end{pmatrix}
\end{equation}
In the above equation, the transformation $\Bf\Sigma_4$ has been
dropped since it is the unit matrix independent of $\psi_H$.

For Laue diffraction when the non-diffracted beam is tracked, the exit
coordinate system corresponds to the entrance coordinate system. That
is, $\bfV$ is the unit matrix. In this case, there is no tilt
correction and $\Bf\Sigma_3 = \bfR_y(-\theta_{B,in})$ is just the
inverse of $\Bf\Sigma_2$.

%-------------------------------------------------------------------------

\begin{figure}
\centering
\includegraphics[width=4in]{crystal-energy.pdf}
  \caption[Reference energy flow for Laue diffraction]{
Energy flow used to determine the reference orbit for Laue diffraction.
  }
\label{f:crystal.energy}
\end{figure}

%-------------------------------------------------------------------------
\subsection{Laue Reference Orbit}
\label{s:laue.ref}

For Laue diffraction, with the reference orbit following the \vn{undiffracted} beam, the reference
orbit at the exit surface is just the extension of the reference orbit at the entrance
surface. Since the reference orbit's direction is $\bfkbar_0$.  the reference orbit displacement
vector $\bfL$ (cf.~\Eq{vwlv}) is given by
\begin{equation}
  \bfL = \frac{t^2}{d\bfkbar_0 \dotproduct \Bf t} \, d\bfkbar_0
  \qquad \text{[undiffracted]}
\end{equation}
where
\begin{equation}
  \Bf t = \begin{pmatrix}
    0 \\ 0 \\ t
  \end{pmatrix}
\end{equation}
with $t$ being the crystal thickness and the $z$-axis pointing into the crystal as illustrated in
\fig{f:crystal.energy}. The $\bfS$ rotation matrix (\Eq{wws}) for the undiffracted beam 
will be the unit matrix.

With the reference orbit following the \vn{forward_diffracted} or \vn{Bragg_diffracted} beam, the
displacement vector $\bfL$ follows the energy flow associated with the tie points labeled $A$ or $B$
in \fig{f:crystal.energy}. These tie points are defined by the intersection of the dispersion
surfaces and the vector $\Bf n$ originating from the point $T$ as shown in the figure.  The energy
flow is perpendicular to the dispersion surface and it can be shown that since, by construction,
$\Bf n$ goes through the $Q$ point, and since the dispersion surfaces are hyperboles, the energy
flows from $A$ and $B$ tie points are colinear. The direction of the energy flow is given by:
\begin{equation}
  \bfKbar_f = \xi_H \, \bfKbar_H + \xi_0 \, \bfKbar_0
\end{equation}
where $\xi_H$ and $\xi_0$ are given by \cite{b:batterman} Eq.~(18) (See section \sref{s:bragg.track} below).
$\bfL$ is thus
\begin{equation}
  \bfL = \frac{t^2}{\bfKbar_f \dotproduct \Bf t} \, \bfKbar_f
\end{equation}
At the exit surface, if the reference orbit is following the \vn{forward_diffracted} beam, the
orientation of the \vn{element exit} coordinates will be the same as the orientation of the
\vn{element entrance} coordinates. That is, $\bfS$ (\Eq{wws}) is the unit matrix.  If the reference
orbit is following the \vn{Bragg diffracted} beam, $\bfS$ is the same as for Bragg diffraction
\begin{equation}
  \bfS = 
  \begin{pmatrix}
    \cos\theta_{bend} & 0 & -\sin\theta_{bend} \\
    0                 & 1 & 0           \\
    \sin\theta_{bend} & 0 & \cos\theta_{bend}
  \end{pmatrix}
\end{equation}

%-------------------------------------------------------------------------
\subsection{Crystal Surface Reflection and Refraction}

%%%\begin{figure}[tb]
%%%  \centering
%%%  \includegraphics[width=5in]{crystal-surface-reflect.pdf}
%%%  \caption[Reflection from a crystal surface.]
%%%{Reflection from a crystal surface.}
%%%  \label{f:crystal.reflect}
%%%\end{figure}


There are corrections to the field amplitude and phase when a photon reflects or refracts from the
surface of a crystal. A plane wave is incident on a crystal surface with
\begin{equation}
  E = \what E_0 \, \exp(i \, \bfk_0 \, \bfr)
\end{equation}
An outgoing plane wave has a field
\begin{equation}
  E = \what E_1 \, \exp(i \, \bfk_1 \, \bfr)
\end{equation}
A simulation of this condition will start with a number of photons with wave vector $\bfk_0$ and
electric field $E_0$. After reflecting from the surface, the photons will have wave vector
$\bfk_1$. Now imagine a set of $N$ photons that flow through an planar area $dA_0$, perpendicular to
the incoming beam, before being reflected from the surface.

Since the electric field is $\what E_0$, when tracking incoherent photons
\begin{equation}
  \what E_0^2 = \frac{\alpha_p \, E_0^2 \, N}{dA_0}
\end{equation}
where $\alpha_p$ is the simulation constant (cf.~\Eq{panda1}. 
After the photons are reflected they will have some field $E_1$ and thus
\begin{equation}
  \what E_1^2 = \frac{\alpha_p \, E_1^2 \, N}{dA_1}
\end{equation}
Where $dA_1$ is the area that the photons flow through which is related
to $dA_0$ via
\begin{equation}
  \frac{dA_1}{dA_0} = \frac{\bfk_1 \dotproduct \bfz}{\bfk_0 \dotproduct \bfz} \equiv |b|
\end{equation}
Combining the above three equations, the change in field for a photon
as it reflects from the surface is
\begin{equation}
  \frac{E_1}{E_0} = \frac{\what E_1}{\what E_0} \, \sqrt{|b|} 
  \qquad \text{Incoherent}
\end{equation}

For coherent photon tracking the electric field at $dA_0$ is
\begin{equation}
  \what E_0 = \frac{\alpha_p \, E_0 \, N}{dA_0}
\end{equation}
After the photons are reflected they will have some field $E_1$ and thus
\begin{equation}
  \what E_1 = \frac{\alpha_p \, E_1 \, N}{dA_1}
\end{equation}
Combining these equations the change in field for a photon
as it reflects from the surface is
\begin{equation}
  \frac{E_1}{E_0} = \frac{\what E_1}{\what E_0} \, |b| 
  \qquad \text{Coherent}
\end{equation}

Additionally, for coherent tracking, all photons in a plane wave must
have the same phase when passing through an area transverse to the
wave. Thus the two photons labeled $a$ and $b$ in
\fig{f:crystal.reflect} must have the same phase advance in going from
$dA_0$ to $dA_1$. The difference in the phase advance for photon $b$
relative to $a$ from $dA_0$ to the surface is $\bfk_0 \dotproduct \bfr$
where $\bfr$ is the vector between where photon $b$ hits the surface
relative to photon $a$. Similarly, the difference in the phase advance
for photon $b$ relative to $a$ from the surface to $dA_0$ is $-\bfk_1
\dotproduct \bfr$. Since the total phase advance for both photons is the
same from $dA_0$ to $dA_1$ the phase shift $d\phi_b$ of photon $b$ as
it is reflected from the surface relative to the phase shift $d\phi_a$
is
\begin{equation}
  d\phi_b = d\phi_a - (\bfk_1 - \bfk_0) \dotproduct \bfr
  \label{dpbdpa}
\end{equation}

This shift in the reflection phase can be related to the lattice
diffraction planes. The wave vector difference can be written
\begin{equation}
  \bfk_1 - \bfk_0 = \bfH + q \, \bfhat n
  \label{k1k0}
\end{equation}
where $\bfhat n$ is perpendicular to the surface. Combining
\Eqs{dpbdpa} and \eq{k1k0} and since $\bfr$ is in
the plane of the surface
\begin{equation}
  d\phi_b = d\phi_a - \bfH \dotproduct \bfr
\end{equation}
This shows that the reflection shift has the same periodicity as the
pattern of the lattice planes at the surface of the crystal. Notice
that for a mirror, where one point on the surface is the same as any
other, $d\phi_b$ must be equal to $d\phi_a$. Using this in \Eq{dpbdpa}
gives 
\begin{equation}
  \bfk_1 \dotproduct \bfr = \bfk_0 \dotproduct \bfr
\end{equation}
and since $|\bfk_1| = |\bfk_0|$ this proves that the angle of
incidence is equal to the angle of reflection for a mirror.

In practice, the registration of the surface planes with respect to
the surface is not specified in a simulation. Thus the reflection
phase shift can only be calculated up to a constant offset. 

%-------------------------------------------------------------------------
\subsection{Bragg Crystal Tracking}
\label{s:bragg.track}

The starting photon coordinates are specified in the laboratory entrance coordinates. The
transformation from laboratory entrance coordinates to element entrance coordinates $\bftilde k_0$
is given in \sref{s:photon.lab.ele}. The transformation to element surface coordinates $\bfk_0$ is
\begin{equation}
  \bfk_0 =  \Bf\Sigma_2 \, \bftilde k_0
\end{equation}
with $\Bf\Sigma_2$ given by \Eq{mt0t010}.
The outgoing wave vector $\bfk_H$ is related to $\bfk_0$ via
\begin{equation}
  \bfk_H =  \bfk_0 + \bfH + q_t \, \bfhat n
\end{equation}
where $q_t$ is determined by using \Eqs{k1lst} and \eq{h1daa} in \Eq{kk1l1}
\begin{align}
  k_{H,x} &= k_{0,x} + H_x \nonumber \CRNO
  k_{H,y} &= k_{0,y} + H_y \\
  k_{H,z} &= \sqrt{\lambda^2 - k_{H,x}^2 - k_{H,y}^2} \nonumber
\end{align}

To compute the field amplitude of the outgoing photon, the equation to be solved is
(\cite{b:batterman} Eq.~(21))
\begin{equation}
  \xi_0 \, \xi_H = \frac{1}{4} \, k^2 \, P^2 \, \Gamma^2 \, F_H \, F_{\bar H}
  \label{xx14}
\end{equation}
where $\xi_0$ and $\xi_H$ are given by \cite{b:batterman} Eq.~(18) and $P$ is the polarization
factor
\begin{equation}
  P = 
  \begin{cases}
    1               & \sigma \text{ polarization state} \\
    \cos 2\theta_g  & \pi \text{ polarization state}
  \end{cases}
\end{equation}
$2\theta_g$ is the angle between $\bfK_0$ and $\bfK_H$ which is well
approximated by $\theta_{B,in} + \theta_{B,out}$.

The solution to \Eq{xx14} is (\cite{b:batterman} Eq.~(31))
\begin{align}
  \xi_0 &= \frac{1}{2} \, k \, |P| \, \Gamma \, [F_H \, F_{\bar H}]^{1/2} \, 
    |b|^{1/2} \, [\eta \pm (\eta^2 + \sign(b))^{1/2}] \CRNO
  \xi_H &= \frac{1}{2} \, k \, |P| \, \Gamma \, [F_H \, F_{\bar H}]^{1/2} \, 
    \frac{1}{|b|^{1/2} \, [\eta \pm (\eta^2 + \sign(b))^{1/2}]}
\end{align}
where the $+$ part of $\pm$ is for the $\alpha$ branch and the $-$ part of $\pm$ is for the $\beta$
branch and $\sign$ is the sign function
\begin{equation}
  \sign(b) \equiv \begin{cases} 1 & b > 0 \\ -1 & b < 0 \end{cases}
\end{equation}
and $\eta$ is given by \cite{b:blasdell} Eq.~(5)
\begin{equation}
  \eta = \frac{-b \, a + \Gamma \, F_0 \, (1 - b)}{2 \, \Gamma \, |P| \, \sqrt{|b| \, F_H \, F_{\bar H}}}
\end{equation}
with the asymmetry factor $b$ for the photon being tracked being given by \cite{b:blasdell} Eq.~(3)
\begin{equation}
  b \equiv \frac{\bfhat n \dotproduct \bfhat k_0}{\bfhat n \dotproduct (\widehat{{\bfk_0 + \bfH}})}
\end{equation}
and the angular deviation variable $a$ is given by \cite{b:blasdell} Eq.~(4)
\begin{equation}
  a \equiv \frac{H^2 + 2 \, \bfk_0 \dotproduct \bfH}{k_0^2} 
  = -2 \, \Delta \theta \, \sin(2\theta_B)
\end{equation}
Once $\xi_0$ and $\xi_H$ are determined, the ratio of the incoming and outgoing fields for the
$\alpha$ or $\beta$ branches can be computed via (\cite{b:batterman} Eq.~(24))
\begin{equation}
  r_E \equiv \frac{\bfE_H}{\bfE_0} 
  = \frac{- \, 2 \, \xi_0}{k \, P \, \Gamma \, F_{\bar H}} \,
  = \, \frac{- \, k \, P \, \Gamma \, F_H}{2 \, \xi_H} 
\end{equation}
where the $\alpha$ or $\beta$ subscript has been suppressed.  The total field which is the sum of the
fields on the branches is computed using the boundary conditions
\begin{equation}
  \bfE_0 = \bfE_{0\alpha} + \bfE_{0\beta}, \qquad\qquad 
  0 = \bfE_{H\alpha} + \bfE_{H\beta}
\end{equation}
Using the above two equations gives
\begin{align}
  \bfE_{0\alpha} &= \bfE_0 \, \frac{r_{E\beta}}{r_{E\beta} - r_{E\alpha}} \qquad\qquad
  \bfE_{H\alpha}  = \bfE_0 \, \frac{r_{E\alpha} \, r_{E\beta}}{r_{E\beta} - r_{E\alpha}} \CRNO
  \bfE_{0\beta} &= -\bfE_0 \, \frac{r_{E\alpha}}{r_{E\beta} - r_{E\alpha}} \qquad\qquad
  \bfE_{H\beta}  = -\bfE_0 \, \frac{r_{E\alpha} \, r_{E\beta}}{r_{E\beta} - r_{E\alpha}} 
\end{align}

As can be seen from Battermann and Cole Figs.~(8) and (29), the $\alpha$ tie point is excited and
the $\beta$ tie point is not if $\xi_{0\alpha} < \xi_{0\beta}$ and vice versa. Since only one tie
point is excited, The external field ratio is equal to the internal field ratio
\begin{equation}
  \frac{E_H^e}{E_0^i} = \frac{E_{Hj}}{E_{0j}}
\end{equation}
where $j$ is $\alpha$ or $\beta$ as appropriate.

%-------------------------------------------------------------------------
\subsection{Coherent Laue Crystal Tracking}
\label{s:coherent.laue}

Laue diffraction has two interior wave fields (branches), labeled $\alpha$ and $\beta$,
corresponding to the two tie points that are excited on the two dispersion surfaces. For coherent
tracking, a photon has some probability to be channeled to follow the $\alpha$ or $\beta$
branch. The electric field ratios $\what E_\alpha$ and $\what E_\beta$ (cf.~\Eq{rpss2}) are taken to
be equal to each other. With this choice, the probabilities $P_\alpha$ and $P_\beta$ for being
channeled to the $\alpha$ or $\beta$ branches are such that a branch with a greater intensity will
have a greater number of photons channeled down it.

When a crystal's \vn{ref_orbit_follows} parameter is set to \vn{bragg_diffracted}, The branching
probabilities are
\begin{equation}
  P_\alpha = \frac{|E_{H\alpha}|}{|E_{H\alpha}| + |E_{H\beta}|} , \qquad
  P_\beta = \frac{|E_{H\beta}|}{|E_{H\alpha}| + |E_{H\beta}|} , \qquad
  \what E_{H\alpha} = \what E_{H\beta}  = \frac{|E_{H\alpha}| + |E_{H\beta}|}{|E_0^i|}
\end{equation}
where (see Battermann and Cole\cite{b:batterman} Eqs~(42)), 
\begin{align}
  E_{H\alpha} &= -E_0^i \, \dsfrac{|b|^{1/2}}{2 \, \cosh v} \, 
    \frac{|P|}{P} \, 
    \frac{[F_H \, F_\Hbar]^{1/2}}{F_H} \, 
    \exp({-2 \, \pi \, i \, \bfK'_{H\alpha} \dotproduct \bfr_\alpha}) \,
    \exp({-2 \, \pi \, \bfK''_{H\alpha} \dotproduct \bfr_\alpha}) \CRNO
  E_{H\beta} &= E_0^i \, \dsfrac{|b|^{1/2}}{2 \, \cosh v} \, 
    \frac{|P|}{P} \, 
    \frac{[F_H \, F_\Hbar]^{1/2}}{F_H} \, 
    \exp({-2 \, \pi \, i \, \bfK'_{H\beta} \dotproduct \bfr_\beta}) \,
    \exp({-2 \, \pi \, \bfK''_{H\beta} \dotproduct \bfr_\beta})
\end{align}
where $\bfr_\alpha$ and $\bfr_\beta$ are the vectors from the entrance surface to the exit surface
for the $\alpha$ and $\beta$ wave fields
\begin{equation}
  \bfr_\alpha = \frac{t^2}{\bfS_\alpha \dotproduct \Bf t} \, \bfS_\alpha , \qquad
  \bfr_\beta = \frac{t^2}{\bfS_\beta \dotproduct \Bf t} \, \bfS_\beta
\end{equation}
with
\begin{align}
  \bfS_\alpha &= e^{-2 \, v} \, \Bf s_0 + \left| b \, \frac{F_H \, F_\Hbar}{F_H^2} \right| \, \Bf s_H \CRNO
  \bfS_\beta &= e^{2 \, v} \, \Bf s_0 + \left| b \, \frac{F_H \, F_\Hbar}{F_H^2} \right| \, \Bf s_H 
\end{align}
The phase shift of the electric field is obtained from the phase of $E_{H\alpha}$ if the photon is
channeled into the $\alpha$ branch and $E_{H\beta}$ if the photon is channeled into the $\beta$
branch.

When a crystal's \vn{ref_orbit_follows} parameter is set to \vn{forward_diffracted} or
\vn{undiffracted}, the algorithm is similar to the \vn{bragg_diffracted} case except $E_{0\alpha}$
and $E_{0\beta}$ are used in place of $E_{H\alpha}$ and $E_{H\beta}$ with
\begin{align}
  E_{0\alpha} &= E_0^i \, \dsfrac{e^{-v}}{2 \, \cosh v} \, 
    \exp({-2 \, \pi \, i \, \bfK'_{0\alpha} \dotproduct \bfr_\alpha}) \,
    \exp({-2 \, \pi \, \bfK''_{0\alpha} \dotproduct \bfr_\alpha}) \CRNO
  E_{0\beta} &= E_0^i \, \dsfrac{e^{-v}}{2 \, \cosh v} \, 
    \exp({-2 \, \pi \, i \, \bfK'_{0\beta} \dotproduct \bfr_\alpha}) \,
    \exp({-2 \, \pi \, \bfK''_{0\beta} \dotproduct \bfr_\beta})
\end{align}

Since a simulation photon has two polarization components, the above equations are used for one
polarization component and for the second polarization component the same branch is used as for the
first with an appropriately scaled $\what E$.

%-------------------------------------------------------------------------
%\subsection{Incoherent Laue Crystal Tracking}
%\label{s:incoherent.laue}
%
%Laue diffraction has two interior wave fields (branches), labeled $\alpha$ and $\beta$,
%corresponding to the two tie points that are excited on the two dispersion surfaces. For incoherent
%tracking it is assumed that these wave fields overlap at the exit surface (\cite{b:batterman}
%Eq.~(87)) and so add coherently. This is a good approximation if the crystal is very thin where the
%wave fields do not travel an appreciable spatial distance and is also a good approximation when the
%crystal is thick since the $\beta$ branch will be heavily attenuated. At intermediate thicknesses
%this approximation is good when a photon is near the Bragg angle since, in this case, the fields
%will be traveling in similar directions.
%
%Another approximation is that the path

%---------------------

%\begin{figure}[tb]
%  \centering
%  \includegraphics[width=5in]{mosaic-crystal.pdf}
%  \caption[Mosaic crystal Monte Carlo tracking.]{Mosaic crystal representation used for Monte Carlo tracking.}
%  \label{f:mosaic.crystal}
%\end{figure}
%
%%-------------------------------------------------------------------------
%\subsection{Mosaic Crystal Tracking}
%\label{s:mosaic.track}
%
%Tracking through a mosaic crystal tracking is illustrated in Figure~\ref{f:mosaic.crystal}. The
%\bmad algorithm is similar to the method presented by Alianelli et al. \cite{b:mosaic}. The mosaic
%crystal is idealized as a number of small perfect crystals, called crystallites, all of which have
%the same specified thickness $T_m$ (the \vn{mosaic_thickness} parameter of the crystal). The
%crystallites are all misoriented with angular probability distribution $W$ given by
%\begin{equation}
%  W(\theta, \phi) = \frac{1}{2 \, \pi \, \sigma_{in} \, \sigma_{out}} \exp \left(-
%  \frac{\theta^2}{2 \, \sigma_{in}^2} - \frac{\phi^2}{2 \, \sigma_{out}^2} \right)
%  \label{w12p}
%\end{equation}
%where $\theta$ is the misorientation angle in the diffraction plane and $\phi$ is the misorientation
%angle out of the diffraction plane. The sigmas, $\sigma_{in}$ and $\sigma_{out}$, correspond to the
%parameters of a crystal element (\sref{s:crystal}) \vn{mosaic_angle_rms_in_plane} and
%\vn{mosaic_angle_rms_out_plane} respectively.
%
%The crystalites are modeled to lie in layers of thickness $T_m$. For \vn{Laue} diffraction
%(\vn{b_param} > 0), the actual thickness $T_m$ used is adjusted so that there is an integer number
%of mosaic layers. That is
%\begin{equation}
%  T_m(\text{in simulation} = \frac{T}{\text{nint} (T/T_m)}
%\end{equation}
%where $T$ is the crystal thickness and \vn{nint} is the nearest integer function. 
%
%Tracking is done step-by-step with a photon either traversing or being reflected by a crystalite.
%Interference effects between crystallites are not taken into account so the tracking is incoherent
%(\sref{s:coher.incoher}).
%
%At the beginning of a step the photon is in between two layers and will be traveling towards one of
%these layers. A random number generator along with \Eq{w12p} is used to determine the misorientation
%of the crystallite that the photon is going through. Factoring this misorientation into account, the
%amplitude scattering $R$ and transmission $T$ probabilities are calculated using dynamical diffraction theory.
%
%A random number generator is used to determine if the photon is scattered or transmitted. If $P_T$ is the
%probability assigned for the probability of transmission, then
%\begin{equation}
%  r_T = 
%\end{equation}

%-------------------------------------------------------------------------
\section{X-ray Targeting}
\label{s:targeting}

X-rays can have a wide spread of trajectories resulting in many ``doomed'' photons that hit
apertures or miss the detector with only a small fraction of ``successful'' photons actually
contributing to the simulation results. The tracking of doomed photons can therefore result in an
appreciable lengthening of the simulation time. To get around this, \bmad can be setup to use what
is called ``targeting'' to greatly reduce the number of doomed photons generated.

Photons can be generated either at a source like a \vn{wiggler} element or at a place where
diffraction is simulated like at a \vn{diffraction_plate} element. To be able to do targeting, an
element with apertures defined must be present downstream from the generating element. The idea is
to only generate photons that are going in the general direction of the ``target'' which is the
space within the aperture.

A necessary restriction for targeting to work is that the photon must travel in a straight line
through all elements between the generating element and the element with the apertures. So, for
example, a \vn{crystal} element would not be allowed between the two two elements. A \vn{crystal}
element could be the aperture element as long as the aperture was defined before photons were
diffracted. That is, if the aperture was at the upstream end of the crystal or was defined with
respect to the \vn{crystal} surface.

The target is defined by the four corners of the aperture. In \vn{element} coordinates, the $(x, y,
z)$ values of the corners are:
\begin{example}
  (-x1_limit, -y1_limit, z_lim)
  (-x1_limit,  y2_limit, z_lim)
  ( x2_limit, -y1_limit, z_lim)
  ( x2_limit,  y2_limit, z_lim)
\end{example}
where \vn{x1_limit}, etc. are the aperture limits (\sref{s:limit}) and \vn{z_lim} will be zero
except if the element's \vn{aperture_at} parameter is set to \vn{entrance_end} in which case
\vn{z_lim} will be set to \vn{-L} where \vn{L} is the length of the element.

If the aperture is associated with a curved surface (for example with a \vn{crystal} element), four
extra corner points are also used to take into account that the aperture is not planar. These extra
points have $(x, y, z)$ values in \vn{element} coordinates of
\begin{example}
  (-x1_limit, -y1_limit, z_surface(-x1_limit, -y1_limit))
  (-x1_limit,  y2_limit, z_surface(-x1_limit,  y2_limit))
  ( x2_limit, -y1_limit, z_surface( x2_limit, -y1_limit))
  ( x2_limit,  y2_limit, z_surface( x2_limit,  y2_limit))
\end{example}
where \vn{z_surface(x,y)} is the $z$ value of the surface at the particular $(x, y)$ point being
used. Notice that in this case \vn{z_lim} is zero.

The coordinates of the four or eight corner points are converted from \vn{element} coordinates of
the aperture element to \vn{element} coordinates of the photon generating element. Additionally, the
approximate center of the aperture, which in \vn{element} coordinates of the aperture element is
$(0, 0, z_lim)$, is converted to \vn{element} coordinates of the photon generating element.

The above calculation only has to be done once at the beginning of a simulation.

When a photon is to be emitted from a given point $(x_{emit}, y_{emit}, z_{emit})$, the problem is
how to restrict the velocity vector $(\beta_x, \beta_y, \beta_z)$ (which is normalized to 1) to
minimize the number of doomed photons generated. The problem is solved by constructing a vector $\Bf
r$ for each corner point:
\begin{equation}
  \Bf r = (x_{lim}, y_{lim}, z_{lim}) - (x_{emit}, y_{emit}, z_{emit})
\end{equation} 
The direction of each $\Bf r$ is characterized in polar coordinates $(\phi, y)$ defined by
\begin{align}
  y &= \frac{r_y}{|r|} \CRNO
  \tan\phi &= \frac{r_x}{r_z} 
\end{align}
For now make the assumption that $r_z$ is positive and larger than $r_x$ and $r_y$ for all $\Bf r$.
Let $\phi_{max}$ and $\phi_{min}$ be the maximum and minimum $\phi$ values over all the $\Bf
r$. Similarly, let $y_{min}$ and $y_{max}$ be the minimum and maximum $y$ values over all the $\Bf
r$. The rectangle in $(\phi, y)$ space defined by these four min and max values almost covers the
projection of the aperture onto the unit sphere. There is a correction that must be made due to the
fact that a straight line of constant $y$ in $(x, y, z)$ space projects to a curved line when
projected onto $(\phi, y)$ space. Therefore a correction must be made to $y_{min}$ when $y_{min} <
0$:
\begin{equation}
  y_{min} \rightarrow 
  \frac{y_min}{\sqrt{(1 - y_{min}^2) \, \cos^2 (phi_{max} - \phi_{min})/2 + y_{min}^2}}
\end{equation}
with a similar correction for $y_{max}$ that must be made when $y_{max} > 0$.


The above prescription works as long as the projection of the aperture onto $(\phi, y)$ space does
not touch the branch cut at $\phi = \pi$ or cover the singular points $y = \pm 1$. Generally these
restrictions are fullfilled since $\Bf z$ is the direction of the reference orbit. If this is not
the case, a transformation can be made where rotation matrices are constructed to transform between
the \vn{element} coordinates of the emitting element and what are called \vn{target} coordinates
defined so that $\Bf r$ for the \vn{center} point has the form $(0, 0, |r|)$. The procedure for
calculating the photon velocity vector is now
  \begin{enumerate}
  \item
Rotate all the corner $\Bf r$ from \vn{element} to \vn{target} coordinates.
  \item
Calculate min and max values for $\phi$ and $y$.
  \item
Calculate the velocity vector such that the $(\phi, y)$ of this vector
falls within the min and max values in the last step.
  \item
Rotate the velocity vector back to \vn{element} coordinates.
  \end{enumerate}

