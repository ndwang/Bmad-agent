\chapter {Element Attributes}
\label{c:attrib}
\index{element attribute}

For a listing of element attributes for each type of element, see Chapter~\sref{c:attrib.list}.

%-----------------------------------------------------------------
\section{Dependent and Independent Attributes} 
\label{s:depend} 
\index{element attribute!dependent and independent}

\index{parameter statement}
\index{dependent attribute}
For convenience, \bmad computes the values of some attributes based upon the values of other
attributes. Some of these dependent variables are listed in Table~\ref{t:dependent}. Also shown in
Table~\ref{t:dependent} are the independent variables they are calculated from.

\index{bbi_constant}\index{charge}\index{sig_x}\index{sig_y}
\index{e_tot}\index{e_field}\index{voltage}
\index{hkick}\index{vkick}\index{gap}\index{l}
\index{e_tot}\index{e_loss}\index{delta_e}\index{gradient}
\index{l}\index{rho}\index{angle}\index{l_chord}
\index{g}\index{l}\index{k1}\index{rho}\index{num_steps}\index{ds_step}
\index{b_max}\index{e_tot}\index{beambeam}\index{elseparator}
\index{lcavity}\index{rbend}\index{sbend}\index{wiggler}
\begin{table}[ht]
\centering {
\begin{tabular}{lll} \toprule
 {\em Element}                & {\em Independent Variables}    & {\em Dependent Variables}          \\ \midrule
 All elements                 & \vn{ds_step}                   & \vn{num_steps}                     \\
 \vn{BeamBeam}                & \vn{charge}, \vn{sig_x}, \vn{sig_y}, \vn{e_tot}
                                                               & \vn{bbi_constant}                  \\
 \vn{Elseparator}             & \vn{hkick}, \vn{vkick}, \vn{gap}, \vn{l}, \vn{e_tot}      
                                                               & \vn{e_field}, \vn{voltage}         \\
 \vn{Lcavity}                 & \vn{gradient}, \vn{l}          & \vn{e_loss}, \vn{voltage}          \\
 \vn{Rbend}, \vn{Sbend}       & \vn{g}, \vn{l}                     
                                                               & \vn{rho}, \vn{angle}, \vn{l_chord} \\

 \vn{Wiggler} (map type)      & \vn{term(i)}                   & \vn{b_max}, \vn{k1}, \vn{rho}      \\
 \vn{Wiggler} (periodic type) & \vn{b_max}, \vn{e_tot}         & \vn{k1}, \vn{rho}                  \\ \bottomrule
\end{tabular}
}
\caption[Table of dependent variables.]{Partial listing of dependent variables and 
  the independent variables they are calculated from.}
\label{t:dependent}
\end{table}

For electric and magnetic field strength parameters, the \vn{field_master} parameter
(\sref{s:field.master}) can be used to determine if the be normalized or unnormalized, values are
dependent or independent.

%-----------------------------------------------------------------
\section{Field_Master and Normalized Vs. Unnormalized Field Strengths}
\label{s:field.master}
\index{field_master}

The \vn{field_master} attribute of an element sets whether the element's normalized (normalized by
the reference energy) field strengths or the unnormalized strengths are the independent variables
(\sref{s:depend}). See \sref{s:mag.field} for details as to how they are related. The setting of
\vn{field_master} also sets whether an element's magnetic multipoles (\sref{s:multip}) are
interpreted as normalized or unnormalized (electric multipoles are always treated as unnormalized).

Table~\ref{t:dep.field} shows some normalized and unnormalized field strength attributes.  The
default value of \vn{field_master} for an element is False if there are no field values set in the
lattice file for that element. If normalized field values are present then the default is also False.
If there are unnormalized field values present then the default is True.

For example:
\begin{example}
  Q1: quadrupole, b1_gradient = 0   ! Field strengths are the independent variables
  Q1: quadrupole, field_master = T  ! Same as above
  Q2: quadrupole        ! Define Q2.
  Q2[b1_gradient] = 0   ! Field strengths now the independent variables.
  Q2[field_master] = T  ! Same as above.
\end{example}

Specifying both normalized and unnormalized strengths for a given element is not
permitted. For example:
\begin{example}
  Q3: quadrupole, k1 = 0.6, bl_hkick = 37.5  ! NO. Not VALID.
\end{example}

\index{g}\index{dg}\index{b_field}\index{bs_field}\index{db_field}
\index{b1_gradient}\index{b2_gradient}\index{b3_gradient}\index{ks}
\index{k1}\index{k2}\index{k3}
\index{bl_kick}\index{bl_hkick}\index{bl_vkick}
\index{kick}\index{hkick}\index{vkick}
\index{sbend}\index{rbend}\index{solenoid}\index{quadrupole}
\index{sol_quad}\index{sextupole}\index{octupole}
\begin{table}[ht]
\centering {
\begin{tabular}{lll} \toprule
  {\em Element}              & {\em Normalized} & {\em Unnormalized}   \\ \midrule
  \vn{Sbend}, \vn{Rbend}     & \vn{g}           &  \vn{b_field}        \\
  \vn{Sbend}, \vn{Rbend}     & \vn{dg}          &  \vn{db_field}       \\
  \vn{Solenoid, Sol_quad}    & \vn{ks}          &  \vn{bs_field}       \\
  \vn{Quadrupole, Sol_quad, Sbend, Rbend}            
                             & \vn{k1}          &  \vn{b1_gradient}    \\
  \vn{Sextupole, Sbend, Rbend}             
                             & \vn{k2}          &  \vn{b2_gradient}    \\
  \vn{Octupole}              & \vn{k3}          &  \vn{b3_gradient}    \\
  \vn{HKicker}, \vn{VKicker} & \vn{kick}        &  \vn{bl_kick}        \\
  Most                       & \vn{hkick}       &  \vn{bl_hkick}       \\
  Most                       & \vn{vkick}       &  \vn{bl_vkick}       \\ \bottomrule
\end{tabular}
}
\caption {Example normalized and unnormalized field strength attributes.}
\label{t:dep.field}
\end{table}

%-----------------------------------------------------------------
\section{Type, Alias and Descrip Attributes}
\label{s:alias}
\index{type|hyperbf}
\index{alias|hyperbf}
\index{descrip|hyperbf}

There are three string labels associated with any element:
\begin{example}
  type    = "<String>"
  alias   = "<String>"
  descrip = "<String>"
\end{example}
The \vn{type} and \vn{alias} attributes can be up to 40 characters in length and \vn{descrip} can be
up to 200 characters. If the attribute string does not contain a blank, comma, or a semicolon, the
quotation marks may be omitted. Unlike the element name, these strings are not converted to upper case.

These labels have a number of purposes. For example, a \bmad based program that is used as an online
machine model could associate a label string with a machine database element to facilitate machine
control. Also these labels can be used with pattern matching when selecting elements (\sref{s:ele.match}).

Example:
\begin{example}
  Q00W: Quad, type = "My Type", alias = Who_knows, &
                                  descrip = "Only the shadow knows"
\end{example}

%-----------------------------------------------------------------
\section{Group, Overlay, and Ramper Element Syntax}
\label{s:go.syntax}
\index{group!syntax}\index{overlay!syntax}
\index{gang|hyperbf}

The syntax for specifying \vn{group} (\sref{s:group}), \vn{overlay} (\sref{s:overlay}), or
\vn{ramper} (\sref{s:ramper}) elements
are virtually identical and is discussed below. The name ``\vn{controller}'' will be used to denote
either a \vn{group}, \vn{overlay}, or \vn{ramper} element.

Controller elements have a set of one or more ``\vn{variables}'' that are used to control the values
of attributes of other elements (called ``\vn{slave}'' attributes).

There are two types of controllers in terms of how slave attribute values are computed ---
\vn{Expression} based controllers and \vn{knot} based controllers. \vn{Expression} based controllers
use mathematical expressions (\sref{s:arith}) to define how slave attribute values are calculated
based upon the values of the controller variables. \vn{Knot} based controllers use an array of
points (called ``\vn{knots}'') with either linear or cubic spline interpolation (in particular, the
cubic non-smoothing Akima spline\cite{b:akima}) to determine the relationship between variables and
slave attributes. With a knot based controller, the number of controller variables is restricted
to be one.

The general syntax for an expression based \vn{group} element is
\begin{example}
  name: GROUP = \{ele1[attrib1]:exp1, ele2[attrib2]:exp2, ...\}, 
          VAR = \{var1, var2, ...\}, var1 = init_val1, 
          old_var1 = old_init_val1, ..., GANG = logical, ...
\end{example}
For an expression based \vn{overlay} element, the syntax is identical except OVERLAY is
substituted for GROUP and there are no old values to set:
\begin{example}
  name: OVERLAY = \{ele1[attrib1]:exp1, ele2[attrib2]:exp2, ...\}, 
          VAR = \{var1, var2, ...\}, var1 = init_val1, GANG = logical, ...
\end{example}
For an expression based \vn{ramper} element, the syntax is identical to the \vn{overlay} element
except \vn{RAMPER} is used in place of \vn{OVERLAY} and there is no \vn{GANG} attribute.

In the above, \vn{Name} is the name of the controller element, \vn{ele1}, \vn{ele2}, ... are the
elements whose attributes are to be controlled, \vn{attrib1}, \vn{attrib2}, etc. are the controlled
attributes (called ``slave'' attributes), \vn{var1}, \vn{var2}, etc. are the control variables, and
\vn{exp1}, \vn{exp2}, etc. are the arithmetical expressions that define the relationship between the
variables and the slave attributes. Example:
\begin{example}
  gr1: group = \{q1[k1]:-tan(a)*b, q2[tilt]:b^2\}, var = \{a, b\}
\end{example}

The function \vn{ran()} and \vn{ran_gauss()} may be used in expressions with \vn{ramper} elements but
may not be used for other types of controllers. These functions can be useful with long-term tracking
to simulate such things as ground noise or RF jitter. Example
\begin{example}
  rf_noise: ramper = \{rfcavity::*[phi0]: 1e-4*ran_gauss()\}, var = \{ \}
\end{example}
Notice that the expression does not depend upon any variables and the variable list is empty. An empty
variable list is permissible only if all slave expressions involve \vn{ran()} or \vn{ran_gauss()}.
With an empty variable list, \bmad will define a variable for the \vn{ramper} called \vn{null} to serve
as a place holder for bookkeeping purposes. 

To define an expression based \vn{controller} element, two lists are needed: One list defines the
slave attributes along with the arithmetic expressions used for computing the value of the slave
attributes. The other list defines the variables within the \vn{controller} element that can be
varied. In the above example, the variables are \vn{a} and \vn{b}, and the slave attributes are the
\vn{k1} attribute of element \vn{q1} and the \vn{tilt} attribute of element \vn{q2}. The arithmetic
expressions used for the control are \vn{-tan(a)*b} and \vn{b\^{}2}.

The general syntax for For a \vn{knot} based \vn{group} element is:
\begin{example}
  name: GROUP = \{ele1[attrib1]:\{y_knot_points1\}, ele2[attrib2]:\{y_knot_points2\}, ...\}, 
              VAR = \{var1\}, X_KNOT = \{x_knot_points\}, INTERPOLATION = \{type\},
              var1 = init_val1, old_var1 = init_val_old1, ..., GANG = logical, ... 
\end{example}
and the general syntax for a \vn{knot} based \vn{overlay} element is similar:
\begin{example}
  name: OVERLAY = \{ele1[attrib1]:\{y_knot_points1\}, ele2[attrib2]:\{y_knot_points2\}, ...\}, 
        VAR = \{var1\}, X_KNOT = \{x_knot_points\}, var1 = init_val1, ..., 
        GANG = logical, INTERPOLATION = \{type\}, ...
\end{example}
For \vn{ramper} elements, the syntax is similar to the \vn{overlay} syntax except that \vn{OVERLAY}
is replaced by \vn{RAMPER} and there is no \vn{GANG} attribute. The \vn{x_knot} array is combined
with a given \vn{y_knot} array to give a series of $N$ knot points where $N$ is the size of the
\vn{x_knot} and \vn{y_knot} arrays. As such, the size of all knot arrays must be the same with the
additional restriction that the \vn{x_knot} array must have values that are strictly ascending.
Example:
\begin{example}
  ov: overlay = \{q1[k1]:\{-0.12, 0.23, 0.53\}, q2[k1], q3[k1], q4[k1]:\{0.37, 0.12, 0.92\}\}, 
        var = {time}, x_knot = \{0.0, 0.7, 1.0\}
\end{example}K
In this example there are three knot points for each slave parameter. For the slave parameter
\vn{q4[k1]} the knot points are $(0.0, 0.37)$, $(0.7, 0.12)$, and $(1.0, 0.92)$.  If a
\vn{y_knot_points} array is not present for a particular slave attribute, the knot array of the
previous slave attribute is used. Thus, in the above example, the y_knot array used for \vn{q2[k1]}
and \vn{q3[k1]} are the same as the y_knot array for \vn{q1[k1]}.

The method used to interpolate between the knot points is determined by the setting of the
\vn{interpolate} parameter which may be one of
\begin{example}
  linear      ! Linear interpolation.
  cubic       ! Cubic spline interpolation (default).
\end{example}
For variable values outside of the range specified by \vn{x_knot}, the slave parameter value will be
extrapolated using the interpolation function calculated from the three points nearest the end. That
is, extrapolation will be linear if \vn{interpolate} is set to \vn{linear} and will be cubic if
\vn{interpolate} is set to \vn{cubic}. For the cubic spline, extrapolation is only permitted over a
distance outside the x_knot range equal to the difference between the value of the x_knot end point
and its neighbor point. Thus in the above example, extrapolation will be allowed in the intervals
$[-0.7, 0.0]$ and $[1.0, 1.7]$.

After a controller has been defined, the individual knot points may be referenced using the syntax
\begin{example}
  name[x_knot(i)]           ! I^th x_knot point
  name[slave(j)%y_knot(i)]  ! I^th y_knot point of j^th slave.
\end{example}
For example:
\begin{example}
  ov: overlay = \{q1[k1]:\{-0.12, 0.23, 0.53\}, q2[tilt]:\{0.37, 0.12, 0.92\}\}, 
                          var = {time}, x_knot = \{0.0, 0.7, 1.0\}
  ov[x_knot(3)] = 2.0           ! Changes from 1.0 to 2.0
  ov[slave(2)%y_knot(2)] = 0.2  ! Changes q2[tilt] y_knot(2) from 0.12 to 0.2
\end{example}

If the \vn{gang} attribute is \vn{True} (which is the default) a \vn{controller} will control all
elements of a given name. Thus, in the above example, if there are multiple elements named \vn{q1}
then \vn{gr1} will control the \vn{k1} attribute of all of them.  If \vn{gang} is set to \vn{False},
a separate controller is created for each element in the lattice of a given name. For example:
\begin{example}
  gr1: group = \{q1[k1]:-tan(a)*b, q2[tilt]:b^2\}, var = \{a, b\}, gang = False
\end{example}
In this example, suppose there are five \vn{q1} and five \vn{q2} elements in the lattice.  In this
case, there will be five \vn{gr1} group elements created. The first \vn{gr1} will control the first
\vn{q1} and \vn{q2} to appear in the lattice, etc. With \vn{gang} set to \vn{False}, it is an error
if the number of instances in the lattice for a given slave name is different from any other slave
name. In this example, it would be an error if the number of \vn{q1} elements in the lattice is
different from the number of \vn{q2} elements in the lattice.

To vary coefficients in expressions, use variables in place of the coefficients and then variables
can be varied. For example:
\begin{example}
  gr1: group = \{s1[k2]: 0.03+1.07*vv\}, var = \{vv\}                     ! Instead of this...
  gr2: group = \{s1[k2]: a+b*vv\}, var = \{vv, a, b\}, a = 0.03, b = 1.07 !   Use this.
\end{example}

Wild card characters can be used in element names. Example:
\begin{example}
  rf_control: overlay = \{rfcavity::*[voltage]:v\}, var = \{v\}
\end{example}

The syntax for specifying an attribute \vn{attrib} of element \vn{ele} to be controlled is
\vn{ele[attrib]}. The attribute part \vn{[attrib]} may be omitted and in this case the name of the
attribute will be taken to be the name of the first variable. Example:
\begin{example}
  ov1: overlay = \{sex1:-tan(k2)^b\}, var = \{k2, b\}
  ov2: overlay = \{sex1[k2]: -tan(k2)^b\}, var = \{k2, b\} ! Equivalent to ov1.
\end{example}
In this example, the controlled attribute of element \vn{sex1} is \vn{k2}.  Except in cases where
this default attribute syntax is used, the names of the variables are arbitrary and do not have to
correspond to the name of any actual attribute. If the slave attribute name is set to ``\vn{*}'', 
multiple slaves will be generated, one for each variable. Example:
\begin{example}
  zz1: overlay = \{sb0[*]\}, var = \{g, e1\}   
  zz2: overlay = \{sb0[g]:g, sb0[e1]:e1\}, var = \{g, e1\} ! Equivalent to zz1.
\end{example}

The arithmetic expressions used to evaluate controlled attribute value changes may be a constant. In
this case, the actual expression used is this constant times the first variable. If the expression
is omitted entirely, along with the separating ``:'', the constant will be taken to be
unity. Example:
\begin{example}
  gr1: group = \{b1, b3:-pi\}, var = \{angle\}
\end{example}
This is equivalent to
\begin{example}
  gr1: group = \{b1[angle]:angle, b3[angle]:-pi*angle\}, var = \{angle\}
\end{example}
The exception is when \vn{ran()} or \vn{ran_gauss()} functions are present in the expression.  In
this case, the expression will \vn{not} be modified. Example:
\begin{example}
  quake: ramper = \{*[y_offset]: 1e-3*ran_gauss()\}, var = \{ \}
  quake: ramper = \{*[y_offset]: 1e-3*ran_gauss()\}, var = \{time\}  ! Equivalent to above
\end{example}

Arithmetic expressions may themselves contain element attributes.
Example:
\begin{example}
  sk_q20W: overlay = \{sex_20W[a1]:-sex_20W[L]\}, k1
\end{example}
Here the \vn{sk_q20w} overlay controls the \vn{a1} multipole attribute of element \vn{sex_20w} and
the length of \vn{sex_20w} is used as a scale factor between the overlay's variable \vn{k1} and the
controlled attribute \vn{a1}. The potential problem here is that, to keep the internal bookkeeping
simple, the value of \vn{sex_20w[L]} is evaluated once during parsing of the lattice file and never
reevaluated (\sref{s:arith}).  If it is desired to use a variable element attribute in an
expression, this may be effectively done by defining a control variable to take its place. Thus the
above overlay may be recast as:
\begin{example}
  sk_q20W: overlay = \{sex_20W[L]:ll, sex_20W[a1]:ll*k1\}, var = \{k1, ll\}
\end{example}

Initial values can be assigned to the variables from within the
definition of the controller element. Example:
\begin{example}
  ov1: overlay = \{...\}, var = \{a, b\}, a = 7, b = 2
\end{example}
Here the initial values 7 and 2 are assigned to \vn{a} and \vn{b} respectively.  Alternatively,
variables can be set after a controller element has been defined.  Example:
\begin{example}
  ov1: overlay = \{...\}, var = \{a, b\}
  ov1[a] = 7
  gr1[b] = 2
\end{example}

Note: There is an old deprecated syntax. For \vn{group} elements the syntax was:
\begin{example}
  name: GROUP = \{ele1[attrib1]:coef1, ele2[attrib2]:coef2, ...\}, 
                       attrib = init_value  ! DO NOT USE THIS SYNTAX!
\end{example}
and for \vn{overlay} elements the old syntax was identical except
that \vn{GROUP} was replaced by \vn{OVERLAY}:
\begin{example}
  name: OVERLAY = \{ele1[attrib1]:coef1, ele2[attrib2]:coef2, ...\}, 
                       attrib = init_value  ! DO NOT USE THIS SYNTAX!
\end{example}

With this old syntax, there is only one variable. Additionally, there are no arithmetic
expressions. Rather, attribute changes are linear in the \vn{command} variable with the constant of
proportionality given by a specified coefficient. For example with the old syntax
\begin{example}
  ov1: overlay = \{sq1:3.7, sq2[tilt]\}, k0 = 2  ! DO NOT USE THIS SYNTAX!
\end{example}
is equivalent, in the present syntax, to:
\begin{example}
  ov1: overlay = \{sq[k0]:3.7, sq2[tilt]\}, var = \{k0\}, k0 = 2  
\end{example}
Note: In this old syntax the colon ``:'' separating the controlled attribute from the linear
coefficient may be replaced by a slash ``/''.  For \vn{group} elements, there was an added wrinkle
that, with the old syntax, the variable's name is fixed to be \vn{command}. For example, with the
old syntax
\begin{example}
  gr1: group = \{sq1:3.7, sq2[tilt]\}, k0 = 2  ! DO NOT USE THIS SYNTAX!
\end{example}
is equivalent, in the present syntax, to:
\begin{example}
  gr1: group = \{sq[k0]:3.7, sq2[tilt]\}, var = \{command\}, command = 2  
\end{example}

%-----------------------------------------------------------------
\section[Energy and Wavelength Attributes]{Energy and Wavelength Attributes: E_tot, P0C, and \\ Ref_Wavelength}
\label{s:energy}

\index{parameter statement}\index{e_tot}
\index{e_tot_start}\index{p0c_start}
\index{patch}\index{lcavity}\index{p0c}\index{e_gun}
\index{ref_wave_length}
The attributes that define the reference energy and momentum at an element are:
\begin{example}
  e_tot  = <Real>  ! Total energy in eV.
  p0c    = <Real>  ! Momentum in eV.
\end{example}
The energy and momentum are defined at the exit end of the element.  For ultra--relativistic
particles, and for photons, these two values are the same (\sref{s:phase.space}). Except for
multipass elements (\sref{c:multipass}), \vn{e_tot} and \vn{p0c} are dependent attributes and,
except for multipass elements, any setting of \vn{e_tot} and \vn{p0c} in the lattice input file is
an error. The value of \vn{e_tot} and \vn{p0c} for an element is calculated by \bmad to be the same
as the previous element except for \vn{e_gun}, \vn{lcavity}, \vn{converter}, and \vn{patch}
elements. To set the \vn{e_tot} or \vn{p0c} at the start of the lattice use the \vn{beginning} or
\vn{parameter} statements.  See~\sref{s:param}. Since the reference energy changes from the start to
the end of an \vn{lcavity}, \vn{converter} or \vn{em_field} element, these elements have the
dependent attributes
\index{em_field}\index{lcavity}
\begin{example}
  e_tot_start   and
  p0c_start
\end{example}
which are just the reference energy and momentum at the start of the element.

\index{beginning_ele}
The \vn{beginning_ele} element (\sref{s:begin.ele}) also has associated \vn{e_tot_start} and
\vn{p0c_start} attributes as well as \vn{e_tot} and \vn{p0c}. Generally, for a \vn{beginning_ele},
\vn{p0c_start} and \vn{p0c} are the same and \vn{e_tot_start} and \vn{e_tot} are the same and the
values for these attributes are set in the lattice file with the appropriate \vn{parameter}
(\sref{s:param}) or \vn{beginning} (\sref{s:beginning}) statement. The exception occurs when there
is an \vn{e_gun} element in the lattice (\sref{s:e.gun}). In this case, the \vn{p0c_start} and
\vn{e_tot_start} attributes of the \vn{beginning_ele} are set to the values as set in the lattice
file and \vn{e_tot} is set to
\begin{example}
  e_tot = e_tot_start + voltage
\end{example}
and \vn{p0c} is calculated from \vn{e_tot} and the mass of the particle being tracked. For example,
if the lattice file contained:
\begin{example}
  beginning[p0c] = 0
  gun: e_gun, voltage = 0.5e6
  injector: line = (gun, ...)
\end{example}
Then the following energy values will be set for the beginning \vn{beginning_ele} element:
\begin{example}
  p0c_start   = 0
  e_tot_start = mc2
  e_tot       = mc2 + 0.5e6
  p0c         = Sqrt(e_tot - mc2^2)
\end{example}
where \vn{mc2} is the particle rest mass.  The reason for using this convoluted convention is to
allow the setting, in the lattice file, of a zero reference momentum at the start of the lattice,
while avoiding the calculational problems that would occur if the \vn{e_gun} element truly had a
starting reference momentum of zero.  Specifically, the problem with zero reference momentum is that
the phase space momentum would be infinity as can be seen from \Eqs{ppp}.

For lattice branches other than the root branch, the reference energy or momentum is set using the
line name of the branch (\sref{s:beginning}). For example:
\begin{example}
  ff: fork, to_line = line_2nd, ...
  line_2nd: line = (...)
  line_2nd[particle] = He+
  line_2nd[p0c] = 1e9
\end{example}

For \vn{multipass} elements, the reference energy may be set by specifying one of \vn{e_tot},
\vn{p0c}, as described in \sref{c:multipass}.

For photons, the reference wavelength, \vn{ref_wavelength} is also a dependent attribute calculated
from the reference energy.

\vfill

%-----------------------------------------------------------------
\section{Orientation: Offset, Pitch, Tilt, and Roll Attributes}
\label{s:offset}
\index{x_offset|hyperbf}
\index{y_offset|hyperbf}\index{z_offset|hyperbf}
\index{x_pitch|hyperbf}\index{y_pitch|hyperbf}
\index{roll|hyperbf}\index{tilt|hyperbf}

By default, an element, like a quadrupole, is aligned in space coincident with the reference orbit
running through it (\sref{s:ref.construct}). A quadrupole can be displaced in space using the
quadrupole's ``\vn{orientational}'' attributes. For a quadrupole, the orientational attributes only
affect the physical element and not the reference orbit. However, the orientational attributes of
some other elements, like the \vn{fiducial} element, do affect the reference orbit. To sort all this
out, lattice elements can be divided into seven classes:
  \begin{enumerate}
%
\item\vn{Straight line elements} (\sref{s:straight.orient}) \Newline
Straight line elements are elements where the reference orbit is a straight line. Examples include
\vn{quadrupoles}, and \vn{sextupoles} as well as zero length elements like \vn{markers}.
%
\item\vn{Dipole bends} (\sref{s:bend.orient}) \Newline
Dipole bends are:
\begin{example}
  sbend \& rbend
\end{example}
%
\item\vn{Photon reflecting elements} (\sref{s:photon.orient}) \Newline
The reflecting elements are
\begin{example}
  crystal
  mirror
  multilayer_mirror
\end{example}
These elements have a kink in the reference orbit at the nominal
element surface.
%
\item\vn{Reference orbit manipulator elements} (\sref{s:manip.orient}) \Newline
Elements that are used to manipulate the reference orbit are
\begin{example}
  fork \& photon_fork
  floor_shift
  patch
\end{example}
%
\item\vn{Fiducial Element} (\sref{s:fiducial.orient}) \Newline
%
\item\vn{Girder Elements} (\sref{s:girder.orient}) \Newline
%
\item\vn{Control Elements} \Newline
Control elements are elements that control attributes of other elements. Except for \vn{girder}
control elements, these elements do not have orientational attributes. Control elements that fall
into this list are:
\begin{example}
  group
  overlay
  ramper
\end{example}
  \end{enumerate}

%-----------------------------------------------------------------
\subsection{Global Random Misalignment of Elements}
\label{s:global.mis}

It is often convenient to randomly misalign sets of elements. This can be done using the \vn{ran}
and \vn{ran_gauss} functions (\sref{s:functions}). For example:
\begin{example}
  quadrupole::bnd10:bnd20[y_offset] = 1.4e-6*ran_gauss(5)
\end{example}
The above line sets the \vn{y_offset} of all the quadrupoles in the range from element \vn{BND10} to
element \vn{BND20}. 

When \vn{ran} or \vn{ran_gauss} is used in a lattice file, each time the file is read in, a new set
of random numbers are generated unless \vn{parameter[ran_seed]} is set to a non-zero value in the
lattice file before \vn{ran} or \vn{ran_gauss} is used. To save a particular set of generated 
random values, write out the lattice file (the \vn{write bmad} command can be used if running \tao)
after it has been read in.

%-----------------------------------------------------------------
\subsection{Straight Line Element Orientation}
\label{s:straight.orient}

The straight line elements have the following orientational attributes:
\begin{example}
  x_offset = <Real>
  y_offset = <Real>
  z_offset = <Real>
  x_pitch  = <Real>
  y_pitch  = <Real>
  tilt     = <Real>    
\end{example}
For straight line elements the orientational attributes only shift the
physical element and do not affect the reference orbit.

\begin{figure}[tb]
  \centering
  \includegraphics{pitch.pdf}
  \caption{Geometry of Pitch and Offset attributes}
  \label{f:pitch}
\end{figure}

\vn{x_offset} translates an element in the local $x$--direction as shown in
\fig{f:pitch}. Similarly, \vn{y_offset} and \vn{z_offset} translate an element along the local $y$
and $z$--directions respectively.

The \vn{x_pitch} attribute rotates an element about the element's center such that with a positive
\vn{x_pitch} the exit face of the element is displaced in the $+x$--direction as shown in
figure~\ref{f:pitch}. [One way to visualize the effect of an \vn{x_pitch} is to think of the element
as an airplane pointing in the $+z$ direction. A positive \vn{x_pitch} would then move the front of
the plane in the $+x$--direction.] An\vn{x_pitch} represents a rotation around the positive
$y$-axis.

Similarly, the \vn{y_pitch} attribute rotates an element about the element's center using the
negative $x$--axis as the rotation axis so that, with a positive \vn{y_pitch} the exit face of the
element is displaced in the $+y$--direction.

Note: the \vn{x_pitch} and \vn{y_pitch} rotations are about the center of the element which is in contrast
to the \vn{dtheta} and \vn{dphi} misalignments of \mad which rotate around the entrance point. The
sense of the rotation between \bmad and MAD is:
\index{MAD!element rotation origin}
\begin{example}
  x_pitch (Bmad) =  dtheta (MAD)
  y_pitch (Bmad) = -dphi (MAD)
\end{example}

\begin{figure}[tb]
  \centering
  \includegraphics{tilt.pdf}
  \caption{Geometry of a Tilt}
  \label{f:tilt}
\end{figure}

The tilt attribute rotates the element in the $(x, y)$ plane as shown in figure~\ref{f:tilt}. The
rotation axis is the positive $z$-axis. For example
\begin{example}
  q1: quad, l = 0.6, x_offset = 0.03, y_pitch = 0.001, tilt
\end{example}
\index{sol_quad!tilt default}\index{quadrupole!tilt default}
\index{sextupole!tilt default}\index{octupole!tilt default}
Like MAD, \bmad allows the use of the \vn{tilt} attribute without a value to designate a skew
element. The default tilt is $\pi/(2(n+1))$ where $n$ is the order of the element:
\begin{example}
  sol_quad       n = 1
  quadrupole     n = 1
  sextupole      n = 2
  octupole       n = 3
\end{example}

Note that \vn{hkick} and \vn{vkick} attributes are not affected by \vn{tilt} except for \vn{kicker}
and \vn{elseparator} elements.

%-----------------------------------------------------------------
\subsection{Bend Element Orientation}
\label{s:bend.orient}

\begin{figure}[ht]
  \centering
  \includegraphics{roll.pdf}
  \caption[Geometry of a Bend]{
Geometry of a Bend. Like straight line elements, offsets and pitches are calculated with respect to
the coordinates at the center of the bend. The exception is the \vn{roll} attribute which is a
rotation around the axis passing through the entrance and exit points.  Shown here is the geometry
for a bend with \vn{ref_tilt} = 0. That is, the bend is in the $x-z$ plane.}
  \label{f:roll}
\end{figure}

The orientation attributes for \vn{sbend}, \vn{rbend} and \vn{rf_bend} elements is
\begin{example}
  x_offset = <Real>
  y_offset = <Real>
  z_offset = <Real>
  x_pitch  = <Real>
  y_pitch  = <Real>
  ref_tilt = <Real>    ! Shifts and reference orbit rotation axis.
  roll     = <Real>    
\end{example}
The geometry for orienting a bend is shown in \fig{f:roll}. Like straight line elements, the offset
and pitch attributes are evaluated with respect to the center of the element.

Unlike the straight line elements, bends do not have a \vn{tilt} attribute. Rather they have a
\vn{ref_tilt} and a \vn{roll} attribute.  The \vn{roll} attribute rotates the bend along an axis
that runs through the entrance point and exit point as shown in figure~\ref{f:roll}. A \vn{roll}
attribute, like the offset and pitch attributes does not affect the reference orbit.  The major
effect of a \vn{roll} is to give a vertical kick to the beam. For a bend with positive bend angle, a
positive \vn{roll} will move the outside portion ($+x$ side) of the bend upward and the inside
portion (-$x$ side) downward. Much like car racetracks which are typically slanted towards the
inside of a turn.

The \vn{ref_tilt} attribute of a bend rotates the bend about the $z$ axis at the upstream end of the
bend as shown in \fig{f:roll}. Unlike \vn{rolls} and \vn{tilts}, \vn{ref_tilt} also shifts the
rotation axis of the reference orbit along with the physical element. A \vn{bend} with a
\vn{ref_tilt} of $\pi/2$ will bend a beam vertically downward (\sref{s:global}). Note that the
\vn{ref_tilt} attribute of \bmad is the same as the MAD \vn{tilt} attribute.

Important! Do not use \vn{ref_tilt} when doing misalignment studies for a machine. Trying to misalign
a dipole by setting \vn{ref_tilt} will affect the positions of all downstream elements! Rather, use the
\vn{roll} parameter.

%-----------------------------------------------------------------
\subsection{Photon Reflecting Element Orientation}
\label{s:photon.orient}

\begin{figure}[ht]
  \centering
  \includegraphics{reflect-orient.pdf}
  \caption[Geometry of a photon reflecting element orientation]{
Geometry of a photon reflecting element orientation.  The reference coordinates used for defining
the orientational attribute is the entrance reference coordinates.  }
  \label{f:reflect.orient}
\end{figure}

Photon reflecting elements have the following orientational attributes:
\begin{example}
  x_offset = <Real>
  y_offset = <Real>
  z_offset = <Real>
  x_pitch  = <Real>
  y_pitch  = <Real>
  ref_tilt = <Real>    ! Shifts both element and reference orbit.
  tilt     = <Real>    
\end{example}
Roughly, these elements can be viewed as zero length bends except, since there is no center
position, the orientational attributes are defined with respect to the entrance coordinates as shown
in \fig{f:reflect.orient}. Like bend elements, the \vn{ref_tilt} attribute rotates both the physical
element and the reference coordinates.  The \vn{tilt} attribute rotates just the physical
element. Thus the total rotation of the physical element about the entrance $z$ axis is the sum
\vn{tilt} + \vn{ref_tilt}.

Frequently, it is desired to orient reflecting elements with respect to the element's surface. This
can be done using a \vn{girder} element (\sref{s:girder}) which supports the reflecting element and
with the \vn{girder}'s \vn{origin_ele_ref_pt} attribute set to \vn{center}.

%-----------------------------------------------------------------
\subsection{Reference Orbit Manipulator Element Orientation}
\label{s:manip.orient}

The \vn{fork}, \vn{photon_fork}, \vn{floor_shift}, and \vn{patch} elements use the following
attributes to orient their exit edge with respect to their entrance edge:
\begin{example}
  x_offset = <Real>
  y_offset = <Real>
  z_offset = <Real>
  x_pitch  = <Real>
  y_pitch  = <Real>
  tilt     = <Real>    
\end{example}
Here "exit" edge for \vn{fork} and \vn{photon_fork} elements is defined to be the start of the line
being branched to. [Within the line containing the fork, the \vn{fork} element is considered to have
zero length so the exit face in the line containing the fork is coincident with the entrance face.]
The placement of the exit edge for these elements defines the reference orbit.  Thus, unlike the
corresponding attributes for other elements, the orientational attributes here directly control the
reference orbit.

%-----------------------------------------------------------------
\subsection{Fiducial Element Orientation}
\label{s:fiducial.orient}

The \vn{fiducial} element (\sref{s:girder}) uses the 
following attributes to define its position:
\begin{example}
  origin_ele        = <Name>     ! Reference element.
  origin_ele_ref_pt = <location> ! Reference pt on reference ele.
  dx_origin         = <Real>     ! x-position offset
  dy_origin         = <Real>     ! y-position offset
  dz_origin         = <Real>     ! z-position offset
  dtheta_origin     = <Real>     ! orientation angle offset.
  dphi_origin       = <Real>     ! orientation angle offset.
  dpsi_origin       = <Real>     ! orientation angle offset.
\end{example}
See Section~\sref{s:fiducial} for more details.

%-----------------------------------------------------------------
\subsection{Girder Orientation}
\label{s:girder.orient}

A \vn{girder} (\sref{s:girder}) element uses the same attributes as a \vn{fiducial} element
(\sref{s:fiducial}) to orient the reference girder position. In addition, the following attributes
are used to move the girder physically from the reference position:
\begin{example}
  x_offset = <Real>
  y_offset = <Real>
  z_offset = <Real>
  x_pitch  = <Real>
  y_pitch  = <Real>
  tilt     = <Real>    
\end{example}
Shifting the girder from its reference position shifts all the elements that are supported by the
girder. See Section~\sref{s:girder} for more details.

\index{x_offset_tot|hyperbf}\index{y_offset_tot|hyperbf}\index{z_offset_tot|hyperbf}
\index{x_pitch_tot|hyperbf}\index{y_pitch_tot|hyperbf}
\index{tilt_tot|hyperbf}\index{roll_tot|hyperbf}\index{tilt_err_tot|hyperbf}
If an element is supported by a \vn{girder} element (\sref{s:girder}), the orientational attributes
of the element are with respect to the orientation of the \vn{girder}. The computed offsets, pitches
and tilt with respect to the local reference coordinates are stored in the dependent attributes
\begin{example}
  x_offset_tot
  y_offset_tot
  z_offset_tot
  x_pitch_tot
  y_pitch_tot
  tilt_tot
  roll_tot
\end{example}
\index{sbend}\index{rbend}
A \vn{*_tot} attribute will only be present if the corresponding non \vn{*_tot} attribute is
present. For example, only \vn{sbend} and \vn{rbend} elements have a \vn{roll_tot} attribute since
only these elements have a \vn{roll} attribute.

If an element is not supported by a \vn{girder}, the values of the \vn{*_tot} attributes will be the
same value as the values of the corresponding non \vn{*_tot} attributes.

%-----------------------------------------------------------------
\section{Hkick, Vkick, and Kick Attributes}
\label{s:kick}
\index{hkick|hyperbf}\index{bl_hkick|hyperbf}
\index{vkick|hyperbf}\index{bl_vkick|hyperbf}
\index{kick|hyperbf}\index{bl_kick|hyperbf}


\index{hkicker}
\index{vkicker}
\index{elseparator}
\index{kicker}
The kick attributes that an element may have are:
\begin{example}
  kick,  bl_kick  = <Real>  ! Used only with a Hkicker or Vkicker
  hkick, bl_hkick = <Real>
  vkick, bl_vkick = <Real>
\end{example}
\vn{kick}, \vn{hkick}, and \vn{vkick} attributes are the integrated kick of an element in
radians. \vn{kick} is only used for \vn{hkicker} and \vn{vkicker} elements. All other elements that
can kick use \vn{hkick} and \vn{vkick}. The \vn{tilt} attribute will only rotate a kick for
\vn{hkicker}, \vn{vkicker}, \vn{elseparator} and \vn{kicker} elements. This rule was implemented so
that, for example, the \vn{hkick} attribute for a skew quadrupole would represent a horizontal
steering. The \vn{bl_kick}, \vn{bl_hkick}, and \vn{bl_vkick} attributes are the integrated field
kick in \vn{meters-Tesla}. Normally these are dependent attributes except if they appear in the
lattice file (\sref{s:depend}).

For an \vn{elseparator} element, the \vn{hkick} and \vn{vkick} are appropriate for a positively
charged particle. The kick for a negatively charged particle is opposite this.

%-----------------------------------------------------------------
\section{Aperture and Limit Attributes}
\label{s:limit}
\index{aperture|hyperbf}
\index{limit|hyperbf}
\index{aperture_at|hyperbf}

\begin{figure}[ht]
  \centering
  \includegraphics{apertures.pdf}
  \caption[Apertures for ecollimator and rcollimator elements.]
  {Apertures for ecollimator and rcollimator elements. 
  [note: positive $z$ points up, out of the page.] 
  As drawn, all limits \vn{x1_limit}, \vn{x2_limit}, 
  \vn{y1_limit}, \vn{y2_limit} are  positive.}  
  \label{f:limit}
\end{figure}

\index{ecollimator}
\index{rcollimator}
\index{x_limit|hyperbf}
\index{y_limit|hyperbf}
\index{x1_limit|hyperbf}
\index{y1_limit|hyperbf}
\index{x2_limit|hyperbf}
\index{y2_limit|hyperbf}
\index{x_offset|hyperbf}
\index{offset_moves_aperture|hyperbf}
\index{aperture_type}
The aperture attributes are:
\begin{example}
  x1_limit      = <Real>      ! Horizontal, negative side, aperture limit
  x2_limit      = <Real>      ! Horizontal, positive side, aperture limit
  y1_limit      = <Real>      ! Vertical, negative side, aperture limit
  y2_limit      = <Real>      ! Vertical, positive side, aperture limit
  x_limit       = <Real>      ! Alternative to specifying x1_limit and x2_limit
  y_limit       = <Real>      ! Alternative to specifying y1_limit and y2_limit
  aperture      = <Real>      ! Alternative to specifying x_limit and y_limit
  aperture_at   = <Switch>    ! What end aperture is at. (\sref{s:ap.place})
  aperture_type = <Switch>    ! What type of aperture it is
  offset_moves_aperture = <Logical> ! Element offsets affect aperture position (\sref{s:offset.ap})
\end{example}
\vn{x1_limit}, \vn{x2_limit}, \vn{y1_limit}, and \vn{y2_limit} specify the half--width of the
aperture of an element as shown in figure~\ref{f:limit}. Note: Normally all of these will be zero or
positive. A zero \vn{x1_limit}, \vn{x2_limit}, \vn{y1_limit}, or \vn{y2_limit} is interpreted as no
aperture in the appropriate plane.

For convenience, \vn{x_limit} can be used to set \vn{x1_limit} and
\vn{x2_limit} to a common value. Example:
\begin{example}
  s: sextupole, x1_limit = 0.09, x2_limit = 0.09
  s: sextupole, x_limit = 0.09   ! Same as above
\end{example}
Similarly, \vn{y_limit} can be used
to set \vn{y1_limit} and \vn{y2_limit}.  The \vn{aperture} attribute
can be use to set all four \vn{x1_limit}, \vn{x2_limit}, \vn{y1_limit}
and \vn{y2_limit} to a common value. Internally, the \bmad code does {\em not}
store \vn{x_limit}, \vn{y_limit}, or \vn{aperture}. This means that
using \vn{x_limit}, \vn{y_limit} or aperture in arithmetic expressions is
an error:
\begin{example}
  q1: quad, aperture = 0.09         
  q2: quad, aperture = q1[aperture]   ! THIS IS AN ERROR!
  q2: quad, aperture = q1[x1_limit]   ! Correct
\end{example}

By default, apertures are assumed to be rectangular except that an \vn{ecollimator} has a elliptical
aperture. This can be changed by setting the \vn{aperture_type} attribute. The possible values of
this attribute are:
\begin{example}
  auto         ! Default for detector, mask and diffraction_plate elements
  custom
  elliptical   ! Default for \vn{ecollimator} elements.
  rectangular  ! Default for most elements.
  wall3d       ! Vacuum chamber wall (\sref{s:wall}).
\end{example}
The \vn{custom} setting is used in the case where programs have been compiled with custom, non-Bmad,
code to handle the aperture calculation.  The \vn{auto} setting is used for automatic calculation of
a rectangular aperture. For \vn{diffraction_plate} and \vn{mask} elements, the \vn{auto} setting
causes the four aperture limits to be set to just cover the clear area of element
(\sref{s:masking.wall}). For all other elements, the \vn{auto} setting is only to be used when there
is an associated surface grid (\sref{s:surf.grid}) for the element and, in this case, \bmad to set
the four limits to just cover the surface grid.

The \vn{wall3d} setting uses the vacuum chamber wall as specified by a \vn{wall} attribute
(\sref{s:wall}). Using the \vn{wall} construct allows for complex apertures to be
constructed. Note that The wall thickness and material type are not used when calculating
if a particle has hit the wall. That is, the wall is considered to be infinitely thin.
Also note that a wall must cover the entire length of the element longitudinally. This is done
in order to be able to spot errors in specifying the wall geometry.

For elliptical apertures, all four \vn{x1_limit}, \vn{x2_limit}, \vn{y1_limit}, and \vn{y2_limit}
must be non-zero.

For rectangular apertures, the limits \vn{x1_limit}, \vn{x2_limit}, \vn{y1_limit}, or \vn{y2_limit}
may be negative. For example:
\begin{example}
  s: sextupole, x1_limit = -0.02, x2_limit = 0.09
\end{example}
In this case, particles will hit the aperture if their $x$-coordinate is outside the interval [0.02,
0.09]. That is, particles at the origin will be lost.

To avoid numerical overflow and other errors in tracking, a particle will be considered to have hit
an aperture in an element, even if there are no apertures set for that element, if its orbit exceeds
1000 meters. Additionally, there are other situations where a particle will be considered lost. For
example, if a particle's trajectory does not intersect the output face in a bend.

Examples:
\begin{example}
  q1, quadrupole, y1_limit = 0.03
  q1[y2_limit] = 0.03
  q1[y_limit] = 0.03  ! equivalent to the proceeding 2 lines.  
  q1[aperture_at] = both_ends
\end{example}

%-----------------------------------------------------------------
\subsection{Apertures and Element Offsets}
\label{s:offset.ap}

\index{tilt}
\index{x_offset}
\index{y_offset}
\index{x_pitch}
\index{y_pitch}
\index{rcollimator}\index{ecollimator}
\index{multilayer_mirror}\index{mirror}\index{crystal}
Normally, whether a particle hits an aperture or not is evaluated independent of any element offsets
(\sref{s:offset}). This is equivalent to the situation where a beam pipe containing an aperture is
independent of the placement of the physical element the beam pipe passes through. That is, the beam
pipe does not ``touch'' the physical element. This can be changed by setting the
\vn{offset_moves_aperture} attribute to \vn{True}. In this case any offsets or pitches will be
considered to have shifted the aperture boundary. The exceptions here is that the default for the
following elements is for \vn{offset_moves_aperture} to be \vn{True}:
\begin{example}
  rcollimator, 
  ecollimator,
  multilayer_mirror, 
  mirror, and 
  crystal 
\end{example}

Even with \vn{offset_moves_aperture} set to \vn{True}, \vn{tilt}s will not affect the aperture
calculation. This is done, for example, so that the tilt of a skew quadrupole does not affect the
aperture. The exception here is that tilting an \vn{rcollimator} or \vn{ecollimator} element will
tilt the aperture. Additionally, when the aperture is at the \vn{surface} (see below), any \vn{tilt}
will be used in the calculation.

Example:
\begin{example}
  q1: quad, l = 0.6, x1_limit = 0.045, offset_moves_aperture = T
\end{example}

%-----------------------------------------------------------------
\subsection{Aperture Placement}
\label{s:ap.place}

\index{both_ends|hyperbf}\index{continuous|hyperbf}
\index{entrance_end|hyperbf}\index{exit_end|hyperbf}\index{wall_transition|hyperbf}
\index{surface|hyperbf}\index{aperture_at}\index{no_aperture}
By default, for most elements, the aperture is evaluated at the exit face of the element. This can
be changed by setting the \vn{aperture_at} attribute.  Possible settings for \vn{aperture_at} are:
\begin{example}
  both_ends
  continuous
  entrance_end
  exit_end       ! Default for most elements
  no_aperture
  surface  
  wall_transition
\end{example}
The \vn{exit_end} setting is the default for most elements except for
the following elements who have a default of \vn{surface}:
\index{mirror}\index{multilayer_mirror}\index{crystal}
\index{diffraction_plate}\index{sample}
\begin{example}
  crystal
  detector
  diffraction_plate
  mask
  mirror
  multilayer_mirror
  sample
\end{example}

In fact, for the following elements:
\begin{example}
  mirror, 
  multilayer_mirror
  crystal
\end{example}
The \vn{surface} setting for \vn{aperture_at} must be used.  Additionally, due to the complicated
geometry of these elements, to keep things conceptionally simple, the rule is imposed that, for an
aperture at the surface, the \vn{offset_moves_aperture} setting must be left in its default state of
True. Additionally, For \vn{entrance_end} or \vn{exit_end} apertures, \vn{offset_moves_aperture}
must be set to False.

Note: The entrance and exit ends of an element are independent of which direction particles are
tracked through an element. Thus if a particle is tracked backwards it enters an element at the
``exit end'' and exits at the ``entrance end''. The \vn{continuous} setting indicates that the
aperture is continuous along the length of the element. This only matters when particle tracking
involves stepping through an element a little bit at a time. For example, as in Runge-Kutta tracking
(\sref{s:tkm}). For tracking where a formula is used to transform the particle coordinates at the
entrance of an element to the coordinates at the exit end, the aperture is only checked at the end
points so, in this situation, a \vn{continuous} aperture is equivalent to the \vn{both_ends}
setting.

The \vn{wall_transition} setting is like the \vn{continuous} setting in that the aperture boundary
is considered to be continuous along the element's length. However, unlike the \vn{continuous}
setting, with the \vn{wall_transition} setting a particle outside the wall is considered alive and
it is only when a particle moves through the wall that it is lost.  The \vn{wall_transition} setting
is used for things like septum magnets where a particle may be safely outside or inside the
wall. Note to programmers: By supplying a custom \vn{wall_hit_handler_custom} routine, scattering of
particles through a wall may be simulated.

Examples:
\begin{example}
  q2: quad, aperture_type = elliptical, aperture_at = continuous
  q1: quad, l = 0.6, x1_limit = 0.045, offset_moves_aperture = T
\end{example}

%-----------------------------------------------------------------
\subsection{Apertures and X-Ray Generation}
\label{s:aper.x.ray}

With X-ray simulation apertures can be used by \bmad to limit the directions in which photons are
generated. This can greatly decrease simulation times. For example, a photon passing through a
\vn{diffraction_plate} element will diffract in an arbitrary direction. If a {\em downstream}
element has an aperture set, \bmad can restrict the velocity directions so that the photons will
fill the downstream aperture and the amount of time wasted tracking photons that ultimately would be
collimated is minimal.

%-----------------------------------------------------------------
\section{X-Rays Crystal \& Compound Materials}
\label{s:cryst.list}

For basic crystallographic and X-ray matter interaction cross-sections, \bmad uses the
XRAYLIB\cite{b:xraylib} library. Crystal structure parameters in XRAYLIB are mainly from
R.~W.~G.~Wyckoff\cite{b:wyckoff} with some structure parameters coming from NIST. The list of
available structures is:
\begin{center}
\begin{tabular}{llll} \toprule
AlphaAlumina & GaP       & KCl        & Platinum  \\
AlphaQuartz  & GaSb      & KTP        & RbAP      \\
Aluminum     & Ge        & LaB6       & Sapphire  \\
Be           & Gold      & LaB6_NIST  & Si        \\
Beryl        & Graphite  & LiF        & Si_NIST   \\
Copper       & InAs      & LiNbO3     & Si2       \\
CsCl         & InP       & Muscovite  & SiC       \\
CsF          & InSb      & NaCl       & Titanium  \\
Diamond      & Iron      & PET        & TlAP      \\
GaAs         & KAP       &            &           \\ \bottomrule
\end{tabular}
\end{center}
These names are case sensitive

Besides the above crystal list, \bmad can calculate structure factors for all the elements and the
following list of materials. Material properties are from NIST. These names are case sensitive. That
is, the NIST materials all use upper case. As noted in the table, several of the materials may be
specified using the appropriate chemical formula. For example, liquid water may be referenced using
the name \vn{H2O}.
\begin{center}
\footnotesize
\begin{longtable}{lll}
\multicolumn{3}{r}{{\normalsize Continued on next page}} \\
\endfoot
\endlastfoot
A_150_TISSUE_EQUIVALENT_PLASTIC     & LITHIUM_TETRABORATE                       \\
ACETONE                             & LUNG_ICRP                                 \\
ACETYLENE                           & M3_WAX                                    \\
ADENINE                             & MAGNESIUM_CARBONATE                       \\
ADIPOSE_TISSUE_ICRP                 & MAGNESIUM_FLUORIDE                        \\
AIR_DRY_NEAR_SEA_LEVEL              & MAGNESIUM_OXIDE                           \\
ALANINE                             & MAGNESIUM_TETRABORATE                     \\
ALUMINUM_OXIDE, Al2O3               & MERCURIC_IODIDE                           \\
AMBER                               & METHANE                                   \\
AMMONIA, NH3                        & METHANOL                                  \\
ANILINE                             & MIX_D_WAX                                 \\
ANTHRACENE                          & MS20_TISSUE_SUBSTITUTE                    \\
B_100_BONE_EQUIVALENT_PLASTIC       & MUSCLE_SKELETAL                           \\
BAKELITE                            & MUSCLE_STRIATED                           \\
BARIUM_FLUORIDE                     & MUSCLE_EQUIVALENT_LIQUID_WITH_SUCROSE     \\
BARIUM_SULFATE                      & MUSCLE_EQUIVALENT_LIQUID_WITHOUT_SUCROSE  \\
BENZENE, C6H6                       & NAPHTHALENE                               \\
BERYLLIUM_OXIDE                     & NITROBENZENE                              \\
BISMUTH_GERMANIUM_OXIDE             & NITROUS_OXIDE                             \\
BLOOD_ICRP                          & NYLON_DU_PONT_ELVAMIDE_8062               \\
BONE_COMPACT_ICRU                   & NYLON_TYPE_6_AND_TYPE_6_6                 \\
BONE_CORTICAL_ICRP                  & NYLON_TYPE_6_10                           \\
BORON_CARBIDE, B4C                  & NYLON_TYPE_11_RILSAN                      \\
BORON_OXIDE, B2O3                   & OCTANE_LIQUID                             \\
BRAIN_ICRP                          & PARAFFIN_WAX                              \\
BUTANE                              & N_PENTANE                                 \\
N_BUTYL_ALCOHOL                     & PHOTOGRAPHIC_EMULSION                     \\
C_552_AIR_EQUIVALENT_PLASTIC        & PLASTIC_SCINTILLATOR_VINYLTOLUENE_BASED   \\
CADMIUM_TELLURIDE                   & PLUTONIUM_DIOXIDE                         \\
CADMIUM_TUNGSTATE                   & POLYACRYLONITRILE                         \\
CALCIUM_CARBONATE                   & POLYCARBONATE_MAKROLON_LEXAN              \\
CALCIUM_FLUORIDE                    & POLYCHLOROSTYRENE                         \\
CALCIUM_OXIDE                       & POLYETHYLENE                              \\
CALCIUM_SULFATE                     & POLYETHYLENE_TEREPHTHALATE_MYLAR          \\
CALCIUM_TUNGSTATE                   & POLYMETHYL_METHACRALATE_LUCITE_PERSPEX    \\
CARBON_DIOXIDE                      & POLYOXYMETHYLENE                          \\
CARBON_TETRACHLORIDE                & POLYPROPYLENE                             \\
CELLULOSE_ACETATE_CELLOPHANE        & POLYSTYRENE                               \\
CELLULOSE_ACETATE_BUTYRATE          & POLYTETRAFLUOROETHYLENE_TEFLON            \\
CELLULOSE_NITRATE                   & POLYTRIFLUOROCHLOROETHYLENE               \\
CERIC_SULFATE_DOSIMETER_SOLUTION    & POLYVINYL_ACETATE                         \\
CESIUM_FLUORIDE                     & POLYVINYL_ALCOHOL                         \\
CESIUM_IODIDE                       & POLYVINYL_BUTYRAL                         \\
CHLOROBENZENE                       & POLYVINYL_CHLORIDE                        \\
CHLOROFORM                          & POLYVINYLIDENE_CHLORIDE_SARAN             \\
CONCRETE_PORTLAND                   & POLYVINYLIDENE_FLUORIDE                   \\
CYCLOHEXANE                         & POLYVINYL_PYRROLIDONE                     \\
12_DDIHLOROBENZENE                  & POTASSIUM_IODIDE                          \\
DICHLORODIETHYL_ETHER               & POTASSIUM_OXIDE                           \\
12_DICHLOROETHANE                   & PROPANE                                   \\
DIETHYL_ETHER                       & PROPANE_LIQUID                            \\
NN_DIMETHYL_FORMAMIDE               & N_PROPYL_ALCOHOL                          \\
DIMETHYL_SULFOXIDE                  & PYRIDINE                                  \\
ETHANE                              & RUBBER_BUTYL                              \\
ETHYL_ALCOHOL                       & RUBBER_NATURAL                            \\
ETHYL_CELLULOSE                     & RUBBER_NEOPRENE                           \\
ETHYLENE                            & SILICON_DIOXIDE                           \\
EYE_LENS_ICRP                       & SILVER_BROMIDE                            \\
FERRIC_OXIDE                        & SILVER_CHLORIDE                           \\
FERROBORIDE                         & SILVER_HALIDES_IN_PHOTOGRAPHIC_EMULSION   \\
FERROUS_OXIDE                       & SILVER_IODIDE                             \\
FERROUS_SULFATE_DOSIMETER_SOLUTION  & SKIN_ICRP                                 \\
FREON_12                            & SODIUM_CARBONATE                          \\
FREON_12B2                          & SODIUM_IODIDE                             \\
FREON_13                            & SODIUM_MONOXIDE                           \\
FREON_13B1                          & SODIUM_NITRATE                            \\
FREON_13I1                          & STILBENE                                  \\
GADOLINIUM_OXYSULFIDE               & SUCROSE                                   \\
GALLIUM_ARSENIDE                    & TERPHENYL                                 \\
GEL_IN_PHOTOGRAPHIC_EMULSION        & TESTES_ICRP                               \\
GLASS_PYREX                         & TETRACHLOROETHYLENE                       \\
GLASS_LEAD                          & THALLIUM_CHLORIDE                         \\
GLASS_PLATE                         & TISSUE_SOFT_ICRP                          \\
GLUCOSE                             & TISSUE_SOFT_ICRU_FOUR_COMPONENT           \\
GLUTAMINE                           & TISSUE_EQUIVALENT_GAS_METHANE_BASED       \\
GLYCEROL                            & TISSUE_EQUIVALENT_GAS_PROPANE_BASED       \\
GUANINE                             & TITANIUM_DIOXIDE                          \\
GYPSUM_PLASTER_OF_PARIS             & TOLUENE                                   \\
N_HEPTANE                           & TRICHLOROETHYLENE                         \\
N_HEXANE                            & TRIETHYL_PHOSPHATE                        \\
KAPTON_POLYIMIDE_FILM               & TUNGSTEN_HEXAFLUORIDE                     \\
LANTHANUM_OXYBROMIDE                & URANIUM_DICARBIDE                         \\
LANTHANUM_OXYSULFIDE                & URANIUM_MONOCARBIDE                       \\
LEAD_OXIDE                          & URANIUM_OXIDE                             \\
LITHIUM_AMIDE                       & UREA                                      \\
LITHIUM_CARBONATE                   & VALINE                                    \\
LITHIUM_FLUORIDE                    & VITON_FLUOROELASTOMER                     \\
LITHIUM_HYDRIDE                     & WATER_LIQUID, H2O                         \\
LITHIUM_IODIDE                      & WATER_VAPOR                               \\
LITHIUM_OXIDE                       & XYLENE                                    \\
\end{longtable}
\end{center}

%-----------------------------------------------------------------

\begin{figure}[tb]
  \centering
  \includegraphics[width=5in]{surface-curvature.pdf}
  \caption[Surface curvature geometry.]
{Surface curvature geometry. The element reference frame used to describe surface curvature has the
$z$ axis pointing towards the interior of the element, and the $x$ axis in the plane defined by
the entrance and exit reference orbit.}
  \label{f:surface}
\end{figure}

%-----------------------------------------------------------------
\section{X-Ray Reflectivity Tables}
\label{s:reflect}

Reflectivity tables are used to define reflectivity probabilities as functions of incidence angle and
photon energy for \vn{crystal} and \vn{mirror} elements.

The general syntax is:
\begin{example}
  m: mirror, reflectivity_table = \{
    ! A reflection table
    \{
      polarization = <name>
      angles = (<ang1 <ang2 ... <angN>),
      p_reflect = \{
        <energy1> <P_11> <P_12> <P_13> ... <P_1N>,
        <energy2> <P_21> <P_22> <P_23> ... <P_2N>, 
        ...
      \}
    \}
  \}
\end{example}
The \vn{angles} array must be before the \vn{p_reflect} table. Angles are in radians and energy is
in eV. Angle and energy points do not need to be evenly spaced. Probabilities \vn{P_ij} are between
0 and 1. For particles outside the range of angles, the probability is taken to be zero. The
\vn{p_reflect} portion of the reflectivity table must come last.

Possible \vn{polarization} names are:
\begin{example}
  pi        ! Table is for pi mode
  sigma     ! Table is for sigma mode
  both      ! Table is for both polarizations. Default.
\end{example}
An element needs a single table with \vn{polarization} marked as \vn{both} or two tables,
one for \vn{sigma} and the other for \vn{pi}.

For \vn{crystal} elements, angles are always with respect to the Bragg angle for the energy where
the Bragg angle is calculated without any refraction corrections.

%-----------------------------------------------------------------
\section{Surface Properties for X-Ray elements}
\label{s:surface}

The following X-ray elements have a surface which X-rays impinge upon:
\begin{example}
  crystal               \sref{s:crystal}
  detector              \sref{s:detector}
  diffraction_plate     \sref{s:diff.plate}
  mask                  \sref{s:mask}
  mirror, and           \sref{s:mirror}
  multilayer_mirror     \sref{s:multilayer}
  sample                \sref{s:sample}
\end{example}
[There is also the \vn{capillary} element but this element specifies
its surface differently.]

The coordinate system used for characterizing the curvature of a surface is the element reference
frame as shown in \fig{f:surface}). This coordinate system has the $z$ axis pointing towards the
interior of the element, and the $x$ axis in the plane defined by the entrance and exit reference
orbit. In this coordinate system, the surface is an ellipsoid plus a fourth order polynomial in $x$
and $y$ plus a possible ``figure error'' contribution $z_\text{off}$ defined by a surface grid:
\begin{align}
  \label{xs2ij4}
  {-z} &= \frac{1}{g_z} \, \left[ 1 - \sqrt{1 - (g_x \, x)^2 - (g_y \, y)^2} \right] + \\
  &\qquad \qquad \frac{1}{g_{sp}} \, \left[ 1 - \sqrt{1 - (g_{sp} \, x)^2 - (g_{sp} \, y)^2} \right] + 
  \sum_{2 \le i+j \le 6} c_{ij} \, x^i \, y^j - z_\text{off} \nonumber
\end{align}
$z_\text{off}$ is discussed in section~\sref{s:surf.grid} and is only present when the surface grid
\vn{type} is set to \vn{Displacement}. In \Eq{xs2ij4} the $c_{ij}$ coefficients parameterize the
fourth order polynomial, $g_{sp}$ parameterize the spherical curvature, and $g_x, g_y$, and $g_z$
parameterize ellipsoid curvature. [In principle, the spherical curvature is not needed since the
elliptical curvature is more general. In practice, it is sometimes convenient to be able to specify
spherical curvature.]  If $g_z$ is zero, the elliptical curvature is ignored. If $g_z$ or $g_{sp}$
is positive, the curvature is concave towards the incoming photon. If negative, the curvature is
convex. 

The spherical, ellipsoid and polynomial parameters are set for an element by setting the element's
curvature parameter. The syntax is
\begin{example}
  curvature = \{
      spherical    = <Real>,   ! \(g_{sp}\)
      elliptical_x = <Real>,   ! \(g_{x}\)
      elliptical_y = <Real>,   ! \(g_{y}\)
      elliptical_z = <Real>,   ! \(g_{z}\)
      xMyN         = <Real>    ! \(c_{ij}\)
\end{example}

The polynomial coefficients $c_{ij}$ are set via the \vn{xMyN} components where \vn{M} and \vn{N}
are integers in the range 0 through 6 with the restriction
\begin{example}
  2 \(\le\) M + N \(\le\) 6
\end{example}
Example:
\begin{example}
  c2: crystal, curvature = \{spherical = 1/4.7, x2y0 = 0.37, ...\}
\end{example}
in this example, \vn{x2y0} corresponds to the $c_{20}$ term in \Eq{xs2ij4}. To get the
effect of a nonzero $x^0\, y^0$, $x^1 \, y^0$, or $x^0 \, y^1$ terms (since corresponding
\vn{curvature} \vn{xNyM} are not permitted), element offsets and pitches can be used (\sref{s:offset}).

Some useful formulas: Series expansion for a sphere of radius $R$:
\begin{equation}
  {-z} = \frac{x^2}{2 \, R} + \frac{x^4}{8 \, R^3} + \frac{x^6}{16 \, R^5} +
         \frac{y^2}{2 \, R} + \frac{y^4}{8 \, R^3} + \frac{y^6}{16 \, R^5} +
         \frac{x^2 \, y^2}{4 \, R^3} + \frac{3 \, x^4 \, y^2}{16 \, R^5} +
         \frac{3 \, x^2 \, y^4}{16 \, R^5} 
\end{equation}

For a torus with equation
\begin{equation}
  \left( \sqrt{x^2 + (z + r + R)^2} - R \right)^2 + y^2 = r^2
\end{equation}
The series expansion is:
\begin{align}
  {-z} &= \frac{x^2}{2 \, (r+R)} + \frac{x^4}{8 \, (r+R)^3} + \frac{x^6}{16 \, (r+R)^5} + 
         \frac{y^2}{2 \, r} + \frac{y^4}{8 \, r^3} + \\
  & \qquad \frac{y^6}{16 \, r^5} +
         \frac{x^2 \, y^2}{4 \, r \, (r+R)^2} + \frac{3 \, x^4 \, y^2}{16 \, r \, (r + R)^4} + 
         \frac{(3 \, r + R) \, x^2 \, y^4}{16 \, r^3 \, (r + R)^3} \nonumber
\end{align}

For a curved crystal, if $p$ is the distance from the source to the crystal, and $q$ is the distance
from the crystal to the detector, the needed radius of curvature $R_s$ in the sagittal (transverse)
plane to give focusing is \cite{b:del.rio}:
\begin{equation}
  \frac{1}{p} + \frac{1}{q} = \frac{\sin\theta_{g,in} + \sin\theta_{g,out}}{R_s}
\end{equation}
where $\theta_{g,in}$ and $\theta_{g,out}$ are the entrance and exit graze angles. In the tangential
(meridional) plane, the radius $R_t$ needed for focusing is
\begin{equation}
  \frac{\sin^2\theta_{g,in}}{p} + \frac{\sin^2\theta_{g,out}}{q} = \frac{\sin\theta_{g,in} + \sin\theta_{g,out}}{R_t}
\end{equation}
The above formulas assume that the crystal is constructed so that the orientation of the Bragg planes
follows the orientation of the surface. Mirrors have similar formulas with $\theta_{g,in} =
\theta_{g,out} = \theta$.

Example:
\begin{example}
  t_bragg = 1.3950647
  rt = 1  ! Crystal transverse radius
  rs = rt*(sin(t_bragg))^2
  c: crystal, crystal_type =  "Si(553)", b_param = -1, curvature = \{x0y2 = 1 / (2 * rs), 
                      x0y4 = 1 / (8 * rs^3), x2y0 = 1 / (2 * rt), x4y0 = 1 / (8 * rt^3)\}
  c[curvature%x0y2] = 1.01 * c[curvature%x0y2]  ! Put in an error
\end{example}

%-----------------------------------------------------------------
\subsection{Displacement, H_Misalign, and Segmented Surface Grids}
\label{s:surf.grid}
\index{surface grid}

A surface can be broken up into a grid of rectangles. This is useful, for example, in simulating
crystal surface roughness. The case of the \vn{pixel} grid for a \vn{detector} element is discussed
in \Sref{s:detector}. Here the other three types of grids are explained. These are:
\begin{example}
  Displacement       ! Mesh defines an offset from the nominal surface.
  H_Misalign         ! Misalignment of crystal H vector
  Segmented          ! Surface is a matrix of flat rectangles
\end{example}

All grids have the following common parameters:
\begin{example}
  active = <T/F>                     ! Turn on/off effect of grid. Default = True
  ix_bounds = (<ix_min>, <ix_max>),  ! Min/max grid index bounds in x-direction
  iy_bounds = (<iy_min>, <iy_max>),  ! Min/max grid index bounds in y-direction
  r0 = (<x0>, <y0>),                 ! (x,y) coordinates at grid origin
  dr = (<dx>, <dy>),                 ! Spacing between grid points.
\end{example}

Example:
\begin{example}
  ccd: crystal, h_misalign = \{
            r0 = (0.0, 0.01), dr = (0.005, 0.005),
            ix_bounds = (1, 57), iy_bounds = (-30, 10),
            pt(1,-30) = (0.001, -0.002, 0, 0), 
            pt(1,-29) = ..., 
          \}
\end{example}

The grid is a two dimensional rectangular mesh with bounds given by the \vn{ix_bounds} and
\vn{iy_bounds} components. In the above example the grid is 57 pixels in $x$ and 41 pixels in $y$.

The physical placement of the grid on the element is determined by the \vn{r0} and \vn{dr}
components. \vn{r0} is optional and gives the $(x,y)$ coordinates of the center of the pixel with
index $(0,0)$. The \vn{dr} component, which must be present, gives the pixel width and height. Thus
the center of the $(i,j)$ pixel is:
\begin{example}
  (x,y) = (r0(1), r0(2)) + (i*dr(1), j*dr(2))
\end{example}

The different grid types are:
\begin{description}
%
\item[Displacement grid] \Newline
The general syntax for a \vn{displacement} grid is:
\begin{example}
  displacement = \{
      active = <T/F>                     ! Turn on/off effect of grid. Default = True
      ix_bounds = (<ix_min>, <ix_max>),  ! Min/max index bounds in x-direction
      iy_bounds = (<iy_min>, <iy_max>),  ! Min/max index bounds in y-direction
      r0 = (<x0>, <y0>),                 ! (x,y) coordinates at grid origin
      dr = (<dx>, <dy>),                 ! Spacing between grid points.
      pt(<i>,<j>) = (<z>),                                  ! or
      pt(<i>,<j>) = (<z>, <dz/dx>, <dz/dy>),                ! or
      pt(<i>,<j>) = (<z>, <dz/dx>, <dz/dy>, <d2z/dxdy>),
          \}
\end{example}

With a \vn{displacement} grid, an offset $z_\text{off}$ is added to the surface
position defined in \Eq{xs2ij4}. The $z$ offset at a given $(x,y)$ position is determined by a
spline interpolation of the $z$ values of the grid of points defined by \vn{pt(<i>,<j>)}. For each
point \vn{pt(<i>,<j>)}, the $z$ value along with the $dz/dx$, $dz/dy$, $d2z/dxdy$ slopes can be specified or
only the $z$ value needs to be specified. With the later option the slopes will be computed by
taking finite differences of nearest neighbors.
%
\item[H_Misalign grid] \Newline
The general syntax for a \vn{h_misalign} grid:
\begin{example}
  h_misalign = \{
      active = <T/F>                     ! Turn on/off effect of grid. Default = True
      ix_bounds = (<ix_min>, <ix_max>),  ! Min/max index bounds in x-direction
      iy_bounds = (<iy_min>, <iy_max>),  ! Min/max index bounds in y-direction
      r0 = (<x0>, <y0>),                 ! (x,y) coordinates at grid origin
      dr = (<dx>, <dy>)                  ! Spacing between grid points.
      pt(<i>,<j>) = (<rot_y, <rot_x>, <rot_y_rms>, <rot_x_rms>)  ! For Bragg diffraction
      pt(<i>,<j>) = (<rot_y, <rot_z>, <rot_y_rms>, <rot_z_rms>)  ! For Laue diffraction
          \}
\end{example}

An \vn{h_misalign} grid is used with crystals only. With \vn{H_Misalign}, the grid defines
misalignment of the $\bfH$ vector which is the normal to the diffracting planes of the crystal
(\sref{s:crystal.tracking}).  When using \vn{H_Misalign}, each \vn{pt(i,j)} component gives the
angular misalignment of $\bfH$ for the region around the point. for Bragg diffraction where $\bfH$
is oriented approximately along the $-z$-axis, the
misalignment of $\bfH$ is characterized by rotations about the $y$-axis and $x$-axis
\begin{example}
  rot_y_tot = <rot_y> + r2 * <rot_y_rms>
  rot_x_tot = <rot_x> + r1 * <rot_x_rms>
\end{example}
where \vn{rot_x_tot} and \vn{rot_y_tot} are the rotational misalignment (in radians) of $\bfH$
around the $y$-axis and $x$-axis respectively. The small angle approximation is used, That is, it is
assumed that both \vn{rot_x_tot} and \vn{rot_y_tot} are small compared to one. 
the quantities in brackets \vn{<...>} are components of \vn{pt}, and \vn{r1} and
\vn{r2} are Gaussian distributed random numbers with unit rms. These random numbers are regenerated
for each photon.

For Laure diffraction, since the $\bfH$ vector is approximately aligned with the $-x$-axis, the
misalignment is characterized by rotations about the $y$-axis and $z$-axis. Notice that for both
Bragg and Laue diffraction, the rotation around the $y$-axis misaligns (approximately) $\bfH$ in the
plane of the diffraction.
%
\item[Segmented grid] \Newline
The general syntax for a \vn{segmented} grid:
\begin{example}
  segmented = \{
      active = <T/F>                     ! Turn on/off effect of grid. Default = True
      ix_bounds = (<ix_min>, <ix_max>),  ! Min/max index bounds in x-direction
      iy_bounds = (<iy_min>, <iy_max>),  ! Min/max index bounds in y-direction
      r0 = (<x0>, <y0>),                 ! (x,y) coordinates at grid origin
      dr = (<dx>, <dy>)                  ! Spacing between grid points.
          \}
\end{example}

With a \vn{segmented} grid the crystal surface is modeled as a grid of flat ``rectangles'' (the
actual shape is very close but not quite rectangular). Using a segmented surface only makes sense
when the surface is curved (see \Eq{xs2ij4}). There is one rectangle for each grid point. Each rectangle
has an extent in the $(x,y)$ transverse dimensions equal to the spacing between grid points.
\Eq{xs2ij4} is used to calculate the $z$ coordinate of the vertices of a given rectangle and
then these $z$ values are adjusted so that
\begin{example}
  1) The rectangle is flat (the vertices all lie on a plane).
  2) The rectangle contacts the unsegmented surface (\Eq{xs2ij4}) at two diagonally 
       opposite vertices.
  3) The other two diagonally opposite vertices will be as close as possible in the 
       least squares sense from the unsegmented surface.
\end{example}
Note: The \vn{pt} component is not used here.
\end{description}

%-----------------------------------------------------------------
\section{Walls: Vacuum Chamber, Capillary and Mask}
\label{s:wall}
\index{wall}

The \vn{wall} attribute for an element is used to define:
\begin{example}
  vacuum chamber wall
  capillary element (\sref{s:capillary}) inside wall
  diffraction_plate (\sref{s:diff.plate}) geometry
\end{example}

The topics of the following subsections are:
\begin{example}
  \sref{s:wall.syntax}      General wall syntax.
  \sref{s:wall.section}      Cross-section construction. 
  \sref{s:wall.interpolation}      Capillary and vacuum chamber wall interpolation.
  \sref{s:wall.capillary}      Capillary wall.
  \sref{s:wall.vacuum}      Vacuum chamber wall.
  \sref{s:masking.wall}      Diffraction_plate and mask element geometries.
\end{example}

%-----------------------------------------------------------------
\subsection{Wall Syntax}
\label{s:wall.syntax}

The syntax of the \vn{wall} attribute is:
\begin{example}
  wall = \{
    superimpose = <T/F>,               ! Chamber wall only
    thickness = <real>                 ! Default thickness. 
    opaque_material = <material_type>  ! Default opaque material. 
    clear_material = <material_type>   ! Default clear material. 
    section = \{ 
      type = <section_type>,           ! Chamber Mask, and Diffraction_plate only
      s = <longitudinal_position>,     ! Relative to beginning of element.
                                       !   Not used for mask or diffraction_plate.
      r0 = (<x0>, <y0>),               ! section (x,y) origin. Default = (0, 0).
      absolute_vertices = <T/F>,       ! Vertex relative to r0? Default = F.
      material = <material_type>,      ! Mask and Diffraction_plate only.
      thickness = <real>,              ! Mask and Diffraction_plate only.
      dr_ds = <value>,                 ! Capillary and Chamber only
      v(1) = \{<x>, <y>, <radius_x>, <radius_y>, <tilt>\}, 
      v(2) = \{ ... \},
      ...\},
    section = \{
      s = <longitudinal_position>, 
      v(1) = \{... \},
      ... \},
    ... \}
\end{example}
A \vn{wall} begins with ``\vn{wall = \{}'' and ends with a ``\vn{\}}''. In between are a number of
individual cross-section structures. Each individual cross-section begins with ``\vn{section = \{}''
and ends with a ``\vn{\}}''. The \vn{s} parameter of a cross-section gives the longitudinal position
of the cross-section.  Example:
\begin{example}
  this_cap: capillary, 
    wall = \{   
      section = \{ ! cross-section with top/bottom symmetry
        s = 0, v(1) =  \{0.02, 0.00\}, 
        v(2) = \{0.00, 0.02, 0.02\}, v(3) = \{-0.01, 0.01\} \}, 
      section = \{  ! Cross-section that is a tilted ellipse.
        s = 0.34, 
        v(1) = \{0.003, -0.001, 0.015, 0.008, 0.2*pi\} \} \}
\end{example}
In this example an element called \vn{this_cap} is a \vn{capillary}
whose wall is defined by two cross-sections.

%------------------

\begin{figure}[tb]
  \centering
  \includegraphics[width=6in]{chamber-wall.pdf}
  \caption[Capillary or vacuum chamber wall.]
{A) The inside wall of a capillary or the vacuum chamber wall of a non-capillary element is defined
by a number of cross-sectional slices.  B) Each cross-section is made up of a number of
vertices. The segments between the vertices can be either a line segment, the arc of a circle, or a
section of an ellipse.}
  \label{f:chamber.wall}
\end{figure}

%-----------------------------------------------------------------
\subsection{Wall Sections}
\label{s:wall.section}

The wall is defined by a number of cross-sectional slices. For \fig{f:chamber.wall}A shows the
geometry for \vn{capillary} or vacuum chamber walls.  Each cross-section is defined by a
longitudinal position $s$ relative to the beginning of the element and a number of vertices. The arc
between each vertex may be either a straight line, an arc of a circle, or a section of an
ellipse. For a capillary it is mandatory that a cross-section be convex. That is, given any two
points within the cross-section, all points on the line segment connecting them must be within the
cross-section.

The \vn{v(<j>)} within a cross-section define the vertices for each cross-section. The vertices are
defined with respect to the section origin given by \vn{r0}. Each \vn{v(<j>)} has five
parameters. It is mandatory to specify the first two parameters \vn{<x>} and \vn{<y>}. Specifying
the rest, \vn{<radius_x>}, \vn{<radius_y>}, and \vn{<tilt>}, is optional. The default values, if not
specified, is zero. The point (\vn{<x>}, \vn{<y>}) defines the position of the vertex. The
parameters \vn{<radius_x>}, \vn{<radius_y>}, and \vn{<tilt>} define the shape of the segment of the
cross-section between the given vertex and the preceding one.
\begin{example}
  <radius_x>  = 0, <radius_y>  = 0   --> Straight line segment.
  <radius_x> != 0, <radius_y>  = 0   --> Circular arc with radius = radius_x
  <radius_x>  = 0, <radius_y> != 0   --> Illegal!
  <radius_x> != 0, <radius_y> != 0   --> Ellipse section.
\end{example}
When an ellipse is specified, \vn{<radius_x>}, and \vn{<radius_y>} are
the half width and half height of the semi-major axes and the
\vn{<tilt>} parameter gives the tilt of the ellipse. \vn{<radius_x>}
and \vn{<radius_y>} must not be negative.

In the example above, for the first cross-section, \vn{v(2)} specifies a non-zero \vn{<radius_x>}
and, by default, \vn{<radius_y>} is zero. Thus the segment of the cross-section between \vn{v(1)}
and \vn{v(2)} is circular in nature with a radius of 0.02. Since \vn{v(3)} does not specify
\vn{<radius_x>} nor \vn{<radius_y>}, the cross-section between \vn{v(2)} and \vn{v(3)} is a straight
line segment.

The vertex points must be arranged in a ``counter clockwise manner''.  For vertices \vn{<v(i)>} and
\vn{<v(i+1)>} connected by a line segment this translates to
\begin{equation}
  0 < \theta_{i+1} - \theta_{i} \pmod{2\pi} < \pi
  \label{0tt2p}
\end{equation}
where $(r_n, \theta_n)$ are the polar coordinates of the $n^{th}$ vertex with respect to the point
\vn{r0}. For vertices connected by an arc, ``counter clockwise manner'' means that the line segment
with one end at the center of the arc and the other end traversing the arc from \vn{<v(i)>} to
\vn{<v(i+1)>} rotates in counter clockwise as shown in \fig{f:chamber.wall}B.

The red line segment with one end at the center of the arc and the other end traversing the arc
from, in this case, $V(2)$ to $V(3)$, rotates in counter clockwise manner. In general, there are two
solutions for constructing such an arc. For positive radii, the solution chosen is the one whose
center is closest to the section origin $(x_0, y_0)$. If the radii are negative, the center point
will be the point farthest from the origin (the dashed line between $V(2)$ and $V(3)$ in the
figure).

A restriction on cross-sections is that the section origin $(x_0, y_0)$ must be in the interior of
any cross-section and that for any cross-section a line drawn from the origin at any given angle
$\theta$ will intersect the cross-section at exactly one point as shown in
\fig{f:chamber.wall}B. This is an important point in the construction of the wall between
cross-sections as explained below.

The last vertex specified, call it \vn{<v(n)>}, should not have the same \vn{<x>}, \vn{<y>} values
as the first vertex \vn{<v(1)>}. That is, there will be a segment of the cross-section connecting
\vn{<v(n)>} to \vn{<v(1)>}. The geometry of this segment is determined by the parameters of
\vn{<v(1)>}.

If there is mirror symmetry about the $x$ or $y$ axis for a cross-section, the ``mirrored''
vertices, on the ``negative'' side of the mirror plane, do not have to be specified. Thus if all the
vertex points of a cross-section are in the first quadrant, that is, all \vn{<x>} and \vn{<y>} are
zero or positive, mirror symmetry about both the $x$ and $y$ axes is assumed. If all the \vn{<y>}
values are zero or positive and some \vn{<x>} values are positive and some are negative, mirror
symmetry about the $x$ axis is assumed. Finally, if all the \vn{<x>} values are zero or positive but
some \vn{<y>} values are positive and some are negative, symmetry about the $y$ axis is assumed. For
example, for the first in the above example, since all the \vn{<y>} values are non-negative and
there are positive and negative \vn{<x>} values, symmetry about the $x$ axis is assumed.

The one exception to the above rule that (\vn{<x>}, \vn{<y>}) is the vertex center is when a single
vertex \vn{v(1)} is specified for a cross-section with a non-zero \vn{<radius_x>}. In this case,
(\vn{<x>}, \vn{<y>}) are taken to be the center of the circle or ellipse. For example, if a single
vertex is specified for a cross-section as:
\begin{example}
  section = \{s = 0.3, v(1) = \{0.03, -0.01, 0.15, 0.08, 0.2\}\}
\end{example}
the cross-section will be an ellipse with center at $(0.03, -0.01)$ with a tilt of $0.2$ and axes
radii of $0.15$ and $0.08$. If a cross-section has a single vertex and \vn{<radius_x>} is not
specified, the cross-section is a rectangle. For example
\begin{example}
  section = \{s = 0.3, v(1) = \{0.03, 0.01\}\}
\end{example}

The vertices are defined with respect to the local sector origin $r_0$ except if
\vn{absolute_vertices} is set to \vn{True} in which case the vertex numbers are taken as absolute.
For example, the following two cross-sections are identical and describe a rectangle with edges at
$x = 1$ and 5 and $y = -6$ and 6
\begin{example}
  section = \{absolute_vertices = T, r0 = (4, 0), 
                  v(1) = \{5, 6\}, v(2) = \{1, 6\}, v(3) = \{1, -6\}, v(4) = \{5, -6\}\}
  section = \{r0 = (4, 0), v(1) = \{2, 6\}\}
\end{example}
Notice that while $r0$ is not needed in the first section for positioning of the vertices, it is
needed in the first section to make \Eq{0tt2p} true.

%-----------------------------------------------------------------
\subsection{Interpolation Between Sections}
\label{s:wall.interpolation}

\begin{figure}[tb]
  \centering
  \includegraphics[width=4in]{concave-capillary.pdf}
  \caption[Convex cross-sections do not guarantee a convex volume.]
{Example where convex cross-sections do not produce a convex volume.  Cross-sections (A) and (C) are
ellipses with a 5 to 1 aspect ratio.  Half way in between, linear interpolation produces a convex
cross-section as shown in (B).}
  \label{f:concave.capillary}
\end{figure}

For \vn{capillary} and vacuum chamber walls, the wall between cross-sections, is defined by
interpolation. At a given $s$ position, the $r, \theta$ coordinate system in the transverse $x, y$
plane is defined with respect to an origin $\bfr_O(s)$ given by a linear interpolation of the
origins of the cross-sections to either side of the given $s$ position. Let $s_1$ denote the
position of the cross-section just before $s$ and $s_2$ denote the position of the cross-section
just after $s$. Let $\bfr_{01}$ be the $(x_0, y_0)$ origin defined for the cross section at $s_1$
and $\bfr_{02}$ be the $(x_0, y_0)$ origin defined for the cross section at $s_2$. Then
\begin{equation}
  \bfr_O(s) = (1 - \stilde) \, \bfr_{01} + \stilde \, \bfr_{02}
\end{equation}
where 
\begin{equation}
  \stilde \equiv \frac{s - s_1}{s_2 - s_1}
\end{equation}

Let $r_{c1}(\theta)$ and $r_{c2}(\theta)$ be the radius of the wall as a function of $\theta$ for
the cross-sections at $s = s_1$ and $s = s_2$ respectively. The wall $r_c(\theta, s)$ at any point
$s$ between $s_1$ and $s_2$ is then defined by the equation
\begin{equation}
  r_c(\theta, s) = p_1(\stilde) \, r_{c1}(\theta) + p_2(\stilde) \, r_{c2}(\theta)
\end{equation}
where $p_1$ and $p_2$ are cubic polynomials parameterized by
\begin{align}
  p_1 &= 1 - \stilde + a_1 \, \stilde + a_2 \, \stilde^2 + a_3 \, \stilde^3 \CRNO
  p_2 &= \stilde + b_1 \, \stilde + b_2 \, \stilde^2 + b_3 \, \stilde^3 
\end{align}
If $a_i = b_i = 0$ for all $i = 1, 2, 3$, the interpolation is linear and this is the default if
either of the parameters \vn{dr_ds1} and \vn{dr_ds2} are not given in the wall definition. These
parameters are the slopes of the wall with respect to $s$ at the end points
\begin{equation}
  \text{dr_ds1} \equiv \left. \frac{d\overline{r}}{ds} \right|_{s = s_1} \comma \qquad
  \text{dr_ds2} \equiv \left. \frac{d\overline{r}}{ds} \right|_{s = s_2} 
\end{equation}
where $\overline{r}$ is the average $r$ averaged over all $\theta$. When {\em both} \vn{dr_ds1} and
\vn{dr_ds2} are specified, the $a_i$ and $b_i$ are calculated so that the slopes of the wall match
the values of \vn{dr_ds1} and \vn{dr_ds2} along with the constraints.
\begin{align}
  p_1(0) &= 1 \comma \qquad p_1(1) = 0 \CRNO
  p_2(0) &= 0 \comma \qquad p_2(1) = 1 \\
  M &\equiv a_1^2 + a_2^2 + a_3^2 + b_1^2 + b_2^2 + b_3^2 \text{ is a minimum}
  \nonumber
\end{align}
The last constraint ensures a ``smooth'' transition between the two cross-sections.

To refer to a cross-section parameters after an element has been defined, the following syntax is
used:
\begin{example}
  ele_name[wall%section(n)%v(j)%x]   ! x value of j^th vertex of n^th cross-section
\end{example}

%-----------------------------------------------------------------
\subsection{Capillary Wall}
\label{s:wall.capillary}
\index{capillary!wall}

For a \vn{capillary}, \vn{s} must be zero for the first cross-section and the length of the
capillary is given by the value of \vn{s} of the last cross-section.

For a \vn{capillary}, in order for \bmad to quickly track photons, \bmad assumes that the volume
between the cross-sections is convex. The volume will be convex if each cross-section $r_c(\theta,
s)$ at any given $s$ is convex. Note that it is {\em not} sufficient for $r_c(\theta, s)$ to be
convex at the specified cross-sections as shown in \fig{f:concave.capillary}. Also note that it is
perfectly fine for the total capillary volume to not be convex.

%-----------------------------------------------------------------
\subsection{Vacuum Chamber Wall}
\label{s:wall.vacuum}

The vacuum chamber wall is independent of the element apertures (\sref{s:limit}). Unless a program
is specifically constructed, the presence of a vacuum chamber wall will not affect particle
tracking.

The vacuum chamber wall defined for an element may be shorter or longer than the element.  The
vacuum chamber wall for a particular lattice branch is the sum of all the chamber walls of the
individual elements. That is, the chamber wall at any given point is determined by interpolation of
the nearest sections upstream and downstream to the point.  Thus a given lattice element need not
contain a \vn{wall} component for the chamber wall to be well defined at the element.

The exception to the above rule is when a \vn{section} has its
\vn{type} component set to either:
\begin{example}
  wall_start
  wall_end
\end{example}
\vn{wall_start} and \vn{wall_end} sections must come in pairs. The next section after a
\vn{wall_end} section (if this section is not the last section in the lattice) must be a
\vn{wall_start} section.  If a section has a \vn{type} of \vn{wall_start}, the region between that
section and the previous section (which must be a \vn{wall_end} section) will be considered to have
no wall. If the \vn{wall_start} section is the first section of the lattice branch, the region of no
wall will start at the beginning of the branch. Similarly, if a section has a \vn{type} of
\vn{wall_end}, the region between that section and the next section (or the end of the lattice
branch if there is no next section) will not have a wall.

The chamber walls of any two elements may not overlap. The exception is when the \vn{superimpose}
attribute for a wall of an element is set to True. In this case, any other wall cross-sections from
any other elements that overlap the superimposed wall are discarded.  Superposition of a wall is
useful, for example, in introducing mask regions into the wall.

If a branch has a closed geometry (\sref{s:param}), wall sections that extend beyond the ends of the
branch are ``wrapped'' around.

If a particle is past the last wall cross-section or before the first wall cross-section, The
following rules are used: If the branch has a \vn{closed geometry}, the wall will be interpolated
between the last and first cross-sections. If the branch has an \vn{open} geometry, the wall is
taken to have a constant cross-section in these regions.

The chamber wall is defined with respect to the local coordinate system (\sref{s:ref}). That is, in
a bend a wall that has a constant cross section is a section of a torus.

\index{patch!and chamber wall}
\vn{Patch} elements (\sref{s:patch}) complicate the wall geometry since the coordinate system at the
end of the \vn{patch} may be arbitrarily located relative to the beginning of the patch. To avoid
confusion as to what coordinate system a wall section belongs to, \vn{patch} elements are not
allowed to define a wall. The wall through a patch is determined by the closest wall sections of
neighboring elements.

%-------------------

\begin{figure}[tb]
  \centering
  \includegraphics[width=6in]{crotch.pdf}
  \caption[vacuum chamber crotch geometry.]
{A) Crotch geometry: Two pipes labeled ``leg1'' and ``leg2'' merge into a single pipe called the
``trunk'' pipe. Five wall sections are used to define the crotch geometry (solid lines). Dashed
lines represent sections not involved in defining the crotch. For purposes of illustration, the
three trunk sections are displaced longitudinally but in reality must have the same longitudinal
coordinate.  B) Example layout of the trunk1, trunk2 and trunk wall sections. $O_1$, $O_2$ and $O$
are the $x_0, y_0$ origins of the sections.}
  \label{f:crotch}
\end{figure}

%-------------------

\index{crotch chamber geometry}
Each section has a \vn{type} attribute. This attribute is not used for \vn{capillary} elements. For
a vacuum chamber wall, the \vn{type} attribute is used to describe a ``crotch'' geometry where two
pipes merge into one pipe. The possible values for the \vn{type} attribute are:
\begin{example}
  normal     ! default
  leg1
  leg2
  trunk1
  trunk2
  trunk
\end{example}
The geometry of a crotch is shown in \fig{f:crotch}A. Two pipes, called ``leg1'' and ``leg2'', merge
into one pipe called the ``trunk'' pipe.  The trunk pipe can be either upstream or downstream of the
leg pipes.  To describe this situation, five sections are needed: One section in each leg pipe which
need to have their \vn{type} attribute set to \vn{leg1} and \vn{leg2}, and three sections in the
trunk with one having a a \vn{type} attribute of \vn{trunk1}, another having a \vn{type} attribute
of \vn{trunk2} and the third having a \vn{type} attribute of \vn{trunk}. There can be no sections
between the leg sections and the trunk sections.

All three trunk sections must be associated with the same element and have the same \vn{s} value. In
the list of sections of the element containing the trunk elements, the \vn{trunk1} and \vn{trunk2}
sections must be listed first if the leg pipes are upstream of the trunk pipe (the situation shown
in the figure) and must be listed last if the leg pipes are downstream. That is, the \vn{trunk1} and
\vn{trunk2} sections are ``between'' the leg sections and the \vn{trunk} section. It does not matter
if \vn{trunk1} is before or after \vn{trunk2}.

The \vn{trunk1} and \vn{trunk2} sections must not overlap and the \vn{trunk} section must be
constructed so that its area is the union of the areas of \vn{trunk1} and \vn{trunk2}. An example is
illustrated in \fig{f:crotch}B. Here the \vn{trunk1} and \vn{trunk2} sections are squares with
origins labeled $O_1$ and $O_2$ in the figure. By necessity, these origins must be different since
each must lie within the boundaries of their respective areas. The \vn{trunk} section is a rectangle
encompassing the two squares and has an origin labeled $O$.

Between \vn{leg1} and \vn{trunk1} sections the wall is interpolated using these two
section. Similarly for the region between \vn{leg2} and \vn{trunk2} sections. Away from these
regions interpolation is done as outlined in \sref{s:wall.interpolation}. However, these two regions
need a different interpolation scheme since, \vn{leg1} and \vn{trunk1}, as well as \vn{leg2} and
\vn{trunk2} sections do not have to be parallel to each other.

%-----------------------------------------------------------------
\subsection{Mask Wall For Diffraction Plate and Mask Elements}
\label{s:masking.wall}

\begin{figure}[tb]
  \centering
  \includegraphics[width=6in]{diffraction-plate.pdf}
  \caption[Example mask wall]{
A) The surface of a \vn{diffraction_plate} or \vn{mask} element is divided into ``clear'' (white)
and ``opaque'' (black) areas. As explained in the text, these areas are defined by five
\vn{sections} labeled $s_1$ through $s_5$. B) All wall sections must be star shaped with respect to
the section's origin. In this example, The section is {\em not} star shaped since a line drawn from
the origin point $o$ to the point $p$ on the boundary intersects the boundary twice in between. In
this case the section can be made star shaped by moving the origin to $o'$.
  }
  \label{f:diff.plate}
\end{figure}

The \vn{wall} attribute (\sref{s:wall}) of a \vn{diffraction_plate} or \vn{mask} element specifies
what areas of the element will transmit or reflect particle and what areas will not. Here the
``wall'' defines a 2-dimensional area where particles (or X-rays) impinge upon and not a
3-dimensional surface.  As such, the \vn{s} longitudinal position parameter of wall \vn{sections} is
not used with \vn{mask} and \vn{diffraction_plate} elements.

The algorithm used to decide if a particle hitting a given point will be transmitted or not is as
follows.  A \vn{wall} is comprised a a ordered list of \vn{sections} as discussed in
\sref{s:wall.section}.  Sections will be labeled $s_1, s_2, s_3, \ldots$ with $s_i$ being the $i$\Th
section defined in the \vn{wall} structure. Each section of the wall must have its \vn{type}
attribute set to one of:
\begin{example}
  clear
  opaque
\end{example}
A section is called ``clear'' or ``opaque'' depending upon the setting of its \vn{type} attribute.
The first section $s_1$ must be labeled \vn{clear}. Each \vn{clear} section has zero or more
associated \vn{opaque} sections that follow the \vn{clear} section. An opaque section $s_j$ is
associated with clear section $s_i$ if and only if $i$ is less than $j$, and all the sections $k$
between $i$ and $j$ (that is, $i < k < j$) are opaque sections. For example, if the \vn{wall}
defines 6 sections:
\begin{example}
  \(s_1\)   Clear
  \(s_2\)   Clear
  \(s_3\)   Opaque
  \(s_4\)   Opaque
  \(s_5\)   Clear
  \(s_6\)   Opaque
\end{example}
with this arrangement, clear section $s_1$ does not have any associated opaque sections, clear section
$s_2$ has two associated opaque section $s_3$ and $s_4$, and clear section $s_5$ has one associated
opaque section $s_6$.

Each \vn{section} covers an area specified by the vertex list associated with the \vn{section}. To
decide if a particle is transmitted, each clear section, starting from $s_1$, is tested to see if
the particle's $(x,y)$ coordinates is within the section. If the particle position is not within any
clear section, the particle is considered to have hit an opaque region. If a particle is within one
or more clear regions, let $s_i$ be the clear region with the smallest index $i$ that the particle
is within. The particle is transmitted if the particle is outside of all associated opaque region of
$s_i$.

When tracking photons, any clear section can be given a \vn{material} and \vn{thickness}. Available
materials are listed in \sref{s:cryst.list}. A photon transversing a clear area with a defined
material will be attenuated and have a phase shift. Note that \vn{material} and \vn{thickness}
properties are not to be assigned to \vn{opaque} sections.

To enable \bmad to quickly calculate whether a particle has landed on a clear or opaque section, All
sections, both clear and opaque, must be ``star shaped'' with respect to the $(x_0, y_0)$ origin
used by the section. That is, a line drawn from the section origin to any point on the section
boundary must not pass through any boundary points of the section in between. This is illustrated in
\fig{f:diff.plate}B where the section is not star shaped since a line drawn from the origin $o$ to
the point $p$ on the boundary passes through two boundary points in between. In this case the
section can trivially be made star shaped by moving the origin to point $o'$. If it is not possible
to make a section star shaped by moving the origin, the section must be divided into multiple
sections.

An example geometry is shown in \fig{f:diff.plate}A. A wall that constructs this geometry is:
\begin{example}
  z_plate: diffraction_plate, wall = \{
    section = \{          ! \(s_1\) 
      type = clear,
      v(1) = \{0, 0, 0.03, 0.013\},
    section = \{          ! \(s_2\)
      type = opaque,
      v(1) = \{0, 0, 0.005\},
    section = \{          ! \(s_3\)
      type = opaque,
      r0 = (0.02, 0.00),
      v(1) = \{0, 0, 0.005\} \}
    section = \{          ! \(s_4\)
      type = clear, 
      v(1) = \{0.04, 0\}, v(2) = \{0.04, 0.022\},
      v(3) = \{0, 0.03\},
    section = \{          ! \(s_5\)
      type = opaque,
      v(1) = \{0.032, 0.016\},
\end{example}
There are two clear sections $s_1$ and $s_4$. Clear section $s_1$ has an oval shape and has two
associated circular opaque sections $s_2$ and $s_3$. Clear section $s_4$ has a hexagonal shape and
has one associated rectangular opaque section $s_5$. Notice that because opaque section $s_5$ is
associated with clear section $s_4$, and since clear section $s_4$ comes after clear section $s_1$,
opaque section $s_5$ does not affect clear section $s_1$ even though $s_5$ (completely) overlaps
$s_1$. This shows that the ordering of the clear sections is important but the ordering of the
opaque sections associated with a given clear section is not important.

%-----------------------------------------------------------------
\section{Length Attributes}
\label{s:l}

\index{length of elements}
\index{l|hyperbf}
\index{l_chord|hyperbf}
\index{rbend}
\index{sbend}
The length attributes are
\begin{example}
  l           = <Real>  ! 
  l_chord     = <Real>  ! Chord length of a bend. Dependent attribute.
  l_rectangle = <Real>  ! Rectangular length. See \sref{s:bend}
\end{example}
The length \vn{l} is the path length of the reference particle. The one exception is for an
\vn{rbend}, the length \vn{l} set in the lattice file is the chord length
(\sref{s:bend}). internally, \bmad converts all \vn{rbend}s to \vn{sbend}s and stores the chord
length under the \vn{l_chord} attribute.  Example:
\begin{example}
  b: rbend, l = 0.6   ! For rbends, l will be converted to l_chord
\end{example}

\index{girder}
For a \vn{girder} element the length \vn{l} is a dependent attribute and is set by \bmad to be the
difference in longitudinal position $s$ of the downstream end of the last element supported relative
to the upstream end of the first element.

\index{wiggler}
For \vn{wiggler}s, the length \vn{l} is not the same as the path length for a particle with the
reference energy starting on the reference orbit. See~\sref{s:ref}.

\index{patch}
For \vn{patch} elements the \vn{l} length is, by definition, equal to \vn{z_offset}. For \vn{patch}
elements, \vn{l} is a dependent attribute and will be automatically set to \vn{z_offset} by \bmad.

\index{capillary}
The length of a \vn{capillary} element is a dependent variable and is given by the value of \vn{s}
of the last wall cross-section (\sref{s:wall.capillary}).

\index{crystal}
The length of a crystal is zero for Bragg diffraction and is a dependent attribute dependent upon
the crystal thickness for Laue diffraction. See \sref{s:crystal} for more details.

%-----------------------------------------------------------------
\section{Is_on Attribute}
\label{s:is.on}
\index{is_on|hyperbf}

The \vn{is_on} attribute
\begin{example}
  is_on = <Logical>
\end{example}
is used to turn an element off. Turning
an element off essentially converts it into a drift.
Example
\begin{example}
  q1: quad, l = 0.6, k1 = 0.95
  q1[is_on] = False
\end{example}

\index{aperture}
\index{reference orbit}
\index{reference energy}
\vn{is_on} does not affect any apertures that are set. Additionally, \vn{is_on} does not affect the
reference orbit (\sref{s:ref}) or reference energy (\sref{s:ref.energy}). For example, turning off an
\vn{lcavity} will not affect the reference energy.

The following elements cannot be ``turned off:''
\begin{example}
  beginning_ele       null_ele
  capillary           overlay
  crystal             hybrid
  drift               mirror
  fiducial            multilayer_mirror
  floor_shift         photon_init
  patch               sample
  group
\end{example}

A related parameter is \vn{multipoles_on} (\sref{s:multip}).

%-----------------------------------------------------------------
\section{Multipole Attributes: Magnetic and Electric}
\label{s:multip}

Multipole formulas for are given in \sref{s:mag.field} and \sref{s:elec.field}. Note that the
setting of \vn{field_master} (\sref{s:field.master}) will determine if multipoles are interpreted as
normalized or unnormalized.

\index{multipole!knl, tn|hyperbf} 
\index{multipole}
A \vn{multipole} (\sref{s:mult}) element specifies its magnetic multipole components using normal
and skew components with a tilt
\begin{example}
  KnL  = <Real>  ! Normal component. n = Integer. 
  KnSL = <Real>  ! Skew component. n = Integer. 
  Tn   = <Real>  ! Tilt. n = Integer. Default is $pi$/(2n + 2)
\end{example}
Where \vn{n} is an integer in the range from 0 (dipole component) through 21.  If \vn{Tn} is given
without a value, a default of $pi$/(2n + 2) will be used producing a skew field. Example:
\begin{example}
  m: multipole, k1l = 0.32, t1  ! Skew quadrupole of strength 0.32
\end{example}
Following \vn{MAD}, a non-zero dipole (\vn{K0L} component will affect the reference orbit (just like
a normal dipole will). This is not true for any other element.

\index{ab_multipole}
\index{multipole!an, bn|hyperbf} 
An \vn{ab_multipole} (\sref{s:ab.m}) specifies magnetic multipoles
using normal (\vn{Bn}) and skew (\vn{An}) components:
\begin{example}
  An = <Real>
  Bn = <Real>
\end{example}
Here \vn{n} ranges from 0 (dipole component) through 21. Example:
\begin{example}
  q1: ab_multipole, b2 = 0.12, a20 = 1e7, field_master = T
\end{example}
Note that there is a factor of $n$-factorial difference between \vn{An}, \vn{Bn} and \vn{KnL},
\vn{KnSL} multipoles \sref{s:mag.field}.

\index{r0_mag}\index{scale_multipoles}
Elements like \vn{quadrupoles} and \vn{sextupoles} can have assigned to them both magnetic and
electric multipole fields. In this case, the magnetic fields are specified using the same convention
as the \vn{ab_multipole}.  For such non-multipole elements, the magnetic multipole strength is
scaled via (\Eq{ababf}) 
\begin{equation}
a_n(\text{actual}) = F \, \frac{r_0^{n_\REF}}{r_0^n} \, a_n(\text{input}), \qquad
b_n(\text{actual}) = F \, \frac{r_0^{n_\REF}}{r_0^n} \, b_n(\text{input})
\end{equation}
where ``\vn{input}'' denotes the input value set in the lattice file, ``\vn{actual}'' denotes the
value that is used to compute the field, $F$ is the strength of the element (for example $F$ is $K1 \cdot
L$ for a quadrupole), and $r_0$ is the ``measurement radius'' and is set by the \vn{r0_mag}
attribute. The default value of $r_0$ is 0 in which case the factor of $r_0^{n_\REF}/r_0^n$ is
omitted. The scaling may be turned off altogether by setting the \vn{scale_multipoles} attribute.
Example:
\begin{example}
  q1: quadrupole, b2 = 0.12, a20 = 1e7, scale_multipoles = F
\end{example}

Electric multipoles are specified using normal (\vn{Bn_elec}) and skew
(\vn{An_elec}) components. \begin{example}
  An_elec = <Real>
  Bn_elec = <Real>
\end{example}
Here \vn{n} ranges from 0 (dipole component) through 21. Like the magnetic multipoles, a measurement
radius \vn{r0_elec} can be used to scale the multipoles as explained in \sref{s:elec.field}.
Example:
\begin{example}
  q1: quadrupole, l = 1.2, b2_elec = 1e6, r0_elec = 0.034
\end{example}
See \sref{s:elec.field} for how electric multipoles are defined. Notice that Electric multipoles are
never scaled by the element's field strength as they are with magnetic multipoles. If the value of
\vn{r0_elec} is zero (the default) the multipoles will not be scaled.

Unlike magnetic multipoles, there are no factors of the reference momentum nor the element length in
the definition for electric multipoles. That is, electric multipole values represent the field and
not the normalized integrated field. Thus an electric multipole associated with a zero length
element will have no effect on tracking. This being the case, \bmad does not allow electric
multipole values to be specified for \vn{multipole} and \vn{ab_multipole} elements. Indeed, in the
limit of zero element length at constant integrated electric field strength, the equations of motion
are singular since, unlike the magnetic case, the infinite fringe fields give rise to infinite
energy shifts.

\index{multipoles_on}
The magnetic and electric multipole kick can be toggled on or off using the
\vn{multipoles_on} attribute. Example:
\begin{example}
  call, file = "lattice.bmad"             ! Read in a lattice file
  quadrupole::*[multipoles_on] = False    ! But I want the multipoles off.
  q1[k1] = 0.3                            ! k1 attribute not affected.
\end{example}
\vn{multipoles_on} only effect multipoles specified by \vn{An}, \vn{Bn}, \vn{An_elec}, or
\vn{Bn_elec}. Other multipoles, like the \vn{k2} multipole of a \vn{sextupole}, are not
affected. The exception is \vn{multipole} and \vn{ab_multipole} elements do not have the
\vn{multipoles_on} attribute. Rather they can be toggles on/off using the \vn{is_on}
attribute.

%-----------------------------------------------------------------
\section{Field Maps}
\label{s:fieldmap}
\index{field maps}

There are two general ways to specify complicated electro-magnetic field configurations
that cannot be simply modeled using multipoles. One way is to use \vn{custom}
fields. Specifying a custom field is done by using custom code and linking this code with
\bmad into a program. That is, custom fields are defined outside of the \bmad software
(\sref{s:integ}).

\index{cylindrical_map}\index{cartesian_map}\index{grid_field}\index{gen_grad_map}
The other way to specify a complicated field is to use a ``\vn{field map}''. There
are four types of field maps:
\begin{example}
  cartesian_map       ! \sref{s:cart.map}
  cylindrical_map     ! \sref{s:cylind.map}
  grid_field          ! \sref{s:grid.field}
  gen_grad_map        ! \sref{s:gen.grad.map}
\end{example}
Essentially, \vn{cylindrical_map} and \vn{cartesian_map} define fields using a set of
functions with user defined coefficients with the functions formulated to obey Maxwell's
equations. The \vn{grid_field} type defines the field on a grid of points and
interpolation is used to evaluate the field in between the points. Finally, the
\vn{gen_grad_map} type defines a set of ``generalized gradients'' (\sref{s:gen.grad.phys}).

The \vn{cylindrical_map} and \vn{grid_field} types can be used with both RF and DC fields.
The other two types can only be used with DC fields. RF fields may only be used with the following
element classes:
\begin{example}
  e_gun         ! \sref{s:e.gun}
  em_field      ! \sref{s:em.field}
  lcavity       ! \sref{s:lcav}
  rfcavity      ! \sref{s:rfcav}
\end{example}

An element may specify multiple fields of a given type and/or may define multiple fields
of different types. In both these cases, the field in the element is taken to be the sum
of the individual fields. For example:
\begin{example}
  sb: sbend, field_calc = fieldmap, cylindrical_map = \{...\},  cylindrical_map = \{...\}
\end{example}
In this example an element has two \vn{cylindrical_map} fields and the total field is the
sum of the fields of each one. Separating fields like this can be useful, for example, to
decouple the specification of electric from magnetic fields, or to decouple the
specification of AC and DC fields.

The field of one element can overlap onto other elements. This is explained in
Sec.~\sref{s:overlap}.

Field maps are used with integration type tracking methods (\sref{s:integ}).  It is important to
note that field maps are {\em ignored} by \vn{bmad_standard} tracking. Additionally, \vn{grid_field}
field maps cannot be used with \vn{symp_lie_ptc}.

Field maps may extend longitudinally beyond the ends of an element. See Sec~(\sref{s:overlap}) for
more details.

In a lattice file, once a field map is defined for an element, components of the field map
may be redefined using the notation
\begin{example}
  ele_name[fieldmap_name(index)%component_name] = value
\end{example}
where \vn{ele_name} is the name of the element, \vn{fieldmap_name} is the name of the
type of field map, \vn{index} is the index of the field map which is ``1'' for the first
field map defined for an element, etc., \vn{component_name} is the name of the component,
and \vn{value} is the value to set to. Example:
\begin{example}
  qq, quadrupole, grid_field = \{field_scale = 0.5, ...\}, ...
  qq[grid_field(1)%field_scale] = 0.7  ! Change field_scale value
\end{example}

%-----------------------------------------------------------------
\subsection{Field Map Common attributes}
\label{s:fieldmap.com}

\begin{figure}[tb]
  \centering
  \includegraphics{bend-grid-coords.pdf}
  \caption[\vn{Field map} coordinates when used with a bend element.]{
When used with a bend element, by default, field map coordinates will be Cartesian and not curved
like the reference orbit. The orientation of the field map coordinates is determined by the setting
of \vn{ele_anchor_pt}. To use curvilinear coordinates instead, \vn{curved_ref_frame} must be set to
\vn{True} [Available in \vn{grid_field} and \vn{gen_grad_map} only].}
  \label{f:bend.grid}
\end{figure}

This section explains some of the attributes that are common to the \vn{field map}
types. Not all attributes are used in all field map types. See the documentation on the
individual types for a list of the attributes pertinent to that type.

\index{field_type}
\index{field_scale}\index{master_parameter}
  \begin{description}
%
\item[curved_ref_frame] \Newline
For bends, the coordinates of the field are, by default, Cartesian and do not follow the
curved bend coordinates. The orientation of the field map coordinates with respect to the
bend is determined by the placement of the anchor point (specified by \vn{ele_anchor_pt})
as shown in \fig{f:bend.grid}. In this case, when tracking a particle, \bmad will convert
particle coordinates (which are expressed in the bend's curvilinear coordinate system
defined by the reference orbit) to the Cartesian coordinates of the field map and will
rotate the computed field from the field map coordinates back to the particle coordinates.

For \vn{grid_field} and \vn{gen_grad_map} types only, this default behavior can be changed by
setting the \vn{curved_ref_frame} component of the field map to \vn{True}. In this case,
the field grid coordinates will follow curved bend coordinates.  The \vn{curved_ref_frame}
parameter is only pertinent for bend elements (\vn{sbend}s, \vn{rbend}s).  The setting of
\vn{curved_ref_frame} is ignored for non-bend elements.
%
\item[ele_anchor_pt] \Newline
The \vn{ele_anchor_pt}, along with \vn{r0}, determines the origin of the field with
respect to the lattice element. Possible settings are:
\begin{example}
  beginning   ! Beginning of element (default).
  center      ! Center of element.
  end         ! Exit end of element.
\end{example}
Example:
\begin{example}
  rfc0: rfcavity, gen_grad_map = \{ele_anchor_pt = center, ...\}, ...
\end{example}
%
\item[field_type] \Newline
The \vn{field_type} attribute sets the type of field described. Possible settings for
\vn{field_type} are:
\begin{example}
  electric      ! Pure electric field. For DC fields only.
  magnetic      ! Pure magnetic field. For DC fields only.
  mixed         ! Mixed EM fields. Used for \vn{grid_field} only.
\end{example}
Example:
\begin{example}
  bb: sbend, cartesian_map = \{field_type = electric, ...\}, ...
\end{example}
The \vn{cylindrical_map} type does not have a \vn{field_type} since it has explicit arrays
for the electric and magnetic fields.
%
\item[field_scale] \Newline
The \vn{field_scale} attribute is used to scale the overall field magnitude. The default
value is 1. A value of -1 will reverse the field. If the \vn{master_parameter} is
defined, it is multiplied with the \vn{field_scale} to give the overall scale.
Example:
\begin{example}
  qq, quadrupole, grid_field = \{field_scale = 0.5, ...\}, ...
  qq[grid_field(1)%field_scale] = 0.7     ! Change value after element def.
  qq[grid_field(1)%master_parameter] = k1 ! Change value after element def.
\end{example}
%
\item[harmonic] \Newline
The \vn{harmonic} attribute, along with \vn{rf_frequency} element attribute, sets the oscillation
frequency of the field map. The \vn{harmonic} attribute is only used with \vn{cylindrical_map} and
\vn{grid_field} types. The default value of \vn{harmonic} is 0.  The \vn{harmonic} number needs to
be 0 for DC fields. Example:
\begin{example}
  lc1: lcavity, rf_frequency = 500e6, grid_field = \{harmonic = 2, ...\}, ...
\end{example}
Notice that \vn{rf_frequency} is set outside of any field map and is common to all field maps.
  \item[master_parameter] \Newline
The \vn{master_parameter} defines a ``master'' element attribute for scaling the field. Example:
\begin{example}
  qq: quadrupole, gen_grad_map = \{master_parameter = "K1", ...\}, k1 = ...
\end{example}
This example defines the \vn{master_parameter} for the \vn{gen_grad_map} to be the
quadrupole strength \vn{k1}. By using the same master parameter for a set of field map
instances within a given lattice element, the sum field of the set can be controlled by a
single attribute. The \vn{master_parameter} must be set to a valid element attribute. If
the name is blank (""), no master parameter is used. The \vn{master_parameter}, if
defined, is multiplied with the \vn{field_scale} to give the value used to scale the
fields. The default \vn{master_parameter} is blank ("") except for \vn{wiggler} elements
where, for historical reasons, the default is \vn{polarity}.
%
\item[phi0_fieldmap] \Newline
For AC fields, \vn{phi0_fieldmap} is the phase of the field map field relative to the fundamental
mode. The phase \vn{phi0_fieldmap} is relative to the fundamental frequency and not the frequency of
the field map mode. That is, the ``zero crossing'' point of the field map is shifted by a time
\vn{phi0_fieldmap}/$f_0$ where $f_0$ is the fundamental mode frequency.
%
\item[r0] \Newline
The \vn{r0} attribute is the $(x0, y0, z0)$ vector specifying the offset of the origin point that
defines the field relative to the \vn{anchor} point defined by \vn{ele_anchor_pt}.  The origin
position of the field (\vn{r_origin}) is determined by
\begin{example}
  r_origin = r0 + r_anchor
\end{example}
where \vn{r_anchor} is determined by the setting of \vn{ele_anchor_pt}. In the reference
coordinates (\sref{s:ref}) with respect to the element \vn{r_anchor} is:
\begin{example}
  ele_anchor_pt       r_anchor
  -------------       ---------
  beginning           (0, 0, 0)      ! Default
  center              (0, 0, L/2)
  end                 (0, 0, L)
\end{example}
with \vn{L} being the length of the element. 

Example:
\begin{example}
  rfc0: rfcavity, gen_grad_map = \{r0 = (-0.23, ...), ...\}, ...
\end{example}
  \end{description}

%-----------------------------------------------------------------
\subsection{Cartesian_Map Field Map}
\label{s:cart.map}
\index{cartesian_map}

The \vn{cartesian_map} is used to describe DC fields. Each term of a \vn{cartesian_map} is a solution
of Laplace's equation in cartesian coordinates.  as described in Sec.~\sref{s:cart.map.phys}.

The lattice file syntax for the \vn{cartesian_map} type is:
\begin{example}
  cartesian_map = \{
    field_type       = <String>, ! Type of field: Default = Magnetic.
    field_scale      = <Real>,   ! Scale factor for the E & B fields.
    master_parameter = <Name>,   ! Master scaling parameter for E & B fields.
    ele_anchor_pt    = <Real>,   ! Anchor position: Beginning (default), Center, or End.
    r0               = (<x0>, <y0>, <z0>), ! Anchor offset. Default is 0.
    term = \{<A>, <k_x>, <k_y>, <k_z>, <x_0>, <y_0>, <phi_z>, <family>\}, 
    term = \{....\},
    ... more terms ...
  \}
\end{example}
The possible settings of \vn{<family>} are explained in Sec.~\sref{s:cart.map.phys}. Example:
\begin{example}
  q01: quadrupole, l = 0.6, field_calc = fieldmap,
        cartesian_map = \{
          term = \{0.03, 3.00, 4.00, 5.00, 0, 0, 0.63, y\},
          term = \{...\}, ...    \}
\end{example}

See Sec.~\sref{s:fieldmap.com} for an explanation of the attributes that are common with other field
map types.

See Sec.~\sref{s:attrib.non} for the syntax of setting Cartesian map components.

To use with PTC dependent tracking methods (\sref{s:integ}) there are a number of restrictions:
\begin{itemize}
%
\item
There can be only one \vn{cartesian_map} field map and there cannot be any other field maps of any
kind.
%
\item 
\vn{cartesian_map} may not be used with a bend.
%
\item
Only magnetic fields may be used. 
%
\item
The transverse terms in \vn{r0} must be zero.
%
\item
Since PTC evaluates the vector potential (\sref{s:wiggler.std}), and since \vn{k_z} appears in the
denominator of some terms, \vn{k_z} must be non-negative for all terms.
  \end{itemize}

%-----------------------------------------------------------------
\subsection{Cylindrical_Map Field Map}
\label{s:cylind.map}
\index{ccylindrical map}

The \vn{cylindrical_map} is used to describe both DC and AC fields. Each term of a
\vn{cylindrical_map} is a solution of Laplace's equation in cylindrical coordinates.  as described
in Sec.~\sref{s:cylind.map.phys}.

The lattice file syntax for the \vn{cylindrical_map} type is:
\begin{example}
  cylindrical_map = \{
    field_scale      = <Real>,    ! Scale factor for the E & B fields.
    master_parameter = <Name>,    ! Master scaling parameter for E & B fields.
    ele_anchor_pt    = <Real>,    ! Anchor position: Beginning (default), Center, or End.
    m                = <Integer>, ! Azimuthal mode number
    harmonic         = <Integer>, ! RF frequency harmonic number 
    phi0_fieldmap    = <Real>,    ! Phase of oscillations.
    theta0_azimuth   = <Real>,    ! Azimuthal orientation.
    r0               = (<x0>, <y0>, <z0>), ! Anchor offset. Default is 0.
    dz        = <Real>,                    ! Distance between sampled field points.
    e_coef_re = (<Real>, <Real>, ....),    ! Real part of E.
    e_coef_im = (<Real>, <Real>, ....),    ! Imaginary part of E.
    b_coef_re = (<Real>, <Real>, ....),    ! Real part of B.
    b_coef_im = (<Real>, <Real>, ....),    ! Imaginary part of B.
  \}
\end{example}
See Sec.~\sref{s:fieldmap.com} for an explanation of the attributes that are common with
other field map types.

For DC fields, the \vn{e} coefficients specify the electric fields and the \vn{b} coefficients
specify the magnetic fields. For AC fields, the \vn{e} coefficients specify modes that have finite
longitudinal electric fields while the modes associated with the \vn{b} coefficients do not.

To specify the RF frequency, specify the \vn{rf_frequency} {\em element} attribute along with
the \vn{harmonic} attribute. See the discussion of the \vn{harmonic} attribute in 
Sec.~\sref{s:fieldmap.com}.

The basic equations used for the \vn{cylindrical_map} decomposition of the fields are given in
Section~\sref{s:cylind.map.phys}. A lattice element may have multiple \vn{cylindrical_map}
components with each \vn{cylindrical_map} being associated with a particular azimuthal mode $m$.

\vn{e_re} and \vn{e_im} give the real an imaginary part of $e$ and \vn{b_re} and \vn{b_im} give the
real and imaginary part of $b$. All of these vectors must be present and have the same length. The
exception is with an $m = 0$ mode either the $e$ or $b$ arrays can be omitted and will default to
zero. The number of terms $N$ for the $e$ or $b$ vectors must be a power of $2$ and all modes must
have the same number of terms. The $n$\Th element in the $e$ or $b$ arrays, with $n$ running from 0
to $N-1$, is associated with a wavelength $k_n$
\begin{equation}
  k_n = \begin{cases}
    \frac{2 \, \pi \, n}{N \, dz} & 0 \le n < \frac{N}{2} \\
    \frac{2 \, \pi \, (n-N)}{N \, dz} & \frac{N}{2} \le n \le N - 1
  \end{cases}
\end{equation}
This convention produces less high frequency components then the convention of using $k_n = 2 \, \pi
\, n / N dz$.

The longitudinal length of the field is
\begin{equation}
  L_{\text{field}} = \frac{N - 1}{dz}
\end{equation}
this may be different from the length \vn{l} specified for the element.

For AC fields, the time $t$ in \Eq{eseei} is computed depending upon whether \vn{absolute time
tracking} or \vn{relative time tracking} is being used as discussed in \sref{s:rf.time}. For
\vn{rfcavity} elements, the phase factor $\phi_{0j}$ in \Eq{eseei} is computed by
\begin{example}
  \(\phi_{0j}\) = harmonic(j) * [0.25 - (phi0 + phi0_multipass + phi0_err + 
                                                  phi0_autoscale + phi0_fieldmap(j))] 
\end{example}
where \vn{phi0_fieldmap(j)} and \vn{harmonic(j)} are specific to the j\Th grid field
while the other factors are element parameters and so will be the same for all grid field
maps of a given element. For non \vn{rfcavity} elements the phase is
\begin{example}
  \(\phi_{0j}\) = harmonic(j) * [phi0 + phi0_multipass + phi0_err + 
                                                  phi0_autoscale + phi0_fieldmap(j)]
\end{example}
where \vn{phi0_fieldmap(j)} and \vn{harmonic(j)} are specific to the j\Th cylindrical field map
while the other factors are element parameters and so will be the same for all cylindrical field
maps of a given element.

Example:
\begin{example}
  m1: lcavity, rf_frequency = 1e6, voltage = 2e6, cylindrical_map = \{
    m = 2,                   harmonic = 3,
    r0 = (0, 0, 0.001),      dz = 0.1,
    theta0_azimuth = 0.3,    field_scale = 0.7,
    ele_anchor_pt = center,  master_parameter = voltage,
    e_coef_re = (...),       e_coef_im = (...),
    b_coef_re = (...),       b_coef_im = (...)\}, field_calc = fieldmap
\end{example}

See Sec.~\sref{s:attrib.non} for the syntax of setting Cylindrical map components.

Note: When using PTC based tracking (\sref{c:methods}), the following restrictions apply:
\begin{Itemize}
%
\item
The fields must be DC.
%
\item
all the \vn{e_coef} and \vn{b_coef} arrays must have the same length.
%
\item
\vn{r0(1)} and \vn{r0(2)} (the transverse offsets) must be zero.
%
\item
The element containing the map cannot be an \vn{sbend} or \vn{rbend}.
%
\item
May not be combined with other field map types.
\end{Itemize}

%-----------------------------------------------------------------
\subsection{Grid_Field Field Map}
\label{s:grid.field}
\index{grid field}

A \vn{grid_field} is grid of field points specified using the syntax:
\begin{example}
  grid_field = \{ 
    geometry         = <String>,    ! Geometry of the grid.
    field_type       = <String>,    ! Type of field: Default = Mixed.
    field_scale      = <Real>,      ! Scale factor for the E & B fields.
    phi0_fieldmap    = <Real>,      ! Phase of oscillations.
    harmonic         = <Int>,       ! RF frequency harmonic number 
    interpolation_order = <Int>,    ! 1 (default) or 3 interpolation polynomial order.
    master_parameter = <Name>,      ! Master scaling parameter for E & B fields.
    curved_ref_frame = <Logical>,   ! Use a curved reference frame with bends?
    r0   = (...),                   ! Grid origin. Syntax is \vn{geometry} dependent.
    dr   = (...),                   ! Grid spacing. Syntax is \vn{geometry} dependent.
    ele_anchor_pt = <Position>      ! BEGINNING, CENTER, or END
    \{ ... \},                      ! Field table. Syntax is \vn{geometry} dependent.
    pt(<Integer>, \dots) = ( \ldots ), ! Field table point. Old style.
  \}
\end{example}
See Sec.~\sref{s:fieldmap.com} for an explanation of the attributes that are common with
other field map types.

To specify the RF frequency, specify the \vn{rf_frequency} {\em element} attribute along with
the \vn{harmonic} attribute. See the discussion of the \vn{harmonic} attribute in 
Sec.~\sref{s:fieldmap.com}.

For \vn{field_type} set to \vn{electric} or \vn{magnetic}, the field is DC. That is, For
\vn{field_type} set to \vn{electric} or \vn{magnetic}, the value of \vn{harmonic} must be
0. For \vn{field_type} set to \vn{mixed}, the field may be DC or AC. 

For AC fields, the individual field components are complex.  the syntax for specifying a complex
number is:
\begin{example}
  (<Re> <Im>)
\end{example}
Example:
\begin{example}
  \{
    0 0 -7: (0.34 -4.3) (2.37 9.34) ...,    ! Complex field
    0 0 -7: 0.12 -0.33 ...,                 ! Imaginary components are zero
    ...
  \}
\end{example}

The actual fields $\bfE$ and $\bfB$ are computed from the complex fields $\bfE_c$ and $\bfB_c$ via
\begin{equation}
  \bfE = \Re \Bigl[ \bfE_c \exp \left( -2 \, \pi \, i \, (\phi_t + \phi_\REF) \right) \Bigr]
\end{equation}
with a similar equation for $\bfB$. $\phi_t$ is the part of the phase due to when the particle
arrives at the cavity and depends upon whether \vn{absolute time tracking} or \vn{relative time
tracking} is being used as discussed in \sref{s:rf.time}. The phase $\phi_\REF$ for the
$j$\Th grid field in an \vn{rfcavity} element is
\begin{example}
  \(\phi_\text{ref,j}\) = harmonic(j) * [0.25 - (phi0 + phi0_multipass + phi0_err + 
                                                  phi0_autoscale + phi0_fieldmap(j))] 
\end{example}
where \vn{phi0_fieldmap(j)} and \vn{harmonic(j)} are specific to the j\Th grid field while the other
factors are element parameters and so will be the same for all grid field maps of a given
element. For non \vn{rfcavity} elements the phase is
\begin{example}
  \(\phi_\text{ref,j}\) = harmonic(j) * [phi0 + phi0_multipass + phi0_err + 
                                                  phi0_autoscale + phi0_fieldmap(j)]
\end{example}

The \vn{geometry} switch sets the type of the grid and must come before the field table. The
possible settings of \vn{geometry} are:
\begin{example} 
  rotationally_symmetric_rz
  xyz
\end{example}

The \vn{rotationally_symmetric_rz} setting for \vn{geometry} is for fields that are rotationally
symmetric around the $z$ axis. The format for this type of \vn{grid_field} is
\begin{example}
  grid_field = \{ 
    geometry = rotationally_symmetric_rz,
    r0   = (<x0>, <y0>, <z0>),        ! Grid origin 
    dr   = (<dr>, <dz>),              ! Grid spacing
    \{                                ! Field table...
      <ir> <iz>: <E_r> <E_phi> <E_z>,   ! For field_type = Electric
      <ir> <iz>: <B_r> <B_phi> <B_z>,   ! For field_type = Magnetic
      <ir> <iz>: <E_r> <E_phi> <E_z> <B_r> <B_phi> <B_z>,
                                        ! For field_type = Mixed.
      \ldots 
    \} 
  \}
\end{example}
where \vn{<iz>} can be negative but \vn{<ir>} must be non-negative. Notice that commas are only used
in the field table to demarcate between field points.

There is an old style syntax for the field table that looks like:
\begin{example}
    pt(<ir>, <iz>) = (<E_r>, <E_phi>, <E_z>) ! For field_type = Electric
    pt(<ir>, <iz>) = (<B_r>, <B_phi>, <B_z>) ! For field_type = Magnetic
    pt(<ir>, <iz>) = (<E_r>, <E_phi>, <E_z>, <B_r>, <B_phi>, <B_z>)
                                             ! For field_type = Mixed.
\end{example}
While the old style is accepted, parsing the old syntax is almost a factor of two slower. The only
possible advantage to the old style is that expressions can be used for field values.

The \vn{xyz} setting for \vn{geometry} can be used for all rectangular field grids. The format for
this type of \vn{grid_field} is
\begin{example}
  grid_field = \{ 
    geometry = xyz,
    r0   = (<x0>, <y0>, <z0>),    ! Grid origin 
    dr   = (<dx>, <dy>, <dz>),    ! Grid spacing
    \}
      <ix> <iy> <iz>: <E_x> <E_y> <E_z>,  ! For field_type = Electric
      <ix> <iy> <iz>: <B_x> <B_y> <B_z>,  ! For field_type = Magnetic
      <ix> <iy> <iz>: <E_x> <E_y> <E_z> <B_x> <B_y> <B_z>, 
                                                   ! For field_type = Mixed.
      \ldots 
    \}
  \}
\end{example}
where \vn{<ix>}, \vn{<iy>}, and \vn{<iz>} can be negative.

The old style syntax for the \vn{xyz} field table is:
\begin{example}
    pt(<ix>, <iy>, <iz>) = (<E_x>, <E_y>, <E_z>),  ! For field_type = Electric
    pt(<ix>, <iy>, <iz>) = (<B_x>, <B_y>, <B_z>),  ! For field_type = Magnetic
    pt(<ix>, <iy>, <iz>) = (<E_x>, <E_y>, <E_z>, <B_x>, <B_y>, <B_z>), 
                                                   ! For field_type = Mixed.
\end{example}


[For clarity sake, the following discusses the \vn{xyz} case. Extension to other cases is straight
forward.]  There is no restriction on the bounds of the indexes \vn{(ix, iy, iz)} of the \vn{pt(ix,
iy, iz)} array. A point $(ix, iy, iz)$ corresponds in space to the point $(x, y, z)$:
\begin{example}
  (x, y, z) = dr * (ix, iy, iz) + r0 + r_anchor
\end{example}
where \vn{z} is measured from the beginning of the element and
\vn{r_anchor} is determined by the setting of \vn{ele_anchor_pt}:
\begin{example}
  ele_anchor_pt       r_anchor
  -------------       ---------
  beginning           (0, 0, 0)      ! Default
  center              (0, 0, L/2)
  end                 (0, 0, L)
\end{example}
with \vn{L} being the length of the element. 

Example:
\begin{example}
  apex: e_gun, l = 0.23, field_calc = fieldmap, rf_frequency = 187e6, 
                                  grid_field = call::apex_gun_grid.bmad
\end{example}
with the file \vn{apex_gun_grid.bmad} being:
\begin{example}
  \{
    geometry = rotationally_symmetric_rz,
    harmonic = 1,
    master_parameter = voltage,
    r0 = (0, 0),
    dr = (0.001, 0.001),
    \{
      0 0:  (0 0) (0 0) (1 0)     (0 0) (0 0) (0 0),
      0 1:  (0 0) (0 0) (0.99 0)  (0 0) (0 0) (0 0),
      ... 
    \}
  \}
\end{example}

The order of the polynomials used to interpolate the field is determined by the setting of
\vn{interpolation_order}. The value of this integer may be either \vn{1} or \vn{3}. A value of
\vn{1} is the default and gives linear interpolation.  A value of \vn{3} will give cubic
interpolation which will be better if the field values on the grid are smooth and well behaved.
However, a cubic interpolation will be slower and will magnify gird field errors so care should be
taken in choosing the interpolation order.

It is considered an error if the field of the grid is evaluated for a point that is transversely
outside of the grid. That is, a grid must extend transversely to the aperture or at least beyond the
trajectory of any particle. [Actually, to prevent problems when the aperture is set at the grid
boundary, if the distance between the particle and the grid boundary is within 1/2 of the spacing
between grid points, no error is generated and the field will be calculated using extrapolation.] On
the other hand, it is acceptable to evaluate the grid field at a point that is longitudinally
outside of the grid. In this case, the field is assumed to go to zero. This is done by effectively
adding to the grid two planes of zero field longitudinally to either side of the grid. So a particle
traveling outside of the grid longitudinally will see the field drop to zero within one longitudinal
grid spacing length.

Grid fields may stored in \vn{HDF5} binary format which may then be called using an \vn{inline call}
(\sref{s:call.inline}). For example:
\begin{example}
  qq: quadrupole, grid_field = call::my_grid.h5, ...
\end{example}

See Sec.~\sref{s:attrib.non} for the syntax of setting \vn{grid_field} components.

%-----------------------------------------------------------------
\subsection{Gen_Grad_Map}
\label{s:gen.grad.map}
\index{gen_grad_map}

The \vn{gen_grad_map} characterizes DC magnetic or electric fields using ``generalized gradients''
(GG) as described by Venturini and Dragt\cite{b:gen.grad}. Formulas for the GG are given in
\Sref{s:gen.grad.phys}.

An element can store an set of GGs. For example, a GG for the magnetic field and a GG for the
electric field. The syntax for describing a single GG is:
\begin{example}
  gen_grad_map = \{
    field_type       = <String>,    ! Type of field: Default = magnetic.
    field_scale      = <Real>,      ! Scale factor for the E & B fields.
    master_parameter = <Name>,      ! Master scaling parameter for E & B fields.
    curved_ref_frame = <Logical>,   ! Use curved coords with bends?
    ele_anchor_pt    = <Real>,      ! Anchor position: Beginning (default), Center, or End.
    r0               = (<x0>, <y0>, <z0>), ! Anchor offset. Default is 0.
    dz               = <Real>,             ! Distance between sampled field points.
    curve = \{                      ! Generalized gradient curve
      m = <Int>,                    ! Azimuthal m value
      kind = <Sin-or-Cos>           ! Type of curve: sin or cos.
      derivs = \{
        <z0>: <d0> <d1> <d2> ... <dN>,
        <z1>: ...,
        ...
      \}
    \},
    curve = \{
      ...
    \}
  \}
\end{example}

Example:
\begin{example}
  t1: sbend, k1 = 2, gen_grad_map = \{
    field_type = electric,   ele_anchor_pt = end, 
    dz = 1.2,                r0 = (0, 0, 2.0),
    field_scale = 1.3,       curved_ref_frame = False,
    master_parameter = k1,

    ...  \}, field_calc = fieldmap, integrator_order = 6, num_steps = 70, 
  tracking_method = taylor, mat6_calc_method = taylor
\end{example}

See Sec.~\sref{s:fieldmap.com} for an explanation of the attributes that are common with other field
map types. In general there will be multiple \vn{curve} instances.  A given \vn{curve} instance
specifies one generalized gradient with a given azimuthal \vn{m} value and a \vn{kind} which
specifies if the generalized gradient is \vn{sin}-like or \vn{cos}-like. Each \vn{curve} has a table
of derivatives given by the \vn{derivs} component. Each row in the derivative table starts with the
z-position of the derivatives \vn{<z0>}, \vn{<z1>}, \vn{z2}, ... etc. The z-positions must be in
increasing order and be multiples of \vn{dz} with no gaps (that is, \vn{<z1>} = \vn{<z0>} + \vn{dz},
etc.). Each line in the \vn{derivs} table specifies the derivatives \vn{<d0>}, \vn{<d1>}, etc. at
the given z-position where \vn{<d0>} is the zeroth derivative (the value of the generalized gradient
itself), \vn{<d1>} is the first derivative, etc. For a given \vn{curve} instance, all derivative
rows must have the same number of derivatives. Different \vn{curves} can have differing number of
derivatives. All curves must specify the same z-positions but different curves may have differing
number of derivatives.

Like \vn{grid_field}s, the edges of a \vn{gen_grad_map}, can be defined to be outside of the edges
of the element. These areas will be ignored while tracking unless field overlap is defined
(\sref{s:overlap})).

See Sec.~\sref{s:attrib.non} for the syntax of setting \vn{gen_grad_map} components.

To use with PTC dependent tracking methods (\sref{s:integ}) there are a number of restrictions:
  \begin{itemize}
%
\item
There can be only one \vn{gen_grad_map} and there cannot be any other field maps of any kind.
%
\item
Only magnetic fields may be used.
%
\item
In a bend with \vn{curved_ref_frame} = False, The setting of \vn{ele_anchor_pt} must be
\vn{center}.
  \end{itemize}

The \vn{integrator_order} for PTC dependent tracking can be either set to 4 or 6 with 4 being the
default\vn{s:ptc.integ}.

%-----------------------------------------------------------------
\section{RF Couplers}
\label{s:rf.coupler}

\index{lcavity}\index{rfcavity}
\index{coupler_at}\index{coupler_strength}
\index{coupler_angle}\index{coupler_phase}
For \vn{lcavity} and \vn{rfcavity} elements, the attributes that characterize the dipole transverse
kick due to a coupler port are:
\begin{example}
  coupler_at       = <Switch> ! What end the coupler is at
  coupler_strength = <Real>   ! Normalized strength
  coupler_angle    = <Real>   ! Polarization angle (rad/2\(\pi\))
  coupler_phase    = <Real>   ! Phase angle with respect to the RF (rad/2\(\pi\))
\end{example}
The possible \vn{coupler_at} settings are:
\begin{example}
  entrance_end
  exit_end  ! default
  both_ends
\end{example}
The kick due to the coupler is
\begin{example}
  dP_x = amp * cos(phase) * cos(angle) 
  dP_y = amp * cos(phase) * sin(angle)
  dE   = amp * (cos(angle) * x + sin(angle) * y) * sin(phase) * twopi * rf_frequency / c_light 
\end{example}
where \vn{dP_x} and \vn{dP_y} are the transverse momentum kicks, \vn{dE} is an energy kick, and
\begin{example}
  amp   = gradient * coupler_strength 
  phase = twopi * (phase_particle + phase_ref + coupler_phase)         ! For lcavity \sref{s:lcav}
        = pi/2 + twopi * (phase_particle - phase_ref + coupler_phase)  ! For rfcavity \sref{s:lcav}
  angle = twopi * coupler_angle
\end{example}
The energy kick is needed to keep things symplectic. 

Example:
\begin{example}
  rf1: lcav, l = 4.5, gradient = 1.2e6, coupler_at = both_ends,
                                                  coupler_strength = 0.037
\end{example}

%-----------------------------------------------------------------
\section{Field Extending Beyond Element Boundary}
\label{s:overlap}

\index{field_overlaps} 
The \vn{field_overlaps} element attribute can be used to indicate that the electric or
magnetic fields of one element overlap another element. The syntax is:
\begin{example}
  <overlapping_ele>: ... field_overlaps = \{<overlapped_ele1>, <overlapped_ele2>, ...\}
\end{example}
The \{\} braces are optional if there is only one overlapped element.

Example:
\begin{example}
  b1: sbend, l = 2.3, field_overlaps = \{q1, s2\}, ...
  inj_line: line = (..., s2, b1, mark3, q1, ...)
\end{example}
In this example, the field of element \vn{b1} extends beyond the ends of \vn{b1} and overlaps
elements \vn{q1} and \vn{s2}. There is no limit to the number of elements that are overlapped by any
given element and overlapped elements do not have to be next to the overlapping element in the
line. If there are multiple elements whose name matches the name of a overlapped element, the
element closest to the overlapping element is chosen. Thus in the above example, if there are
multiple elements named \vn{q1}, the closest \vn{q1} to \vn{b1} is designated as the overlapped
element.

There can be multipole \vn{field_overlaps = ...} constructs for an overlapping element.
Thus the following is equivalent to the above example:
\begin{example}
  b1: sbend, l = 2.3, field_overlaps = q1, field_overlaps = s2
\end{example}

Note: When the field overlaps elements that are superimposed (\sref{c:super}), the overlapped 
elements must be the \vn{super_lord} elements and never the slaved elements.

The field, when \vn{field_calc} (\sref{s:integ}) is set to \vn{bmad_standard}, never extends beyond
the element boundary and so a \vn{bmad_standard} field will never overlap another element.

%-----------------------------------------------------------------
\section{Automatic Phase and Amplitude Scaling of RF Fields}
\label{s:autoscale}
\index{automatic field scaling|hyperbf}

\index{e_gun}\index{em_field}\index{lcavity}\index{rfcavity}
Elements that have accelerating fields are:
\begin{example}
  e_gun       ! \sref{s:e.gun}
  em_field    ! \sref{s:em.field}
  lcavity     ! \sref{s:lcav}
  rfcavity    ! \sref{s:rfcav}
\end{example}
[Notice that \vn{rfcavity} elements by definition, have a constant reference energy while with all
the other elements the entrance end reference energy will, in general, be different from the exit
end reference energy.]

The problem that arises with accelerating fields is how to set the overall amplitude (and phase if
the fields are oscillating) of the field so that a particle, starting on the reference orbit and
starting with the reference energy, has the desired energy gain at the exit end of the element where
the ``desired'' is set by the \vn{voltage} or \vn{gradient} attribute of the element plus a
\vn{phi0} phase attribute for AC fields.

The scaling problem is not present when \vn{bmad_standard} tracking (\sref{s:tkm}) is used since
\vn{bmad_standard} tracking uses an integrated formula that is designed to give the proper
acceleration. Rather it is a problem for Runge-Kutta and other integration methods.

The problem becomes even more complicated at non-ultra relativistic energies where the particle
velocity is not a constant. In this case, the proper amplitude and/or phase settings will depend
upon what the incoming energy of the reference particle is.

To help with the scaling problem, \bmad has the capability to automatically scale an accelerating
field's amplitude and/or phase. The two lattice element parameters that turn on/off auto scaling are
(\sref{s:param}):
\begin{example}
  autoscale_phase      = <Logical>  ! Automatic phase scaling.
  autoscale_amplitude  = <Logical>  ! Automatic amplitude scaling.
\end{example}
The default value is True for both parameters. Example:
\begin{example}
  rf2: rfcavity, autoscale_phase = F
\end{example}

Scaling takes place during program execution when a lattice is initially created (that is,
when the lattice file is parsed) and when parameters in the lattice that would change the
scaling are varied.  The element parameters varied when autoscaling is done are:
\begin{example}
  field_autoscale       ! Amplitude scale
  phi0_autoscale        ! phase scale
\end{example}
For an \vn{rfcavity} element, the \vn{field_autoscale} parameter is set such that when the phase is
adjusted for maximum acceleration, the voltage gain of a particle on the zero orbit is equal to the
value of the element's \vn{voltage} parameter. 

The \vn{phi0_autoscale} is set so that with \vn{phi0} equal to zero, a particle on the zero orbit
will not see any energy gain through the cavity. There are two values for \vn{phi0_autoscale} where,
with \vn{phi0} equal zero, the energy gain of the zero orbit particle is zero. If the \vn{bmad_com}
global parameter \vn{rf_phase_below_transition_ref} (\sref{s:bmad.common}) is set to \vn{False} (the
default), \vn{phi0_autoscale} will be set such that, with \vn{phi0} equal zero, the zero orbit for a
particle above transition will be at the stable zero-crossing. For a particle below transition,
setting \vn{rf_phase_below_transition_ref} to \vn{True}, will result in \vn{phi0_autoscale} being
set such that, for particles below transition, and with \vn{phi0} equal zero, a particle on the zero
orbit will be at the stable zero-crossing.

It is important to keep in mind that with a ring, the closed orbit will be the equilibrium
orbit. For example, for a ring with a single cavity, changes in \vn{phi0} will just result in
variations in the closed orbit $z$. The voltage kick that the closed orbit particle gets going
through the cavity is independent of \vn{phi0} and will be equal to the radiation losses throughout
the ring (and if radiation of off, the kick will be zero).

It is not possible to autoscale if the voltage is very small. \bmad sets a lower limit of 10~Volts
and if the voltage is set less than this no scaling is done. If no autoscaling is done, the default
setting of \vn{field_autoscale} is 1 and the default setting of \vn{phi0_autoscale} is 0.

Note: If a field map (\sref{s:fieldmap}) is used to define the cavity fields, in order for the
field amplitude to vary with changes in the value of the \vn{voltage}, the \vn{master_parameter} of
the field map must be set to \vn{voltage} (or \vn{gradient} since \vn{gradient} and \vn{voltage}
are linked).

\vn{field_autoscale} and \vn{phi0_autoscale} are not needed and therefore ignored when
\vn{bmad_standard} tracking is done.

%-----------------------------------------------------------------
\section{Wakefields}
\label{s:wakes}

Wakefield modeling is discussed in Chapter~\sref{c:wakefields.eq}. The syntax for specifying
short-range wakefields for an element is given in \Sref{s:sr.wake.syntax}. The syntax for specifying
long-range wakefields is given in \Sref{s:lr.wake.syntax}.

\bmad has two modes for tracking particles. One mode tracks individual particles one at a time. The
other mode tracks bunches of particles. Which mode is used for a given program is decided by the
program. wakefields are ignored when tracking individual particles and only used when tracking
bunches.

%-----------------------------------------------------------------
\subsection{Short-Range Wakes}
\label{s:sr.wake.syntax}
\index{wakes!short-range}

The short-range wakes for a lattice element are specified via a set of ``pseudo'' modes.  Equations
for short-range wakefields are given in \Sref{s:sr.wake.eq}.  The \vn{sr_wake} attribute is used to set
wakefield parameters. The general form of this attribute is:
\begin{example}
  sr_wake = \{
      z_max = <real>, z_scale = <real>, amp_scale = <real>, 
      scale_with_length = <logical>,
      longitudinal = \{<amp>, <damp>, <k>, <phi>, <position_dependence>\},
      ...
      longitudinal = \{...\}, 
      transverse = \{<amp>, <damp>, <k>, <phi>, <polarization>, <particle_dependence>\},
      ...
      transverse = \{...\} 
      z_long = \{...\}
    \}
\end{example}
The \vn{sr_wake} structure has optional components \vn{z_max}, \vn{z_scale}, \vn{amp_scale}, and
\vn{scale_with_length}. To specify wakes, zero or more \vn{longitudinal} sub-structures each one specifying
a single longitudinal mode, zero or more \vn{transverse} sub-structures each one specifying a
single transverse mode and zero or one \vn{z_long} sub-structures that can be used to specify the longitudinal
wake as a function of z-position (\sref{s:sr.wake.z.long}). Example:
\begin{example}
  cav9: lcavity, ..., sr_wake = \{z_max = 1.3e-3, scale_with_length = F, 
       longitudinal = \{3.23e14, 1.23e3, 3.62e3, 0.123, none\}, 
       longitudinal = \{6.95e13, 5.02e2, 1.90e3, -1.503, x_leading\}, 
       transverse = \{4.23e14, 2.23e3, 5.62e3, 0.789, none,  trailing\},
       transverse = \{8.40e13, 5.94e2, 1.92e3, 1.455, x_axis, none\} \}
\end{example}
Note: After an element has been defined, to refer to a given component use the notation:
\begin{example}
  <element-name>[sr_wake%<component-name>]
\end{example}
where \vn{<element-name>} is the name of the element and \vn{<component-name>} is the name of the component.
For example, after the \vn{cav9} element has been defined, the \vn{z_scale} component can be changed via:
\begin{example}
  cav9[sr_wake%z_scale] = 0.4 * cav9[sr_wake%z_max]
\end{example}

The first four components of both the \vn{longitudinal} and \vn{transverse} sub-structures give
$A_i$, $d_i$, $k_i$, and $\phi_i/2\pi$ of \Eq{wadzk}. The units for these components are:
\begin{center} \tt
\begin{tabular}{lll} \toprule
                & {\em Monopole}  & {\em Dipole}     \\ \midrule
  $A$           & V/Coul/m        & V/Coul/m$^2$     \\
  $d$           & 1/m             & 1/m              \\
  $k$           & 1/m             & 1/m              \\
  $\phi/2\pi$   & Radians/2$\pi$  & Radians/2$\pi$   \\ \bottomrule
\end{tabular}
\end{center}
Monopole modes are modes that are independent of transverse position and dipole modes are modes that
are linear in the transverse position. 

For the \vn{longitudinal} sub-structures, there is a 5\Th component, called
\vn{position_dependence}, gives the transverse position dependence of the wake. Possible values are:
\begin{example}
  none          ! No position dependence
  x_leading     ! Linear in the leading particle x-position
  y_leading     ! Linear in the leading particle y-position
  x_trailing    ! Linear in the trailing particle x-position
  y_trailing    ! Linear in the trailing particle y-position
\end{example}
``\vn{x_leading}'', for example, means that the wake left by a ``leading'' particle is linear
in the $x$-position of the particle while the kick to a ``trailing'' particle is independent of the
the trailing particle's transvrse position. ``\vn{y_trailing}'', on the other hand, means that the 
wake left by a leading particle is independent of the leading particle's transverse position but that
the kick felt by a trailing particle is proportional to the trailing particle's $y$-position.

For the \vn{transverse} sub-structures, there is a 5\Th component giving the polarization and a 6\Th
component specifying if the kick is dependent upon the leading or trailing particle transverse position.
Possible values for the polarization are
\begin{example}
  none          ! Kick in both x and y-planes
  x_axis        ! Kick in only the x-plane
  y_axis        ! Kick in only the y-plane
\end{example}
and possible values for the particle dependence are:
\begin{example}
  none          ! No transverse dependence
  leading       ! Depends linearly on the leading particle position.
  trailing      ! Depends linearly on the trailing particle position.
\end{example}

The \vn{z_max} component of \vn{sr_wake} specifies the maximum $z$ value at which the pseudo
mode fit is valid. The \vn{z_max} component is optional and if present and \vn{positive}, \bmad will
check that the distance between particles does not exceed \vn{z_max}. If it does, \bmad will report
an error. If \vn{z_max} is not positive it is ignored.

If \vn{scale_with_length} is False (the default is True), the length factors in Equations like
\Eqs{delvp1} and \eq{pelqxw1} are dropped. This is convenient for using zero length elements with
wake simulations.

The \vn{amp_scale} component is used to scale the amplitude of the modes. This corresponds to $A_{amp}$
in \Eq{wadzk}. The default value is 1.0.

The \vn{z_scale} is used to scale the $z$ distance in the wake equations:
\begin{example}
  z(used in equations) = z_scale * z(actual)
\end{example}
The default value is 1.0.

Note: In a beam chamber with circular symmetry, the linear terms in the \vn{longitudinal} wake are
zero and the transverse wake has no terms independent of the transverse offsets nor terms that
depend upon the trailing particle offset.

%-----------------------------------------------------------------
\subsection{Short-Range Longitudinal Wake With a Wake table}
\label{s:sr.wake.z.long}

The \vn{z_long} substructure of \vn{sr_wake} (\sref{s:sr.wake.syntax}) is an alternative to the
\vn{longitudinal} substructure. With the \vn{z_long} substructure, the wake is specified using a
table of equally spaced points in $z$-position or as a function of time. The syntax for \vn{z_long}
is
\begin{example}
  z_long = \{
    time_based = <T/F>,
    smoothing_sigma = <sigma>,
    position_dependence = <dep>,
    w = \{
      <z1>  <w1>,            ! Wake <w> table as a function of z-position (or time).
      <z2>  <w2>,
      ...
    \}
  \}
\end{example}
The \vn{w} component gives the single particle wake as a function of either z-position or time
depending upon the setting of the optional \vn{time_based} logical. The default is \vn{False}. The z
(or time) values in the \vn{w} table must be equally spaced and in increasing order. Unlike the
\vn{longitudinal} and \vn{transverse} pseudo-mode description, the wake can be finite for both
positive and negative z (or time). When the wake is parsed, if \vn{time_based} is \vn{True}, the
table is converted to be z-based using the conversion $z = -c \, t$ where $c$ is the speed of
light. If the table is not symmetric around $z = 0$ (if there are more points on one size of zero
than the other), the table is extended to be symmetric (this is needed for tracking).  During
tracking, the half-width of the wake (the magnitude of the maximum or minimum z of the symmeterized
table) must be greater than the full width of the bunch (excluding 1\% outlier particles). If not,
an error message is generated and tracking is stopped. 

The optional \vn{position_dependence} component determines if the wake depends upon the transverse coordinates
of the bunch particles. See \sref{s:sr.wake.syntax} for details. The default is \vn{none}. 

The wake is applied to a bunch by convoluting the bunch distribution with the wake.  The optional
\vn{smoothing_sigma} component, if set non-zero, applies a Gaussian smoothing filter to the
convolution. The value of \vn{smoothing_sigma} is in meters if \vn{time_based} is False or seconds
otherwise.

The \vn{z_scale} and \vn{amp_scale} parameters of the \vn{sr_wake} structure (\sref{s:sr.wake.syntax})
are used with \vn{z_long}.

%-----------------------------------------------------------------
\subsection{Short-Range Wakes --- Old Format}
\label{s:sr.wake.old}

There is an old, deprecated, format for specifying short-range wakes where the wake data is
contained in a separate file whose name is given by the \vn{lr_wake_file} attribute.
Example:
\begin{example}
  abc: lcavity, sr_wake_file = "sr.wake", lr_freq_spread = 0.0023, lr_self_wake_on = F
\end{example}
Example file:
\begin{example}
  ! Pseudo Wake modes:
  !                      Amp       Damp          K      Phase  Polar-    Transverse_
  ! Longitudinal:      [V/C/m]     [1/m]      [1/m]     [rad]  ization   Dependence
  ! Transverse:      [V/C/m^2]     [1/m]      [1/m]     [rad]  

  &short_range_modes
    longitudinal(1) = 3.23e14     1.23e3     3.62e3     0.123
    longitudinal(2) = 6.95e13     5.02e2     1.90e3    -1.503
    .. etc ..
    transverse(1) =   4.23e14     2.23e3     5.62e3     0.789   none   linear_trailing
    transverse(2) =   8.40e13     5.94e2     1.92e3     1.455
     .. etc ..
    z_max = 1.3e-3
  /
\end{example}
{\bf Notice that the old format uses radians and not radians/2$\pi$ for the phase.}

\index{none}\index{x_axis}\index{y_axis}
Possible settings for the polarization parameter are:
parameter are:
\begin{example}
  none    ! Default
  x_axis  
  y_axis 
\end{example}
The \vn{polarization} name may be abbreviated.

\index{none}\index{linear_leading}\index{linear_trailing}
The \vn{transverse_dependence} parameter sets whether the wake kick is linear in the offset of the
leading or trailing particle or is independent of the transverse offset.  Possible settings of this
parameter are:
\begin{example}
  none              ! Default for longitudinal modes
  linear_leading    ! Default for transverse modes
  linear_trailing
\end{example}
The \vn{transverse_dependence} parameter may be abbreviated. Note: Due to the way the wake file is
parsed, if \vn{transverse_dependence} is specified for a particular mode, \vn{polarization} must
also be specified.

For \vn{longitudinal} modes: If the \vn{transverse_dependence} is \vn{none} (the default), the
\vn{polarization} must also be \vn{none} (other combinations do not make sense). If the
\vn{transverse_dependence} is \emph{not} \vn{none} for a \vn{longitudinal} mode, the
\vn{polarization} must be set to \vn{x_axis} or \vn{y_axis}.

%-----------------------------------------------------------------
\subsection{Long-Range Wakes}
\label{s:lr.wake.syntax}

The \vn{lr_wake} attribute is used to set the long-range wakefield parameters for a lattice element.
Equations for long-range wakes is given in \Sref{s:lr.wake.eq}.
The general form of this attribute is:
\begin{example}
  lr_wake = \{time_scale = <real>, amp_scale = <real>, freq_spread = <real>,
      self_wake_on = <logical>, t_ref = <real>,
      mode = \{<freq_in>, <R_over_Q>, <damp>, <phi>, <m_order>, <polar_angle>, 
                                                <b_sin>, <b_cos>, <a_sin>, a_cos>\},
      ...
      mode = \{...\} \}
\end{example}
The \vn{lr_wake} structure has optional components \vn{time_scale}, \vn{amp_scale},
\vn{freq_spread}, \vn{self_wake_on}, and \vn{t_ref} along with one or more \vn{mode} sub-structures each
specifying a single long-range mode. Example:
\begin{example}
  f1 = 1.65e9;  q1 = 7e4
  cav9: lcavity, ..., lr_wake = \{time_scale = 1.7, freq_spread = 0.001,  
       mode = \{f1, 0.76, f1/(2*q1), 0, 1, unpolarized\}, 
       mode = \{-1, 0.57, 3e4, 0, 2, 0.15\} \}
\end{example}

The first six components of the \vn{mode} sub-structure correspond to $\omega/2\pi$, $R/Q$, $d$,
$\phi/2\pi$, $m$, and $\theta_p/2\pi$ in \sref{s:lr.wake.eq}. The units of $R/Q$ are
$\Omega$/meter$^{2m}$. The last four components, which are optional, correspond to $b_{\sin}$,
$b_{\cos}$, $a_{\sin}$, and $a_{\cos}$. These four components can be used as a convenient way to
save the state of the long-range wakes. These components will typically not be present in a lattice
file except for lattice files generated by a program that does wake simulations.

A negative frequency is used to designate wakes that are part of the fundamental accelerating
mode. That is, the frequency of such a mode is set to the value of \vn{rf_frequency} for the lattice
element the wake is associated with. It is an error to have a negative frequency for a mode for
elements that do not have a \vn{rf_frequency} attribute. \bmad needs to know if a wake is part of
the fundamental mode due to timing issues as discussed in \sref{s:rf.time}.

The \vn{freq_spread} component of the \vn{lr_wake} structure is used to randomly vary the long-range
mode frequencies. This can be used to spread out the long-range mode frequencies among different
cavities. The default value is zero which means there is no varying the mode frequencies. The
fractional difference between of the mode frequencies used in a simulation and the input mode
frequencies will have a Gaussian distribution with an RMS given by \vn{freq_spread}. For example, a
value of 0.01 for \vn{freq_spread} gives a 1\% variation in frequency. Note: Wake modes that are
locked to the fundamental accelerating mode (\sref{s:lr.wake.old}), are not shifted.

The \vn{self_wake_on} component can be used to turn off the long-range self-wake
which is the longitudinal kick given a particle due to the wake generated by that same
particle. [The transverse self wake is always zero.] The default setting of \vn{self_wake_on} is
\vn{True}. Turning off the self-wake, for example, can be done to avoid double counting if both
long-range and short-range wakes are defined. Also, The standard formulas for the long--range
resistive wall wake are not valid over short time scales. In this case, the self-wake should be
turned off.

The \vn{amp_scale} component is used to scale the amplitude of the modes. This corresponds to $A_{amp}$
in \Eq{wadzk}. The default value is 1.0.

The \vn{time_scale} component  is used to scale the time distance in the wake equations:
\begin{example}
  t (used in equations) = time_scale * t(actual)
\end{example}
The default value is 1.0.

The \vn{t_ref} component is the reference time used in computing the wake (\Eq{ttt}). Like the mode
components \vn{b_sin}, etc., This component is typically not set when a lattice file is
constructed.

After the long--range wake has been defined, components can be referenced or redefined using the notation
\begin{example}
  lr_wake%mode(n)%freq_in      ! Input Frequency
  lr_wake%mode(n)%freq         ! Actual Frequency (set by Bmad)
  lr_wake%mode(n)%r_over_q     ! R/Q
  lr_wake%mode(n)%damp         ! d
  lr_wake%mode(n)%phi          ! Actually phi/2pi
  lr_wake%mode(n)%polar_angle  ! Polarization Angle
  lr_wake%mode(n)%polarized    ! Logical
  lr_wake%freq_spread          ! Frequency spread
  lr_wake%amp_scale            ! Amplitude scale
  lr_wake%time_scale           ! Time scale
\end{example}
\vn{freq_in} is the input frequency set in the \vn{lr_wake} structure and \vn{freq} is the actual
frequency set by \bmad. \vn{Freq} will only be different from \vn{freq_in} if \vn{freq_spread} is
nonzero. Example:
\begin{example}
  cav9[lr_wake%mode(2)%freq_in] = 1.1 * cav9[lr_wake%mode(2)%freq_in] ! Raise frequency by 10\%
\end{example}

%-----------------------------------------------------------------
\subsection{Long-Range Wakes -- Old Format}
\label{s:lr.wake.old}

There is an old, deprecated, format for specifying long-range wakes where the wake data is
contained in a separate file whose name is given by the \vn{lr_wake_file} element attribute.
Example:
\begin{example}
  abc: lcavity, lr_wake_file = "lr.wake", lr_freq_spread = 0.0023, lr_self_wake_on = F
\end{example}
The file gives the wake modes by specifying the frequency ($\omega/2\pi$), R/Q, Q, and m (order
number), and, optionally, the polarization angle $\theta_p/2\pi$ for each cavity mode. The input uses
Fortran90 namelist syntax: The data begins with the string \vn{\&long_range_modes} and ends with a
slash \vn{/}. Everything outside this is ignored. Each mode is labeled \vn{lr(i)} where \vn{i} is
the mode index. An example input file is:
\begin{example}
              Freq      R/Q      Q    m   Polar   b_sin  b_cos a_sin  a_cos  t_ref 
                      [Ohm/               Angle 
              [Hz]     m^(2m)]           [Rad/2pi]
  &long_range_modes
    lr(1) = 1.650e9    0.76    7.0e4  1    "unpol"
    lr(2) = 1.699e9   11.21    5.0e4  1    "0.15"
    lr(3) =    -1      0.57    1.1e6  0    "unpol"
  /
\end{example}
[Note: The quotation marks are needed with some compilers and not with others.]
If the polarization angle is set to ``\vn{unpolarized}'' the mode is taken to be unpolarized. 

Two deprecated element attributes that affect the long-range wake are \vn{lr_freq_spread} and
\vn{lr_self_wake_on} which correspond to \vn{lr_wake%freq_spread} and \vn{lr_wake%self_wake_on}
(\ref{s:lr.wake.syntax}

Notice that with the old format it is not possible to specify the phase offset $\phi$.

%-----------------------------------------------------------------
\section{Fringe Fields}
\label{s:fringe}
\index{fringe fields}

Some lattice elements can have fringe fields at the element edges. Whether \bmad tries to model the
fringe fields using the models described below first depends upon what kind of tracking is
done. Fringe effects are {\em not} applied when an element's \vn{tracking_method} is set to:
\begin{example}
  custom
  mad
\end{example}
Additionally, no fringe effects will be used if the \vn{tracking_method} is \vn{runge_kutta} or
\vn{time_runge_kutta}, and the element's \vn{field_calc} (\sref{s:integ}) is {\em not}
\vn{bmad_standard}. This is done since, for non-\vn{bmad_standard} field calculations, it is assumed
that the field profile includes the fringe regions.

%-----------------------------------------------------------------
\subsection{Turning On/Off Fringe Effects}
\label{s:fringe.at}

\index{fringe_at}
For elements that have a fringe, whether fringe fields are ignored or not is determined by the
setting of the \vn{fringe_at} element parameter. The possible settings are
\begin{example}
  no_end         
  both_ends             ! Default
  entrance_end
  exit_end
\end{example}
This is particularly useful in vetoing the fringe effect in the interior of split elements. The
setting of attributes like \vn{fringe_type} (see below) are ignored for boundaries where the fringe
has been turned off.

\index{spin_fringe_on}
When a particle's spin is being tracked, the \vn{spin_fringe_on} logical attribute of the element
determines how the spin tracking is handled through a fringe region. Example:
\begin{example}
  q: quad, spin_fringe_on = T, fringe_at = exit_end
\end{example}
Here, there is no fringe effect at the entrance of the element and the fringe at the exit end of the
element will affect the spin. The default setting of \vn{spin_fringe_on} is \vn{True}. The
\vn{spin_fringe_on} attribute is useful for examining how much fringe fields affect spin precession.

%-----------------------------------------------------------------
\subsection{Fringe Types}
\label{s:fringe.type}

\index{fringe_type}
\bmad and PTC have several fringe field models for static magnetic fields. Which fringe model is
used is set by two element attributes: \vn{ fringe_type} and \vn{ptc_fringe_geometry} with
\vn{ptc_fringe_geometry} only being used with bends and PTC dependent tracking.
The \vn{fringe_type} switch is used to select how a fringe field is
simulated.  The possible settings of \vn{fringe_type} are:
\begin{example}
  none              ! No fringe effect. 
  soft_edge_only
  hard_edge_only
  full
  linear_edge       ! Sbend, rbend only.
  basic_bend        ! Sbend, rbend only. 
  sad_full          ! Sbend, rbend, sad_mult only.
\end{example}
Default settings for \vn{fringe_type} are
\newline\hspace*{0.1in}
{ \tt
\begin{tabular}{ll}
  Element Type        & Default fringe_type  \\ \midrule
  e_gun               & full                 \\
  lcavity, rfcavity   & full                 \\
  sbend, rbend        & basic_bend           \\
  All others          & none                 \\
  \bottomrule
\end{tabular}
}

Some fringe fields can be divided into two pieces.  The first piece is called the \vn{hard edge}
fringe kick and is the kick in the limit that the longitudinal extent of the fringe is zero. The
second piece is the \vn{soft edge} fringe kick which is the fringe kick with the fringe having a
finite longitudinal extent minus the hard edge fringe kick. That is
\begin{example}
  fringe kick = hard fringe kick + soft fringe kick
\end{example}
The advantage of separating the fringe kick in this way is that the hard fringe can be used without
having to know anything about the longitudinal extent of the fringe (which happens when simulating
magnets that have not yet been fully designed).  In many cases, this is a good enough
approximation. Note that using the soft fringe without the hard fringe is not physical but can be
useful in understanding how the soft edge component affects tracking. See~\sref{s:fringe.std} for
details.

For bend elements the following fringe maps are used: 
\newline\hspace*{0.1in}
{{ \tt
\begin{tabular}{ll}
  fringe_type     & Fringe model    \\ \midrule
  none            & No fringe effect                                  \\
  soft_edge_only  & SAD dipole soft edge (\sref{s:fringe.bend.soft})  \\
  hard_edge_only  & Bend Second Order (\sref{s:fringe.bend.hard}) with \vn{fint} and \vn{hgap} ignored. \\
  full            & Exact bend                                        \\
  linear_edge     & Linear dipole hard edge (\sref{s:lin.dip.fringe}) \\
  basic_bend      & Bend Second Order (\sref{s:fringe.bend.hard})     \\
  sad_full        &  \\
  \bottomrule
\end{tabular}
\hfill\vspace*{0.1in}
}\newline}
The \vn{basic_bend} setting for bend elements, which is the default, is essentially the basic
vertical focusing effect that is present when there is a finite \vn{e1} or \vn{e2} face angle. With
\vn{bmad_standard} tracking, \vn{basic_bend} also includes second order terms
(\sref{s:fringe.bend.hard}). The \vn{linear_edge} setting ignores these second order terms.  In some
cases, for instance in a chicane, \vn{basic_bend} is not good enough. With \vn{fringe_type} set to
\vn{full}, higher order effects are taken into account.

PTC does not have a \vn{linear_edge} fringe model.
With PTC tracking, \vn{basic_bend} tracking is used if \vn{linear_edge} is chosen.

Additionally, for use with PTC, the \vn{ptc_fringe_geometry} switch can be used
to define the symmetry of the fringe fields. Possible settings are:
\begin{example}
  x_invariant
  multipole_symmetry
\end{example}
\index{max_fringe_order}
The difference between \vn{x_invariant} and \vn{multipole_symmetry} is that with
\vn{multipole_symmetry} the fringe field for an $n$\Th order multipole is assumed to have the same
rotational symmetry as the multipole. With this assumption, the fringe field has $n+1$\St order
terms.  With \vn{x_invariant}, the fringe field is calculated assuming that there is translational
invariance along the horizontal $x$ axis. This differs from \vn{multipole_symmetry} by adding terms
of increasing order consistent with the translational invariance. See \'Etienne Forest's
book\cite{b:forest} for more details. Which setting of \vn{ptc_fringe_geometry} is appropriate
depends upon how the dipole under consideration is constructed. The two settings of
\vn{ptc_fringe_geometry} represent two points of a continuum of possible fringe field geometries.

When using PTC tracking (\sref{s:ptc.intro}), the \vn{ptc_com[max_fringe_order]}
(\sref{s:param}) determines the maximum order of the calculated fringe fields.

Example:
\begin{example}
  b1: rbend, angle = pi/4, g = 0.3, fringe_type = full
\end{example}

\index{sad}\index{sbend}\index{rbend}
The \vn{soft_edge_only}, \vn{hard_edge_only} and \vn{sad_full} settings of \vn{fringe_type} emulate
the fringe field tracking used in the SAD program\cite{b:sad}.  The \vn{soft_edge_only} setting only
uses the linear part of the fringe, \vn{hard_edge_only} ignores the linear part of the fringe, and
\vn{sad_full} uses the full fringe.  For an \vn{sbend} or \vn{rbend} element, these SAD fringe
fields are in addition to the fringe fields that occurs with a finite \vn{e1} or \vn{e2} face angle.

Equivalent settings of SAD \vn{fringe} and \vn{disfrin} for \vn{sbend} and \vn{rbend} elements:
\newline\hspace*{0.1in}
{ \tt
\begin{tabular}{lll}
  fringe_type       & fringe & disfrin \\ \midrule
  soft_edge_only    &  1     & 1       \\
  hard_edge_only    &  0     & 0       \\
  basic_bend        &  0     & 1       \\
  sad_full          &  1     & 0       \\
  \bottomrule
\end{tabular}
}

For \vn{quadrupole} and \vn{sad_mult} elements, the translation between the \vn{fringe_at}
and \vn{fringe_type} settings and the \vn{fringe} and \vn{disfrin} switches of SAD is:
\newline\hspace*{0.1in}
{{ \tt
\begin{tabular}{lllll} 
                       & \multicolumn{4}{c}{fringe_at:}                               \\
  fringe_type          & no_end       & entrance_end   & exit_end     & both_ends$^*$ \\ \midrule
  none                 & [0, $\ne 0$] & [0, $\ne 0$]   & [0, $\ne 0$] & [0, $\ne 0$]  \\
  soft_edge_only       & [0, $\ne 0$] & [1, $\ne 0$]   & [2, $\ne 0$] & [3, $\ne 0$]  \\
  hard_edge_only$^*$   & [0, $\ne 0$] & No SAD Equiv   & No SAD Equiv & [0, $= 0$]    \\
  full                 & [0, $\ne 0$] & [1, $= 0$]     & [2, $= 0$]   & [3, $= 0$]    \\ 
  \bottomrule
  $^*$Default value.   & \multicolumn{4}{l}{\textbf{[fringe, disfrin]}} \\
\end{tabular}
\hfill\vspace*{0.1in}
}\newline}
Each entry is the table is of the form [\vn{fringe}, \vn{disfrin}]. The \vn{soft_edge_only} fringe
kick is a kick that is linear in the transverse $(x, p_x, y, p_y)$ coordinates and comes from the
finite width of the quadrupolar fringe field. The width of the quadrupolar fringe field is
characterized by the \vn{f1} and \vn{f2} attributes. The \vn{sad_nonlinear_only} fringe kick comes
from the nonlinear part of the quadrupolar field plus the fringes of the other multipoles.

The soft fringe quadrupole parameters \vn{fq1} and \vn{fq2} (\sref{s:q.soft}) are related
to the corresponding SAD parameters \vn{f1} and \vn{f2} via
\begin{example}
  f1 = -sign(fq1) * sqrt(24 * |fq1|)
  f2 = fq2
\end{example}
In the SAD documentation, the soft edge is called the ``linear'' fringe.

\index{permfringe}\index{bendfringe}
For programmers who deal with PTC directly: The translation between \vn{ptc_fringe_geometry} on the
\bmad side and \vn{bendfringe} on the PTC side is:
\newline\hspace*{0.1in}
{
\begin{tabular}{lll} \toprule 
  ptc_fringe_geometry     & bendfringe \\ \midrule
  x_invariant             & True       \\
  multipole_symmetry      & False      \\ 
  \bottomrule
\end{tabular}
}

%-----------------------------------------------------------------
\section{Instrumental Measurement Attributes}
\label{s:meas.attrib}

\index{instrument}\index{monitor}\index{marker}\index{detector}
\index{x_gain_err}\index{y_gain_err}\index{crunch}\index{noise}
\index{x_gain_calib}\index{y_gain_calib}\index{crunch_calib}
\index{x_offset}\index{y_offset}\index{tilt}
\index{x_offset_calib}\index{y_offset_calib}\index{tilt_calib}
\index{de_eta_meas}\index{n_sample}\index{osc_amplitude}

\vn{instrument}, \vn{monitor}, \vn{detector}, and \vn{marker} elements have special attributes to
describe errors associated with orbit, betatron phase, dispersion and coupling measurements. These
attributes are: \hfill\break \hspace*{0.1in}
\begin{tabular}{lll} \toprule
  {\em Attribute}           & {\em Symbol} (See: \sref{s:meas.calc}) & \\ \midrule
  \vn{tilt}                 & $\theta_t$               & See \sref{s:offset} \\ 
  \vn{x_offset}             & $x_{\text{off}}$         & See \sref{s:offset} \\ 
  \vn{y_offset}             & $y_{\text{off}}$         & See \sref{s:offset} \\ 
  \vn{x_gain_err}           & $dg_{x,\text{err}}$      & Horizontal gain error \\ 
  \vn{y_gain_err}           & $dg_{y,\text{err}}$      & Vertical gain error \\ 
  \vn{crunch}               & $\psi_{\text{err}}$      & Crunch angle \\ 
  \vn{tilt_calib}           & $\theta_{\text{err}}$    & tilt angle calibration \\ 
  \vn{x_offset_calib}       & $x_{\text{calib}}$       & Horizontal offset calibration \\ 
  \vn{y_offset_calib}       & $y_{\text{calib}}$       & Vertical offset calibration \\ 
  \vn{x_gain_calib}         & $dg_{x,\text{calib}}$    & Horizontal gain calibration \\ 
  \vn{y_gain_calib}         & $dg_{y,\text{calib}}$    & Vertical gain calibration \\ 
  \vn{crunch_calib}         & $\psi_{\text{calib}}$    & Crunch angle calibration \\ 
  \vn{noise}                & $n_f$                    & Noise factor \\ 
  \vn{de_eta_meas}          & $dE/E$                   & Percent change in energy \\ 
  \vn{x_dispersion_err}     & $\eta_{x,\text{err}}$    & Horizontal dispersion error \\ 
  \vn{y_dispersion_err}     & $\eta_{y,\text{err}}$    & Vertical dispersion error \\ 
  \vn{x_dispersion_calib}   & $\eta_{x,\text{calib}}$  & Horizontal dispersion calibration \\ 
  \vn{y_dispersion_calib}   & $\eta_{y,\text{calib}}$  & Vertical dispersion calibration \\ 
  \vn{n_sample}             & $N_s$                    & Number of sampling points \\ 
  \vn{osc_amplitude}        & $A_{\text{osc}}$         & Oscillation amplitude \\ \bottomrule
\end{tabular}
\hfill\break A program can use these quantities to calculate ``measured'' values from the
``laboratory'' values. Here, ``laboratory'' means as calculated from some model lattice.  See
\sref{s:meas.calc} for the conversion formulas.

\end{document}